\documentclass{book}
%\documentclass{article}                            %for shorter notes
\usepackage{graphicx}                              %for PNG images (pdflatex)
%\usepackage{graphics}                              %for EPS images (latex)
\usepackage[linkbordercolor={1.0 1.0 0.0}]{hyperref} %for \url tag
\usepackage{color}                                 %for defining custom colors
\usepackage{framed}                                %for shaded and framed paragraphs
\usepackage{textcomp}                              %for various symbols, e.g. Registered Mark
\usepackage{geometry}                              %for defining page size
\usepackage{longtable}                             %for breaking tables
%
\geometry{verbose,a4paper,tmargin=2.5cm,bmargin=2.5cm,lmargin=2.5cm,rmargin=2cm}
\hypersetup{
  pdfauthor = {Author Name},
  pdftitle = {Paper title},
  pdfsubject = {Paper subject},
  pdfkeywords = {Paper,keyword,comma-separated},
  pdfcreator = {PDFLaTeX with hyperref package},
  pdfproducer = {PDFLaTeX}
}
%
\bibliographystyle{IEEEtran}                       %a nice bibliography style
%
\def\efill{\hfill\nopagebreak}%
\hyphenation{Nordu-Grid}
\setlength{\parindent}{0cm}
\setlength{\FrameRule}{1pt}
\setlength{\FrameSep}{8pt}
\addtolength{\parskip}{5pt}
\renewcommand{\thefootnote}{\fnsymbol{footnote}}
\renewcommand{\arraystretch}{1.3}
\newcommand{\dothis}{\colorbox{shadecolor}}
\newcommand{\globus}{Globus Toolkit\textsuperscript{\textregistered}~2~}
\newcommand{\GT}{Globus Toolkit\textsuperscript{\textregistered}}
\newcommand{\ngdl}{\url{http://ftp.nordugrid.org/download}~}
\definecolor{shadecolor}{rgb}{1,1,0.6}
\definecolor{salmon}{rgb}{1,0.9,1}
\definecolor{bordeaux}{rgb}{0.75,0.,0.}
\definecolor{cyan}{rgb}{0,1,1}
%
%----- DON'T CHANGE HEADER MATTER
\begin{document}
\def\today{\number\day/\number\month/\number\year}

\begin{titlepage}

\begin{tabular}{rl}
\resizebox*{3cm}{!}{\includegraphics{ng-logo.png}}
&\parbox[b]{2cm}{\textbf \it {\hspace*{-1.5cm}NORDUGRID\vspace*{0.5cm}}}
\end{tabular}

\hrulefill

%-------- Change this to NORDUGRID-XXXXXXX-NN

{\raggedleft NORDUGRID-XXXXXXX-NN\par}

{\raggedleft \today\par}

\vspace*{2cm}

%%%%---- The title ----
{\centering \textsc{\Large Paper title}\Large \par}
\vspace*{0.5cm}
    
%%%%---- A subtitle, if necessary ----
{\centering \textit{\large Paper subtitle}\large \par}
    
\vspace*{1.5cm}
%%%%---- A list of authors ----
    {\centering \large Paper author\footnote{authors@address} \large \par}
    
%%%%---- An abstract - if style is article ----
%\begin{abstract}
%The abstract
%\end{abstract}
\end{titlepage}

\tableofcontents                          %Comment if use article style
\newpage
\chapter{Preface}
\label{sec:intro}

This document describes from a technical viewpoint a plugin based client library for the new 
Web Service (WS) based Advanced Resource Connector (ARC) middleware. The library is capable of 
handling resource discovery, job submission and job control and interoperates with several 
different grid flavours. The support for different grid flavours is enabled through the modular 
design using plugins specialized for each supported middleware, and presently supported are ARC0 
(ARC classic), ARC1 and the gLite middlewares. Future extensions to support additional middlewares 
involve plugin compilation only i.e. no recompilation of main libraries or client.

Using the library a set of command line tools have been built which puts the library's functionality 
at the fingertips of users. While this documentation will illustrate how such command line tool can be 
built, the main documention of the command line tools is given in the client user manual [ref].

Finally the library has been designed with care for later usage within a graphical user interface. This 
particularly applies to the job management part.

\chapter{Resource Discovery and Information Retrieval}
\label{sec:TargetDiscovery}
Almost every day a new grid cluster is purchased and made available around the world (either to the 
general public or restricted to authorized users). More available clusters create a need for better and 
faster resource discovery and information retrieval, making this task an utmost important component of the 
new client library.

The new client library's resource discovery and information retrieval component consists of three classes, one 
of which a base class for further grid middleware specific specialization (plugin):

\begin{itemize}

\item{{\bf TargetGenerator} is responsible for loading TargetRetriever plugins in accordance with the input URLs. 
e.g. if an URL pointing to an ARC1 resource is given the TargetRetrieverARC1 is loaded. If no URLs are given, the 
TargetGenerator makes use of the user/system defaults. When queried for resources (``targets'') the TargetGenerator 
passes the request to the loaded TargetRetrievers running them as individual threads for improved performance. The 
TargetGenerator keeps records of by TargetRetriever found services (index of computing) in order to avoid multiple 
queries to the same resource. Accepted targets are registered in the FoundTargets array of the TargetGenerator.}

\item{{\bf TargetRetriever} is base class for TargetRetriever grid middleware specializations. Contains the pure 
virtual method GetTargets which is to be implemented by the specialized class. While it is not mandatory it is recommended 
that the specialized class divides this function into two components: QueryIndex and InterrogateTarget. The former is 
responsible for querying index servers, and if the result yields an addresss to a different index server then the 
specialized TargetRetriever should call itself recursively with the found url if the TargetGenerator allows. Discovered 
computing resources are sent to the InterrogateTarget function (or if the TargetRetriever was initialized with a computing 
service url then it should jump directly to this point) which collects the detailed information about the resource.}

\item{{\bf ExecutionTarget} is the class representation of a computing resource (queue) capable of running a grid job. It
serves as input to the broker which is foreseen to be able to select between different ExecutionTargets from different 
grid flavours without apriori knowing their difference. The ExecutionTarget class mimics the glue2 information model with 
a flattened structure, and 

Consequently a mapping a is needed
 
Mapping should be 1 queue equals 1 ExectutionTarget ()}

\end{itemize}

%\begin{figure}[ht]
%\centering{{\scalebox{0.9}{\includegraphics{ng-logo.png}}}
%\caption{\label{fig:myfigure1}The figure shows a logo.} }
%\end{figure}

\chapter{Job Management}
\label{sec:JobManagement}

\chapter{Data Management}
\label{sec:DataManagement}

\chapter{Specialized Classes (Grid Flavour Plugins)}
\label{sec:plugins}
\section{ARC0 Plugins}
\section{ARC1 Plugins}
\section{gLite Plugins}

\chapter{Building Command Line Interfaces}
\label{sec:cli}

\bibliography{grid}
\end{document}
