%\documentclass{book}
\documentclass{article}                            %for shorter notes
\usepackage{graphicx}                              %for PNG images (pdflatex)
%\usepackage{graphics}                              %for EPS images (latex)
\usepackage[linkbordercolor={1.0 1.0 0.0}]{hyperref} %for \url tag
\usepackage{color}                                 %for defining custom colors
\usepackage{framed}                                %for shaded and framed paragraphs
\usepackage{textcomp}                              %for various symbols, e.g. Registered Mark
\usepackage{geometry}                              %for defining page size
\usepackage{longtable}                             %for breaking tables
%\usepackage{bibtex}
\usepackage{moreverb}
%
\geometry{verbose,a4paper,tmargin=2.5cm,bmargin=2.5cm,lmargin=2.5cm,rmargin=2cm}
\hypersetup{
  pdfauthor = {A.U. Thors},
  pdftitle = {WS-Based ARC Clients},
  pdfsubject = {Paper subject},
  pdfkeywords = {Paper,keyword,comma-separated},
  pdfcreator = {PDFLaTeX with hyperref package},
  pdfproducer = {PDFLaTeX}
}
%
\bibliographystyle{IEEEtran}                       %a nice bibliography style
%
\def\efill{\hfill\nopagebreak}%
\hyphenation{Nordu-Grid}
\setlength{\parindent}{0cm}
\setlength{\FrameRule}{1pt}
\setlength{\FrameSep}{8pt}
\addtolength{\parskip}{5pt}
\renewcommand{\thefootnote}{\fnsymbol{footnote}}
\renewcommand{\arraystretch}{1.3}
\newcommand{\dothis}{\colorbox{shadecolor}}
\newcommand{\globus}{Globus Toolkit\textsuperscript{\textregistered}~2~}
\newcommand{\GT}{Globus Toolkit\textsuperscript{\textregistered}}
\newcommand{\ngdl}{\url{http://ftp.nordugrid.org/download}~}
\newcommand{\uicommand}[1]{\textbf{\sffamily #1}}
\definecolor{shadecolor}{rgb}{1,1,0.6}
\definecolor{salmon}{rgb}{1,0.9,1}
\definecolor{bordeaux}{rgb}{0.75,0.,0.}
\definecolor{cyan}{rgb}{0,1,1}
%
%----- DON'T CHANGE HEADER MATTER
\begin{document}
\def\today{\number\day/\number\month/\number\year}

\begin{titlepage}

\begin{tabular}{rl}
\resizebox*{3cm}{!}{\includegraphics{ng-logo.png}}
&\parbox[b]{2cm}{\textbf \it {\hspace*{-1.5cm}NORDUGRID\vspace*{0.5cm}}}
\end{tabular}

\hrulefill

%-------- Change this to NORDUGRID-XXXXXXX-NN

{\raggedleft NORDUGRID-TECH-20\par}

{\raggedleft \today\par}

\vspace*{2cm}

%%%%---- The title ----
{\centering \textsc{\Large WS-Based ARC Clients}\Large \par}
\vspace*{0.5cm}
    
%%%%---- A subtitle, if necessary ----
%{\centering \textit{\large Subtitile}\large \par}
    
\vspace*{1.5cm}
%%%%---- A list of authors ----
{\centering \large A.U. Thors\footnote{authors@gmail.com} \large \par}
    
%%%%---- An abstract - if style is article ----
\begin{abstract}
This NorduGrid Technical document gives an introduction on how to use
the client tools developed for WS-based ARC solution.
\end{abstract}
\end{titlepage}

\tableofcontents                          %Comment if use article style
\newpage
%\chapter{Preface}
\section{Introduction}                    %Use Sections for articles
\label{sec:intro}

This document describes a set of simple client tools for the new Web
Service~\cite{ws} (WS) based Advanced Resource Connector~\cite{arc}
(ARC) middleware. These tools are prototypes and are intended to be a
proof-of-concept for the new ARC.

So far, the only service that is available for the new ARC is the ARC
Resource-Coupled Execution Service~\cite{arex} (A-REX). Therefore,
this prototype release of WS clients consists of job handling tools
with the following functionality:
\begin{itemize}
\item submission of jobs;
\item querying the status of jobs; and
\item termination of jobs.
\end{itemize}

The names of the command line tools begin with ``ap'', which shall be
interpreted as ``ARC Prototype'' to distinguish them from the
full-fledged production tools.

Support for staging of data would also have been desirable, but this
is not yet fully implemented on the server side. Therefore it could
not be included in this client prototype either. Thus, it is not
possible to run any jobs that depend on local files on the computer on
which the submitting client runs. Neither is it possible to retrieve
any form of output (not even stdout/stderr) by using the new client.

Behind the command line tools, there is a minimalistic library which
is documented in Appendix~\ref{app:library}. The library will serve as
a basis for the development of a generalized ARClib. This will be used
by the new client tools of the production releases of the next
generation of ARC.


\section{Configuration}
\label{sec:config}

The tools described here do not perform any brokering, but submit jobs
to a specific A-REX service. Which A-REX service to use is specified
in a configuration file. The configuration file also specifies how to
set up the communication chain for the client. The location of the
configuration file is specified by the environment variable
\verb:ARC_AREX_CONFIG:. If there is no such environment variable, the
configuration file is assumed to be \verb:arex_client.xml: in the
current working directory. An example configuration file is shown in
Appendix~\ref{app:config}.


\section{Job Submission}
\label{sec:arcsub}

The \verb#arcsub# command is used to submit jobs to an A-REX service.

\begin{shaded}
\paragraph{Usage} \verb:arcsub <JSDL-file> <JobID-file>:
\end{shaded}

\paragraph{Arguments}
\begin{description}
\item[\texttt{<JSDL-file>}] The name of a file that contains
the job specification in JSDL format.
\item[\texttt{<JobID-file>}] The name of a file in which the
Job ID will be stored. This file is created when the command is
used. The reason for storing the Job ID in a file is its length
(approximately 500 characters) which makes it unsuitable to use as a
command line argument for the other tools.
\end{description}

\paragraph{Output} A message that the job was submitted and in which
file the Job ID is stored if the submission succeded. If the
submission failed for some reason, an error message is displayed.


\section{Job Status Query}
\label{sec:arcstat}

The \verb#arcstat# command is used to query the status of a job that
has been submitted to an A-REX service.

\begin{shaded}
\paragraph{Usage} \verb:arcstat <JobID-file>:
\end{shaded}

\paragraph{Arguments}
\begin{description}
\item[\texttt{<JobID-file>}] The name of a file in which the
Job ID is stored. This file was created by \verb:arcsub: when the job
was submitted.
\end{description}

\paragraph{Output} The status of the job, e.g. ``Pending/Accepted'',
or an error message if the status request failed for some reason.


\section{Job Termination}
\label{sec:arckill}

The \verb#arckill# command is used to terminate a job that has been
submitted to an A-REX service.

\begin{shaded}
\paragraph{Usage} \verb:arckill <JobID-file>:
\end{shaded}

\paragraph{Arguments}
\begin{description}
\item[\texttt{<JobID-file>}] The name of a file in which the
Job ID is stored. This file was created by \verb:arcsub: when the job
was submitted.
\end{description}

\paragraph{Output} A message that the job was terminated, or an error
message if the termination failed for some reason.


\appendix


\section{The A-REX Client Library}
\label{app:library}

Besides the three command line tools described earlier in this
document, there is also a small library on which the command line
tools are based. The library consists of two classes; the
\verb:AREXClient: class and the \verb:AREXClientError: class.

\subsection{The \texttt{AREXClient} class}

This class is a client class for the A-REX service. It provides
methods for three operations on an A-REX service:
\begin{itemize}
\item Job submission
\item Job status query
\item Job termination
\end{itemize}


\subsubsection{Constructor}

\begin{shaded}
\verb#AREXClient(std::string configFile="")#
\end{shaded}

This is the constructor for the AREXClient class. It creates an A-REX
client that corresponds to a specific A-REX service, which is
specified in a configuration file. The configuration file also
specifies how to set up the communication chain for the client. The
location of the configuration file can be provided as a parameter to
the method. If no such parameter is given, the environment variable
\verb:ARC_AREX_CONFIG: is assumed to contain the location. If there is
no such environment variable, the configuration file is assumed to be
\verb:arex_client.xml: in the current working directory.

\paragraph{Parameters}
\begin{description}
\item[\texttt{configFile}] The location of the configuration file.
\end{description}

\paragraph{Throws}

An \verb:AREXClientError: object if an error occurs.


%\subsubsection{Destructor}

%\begin{shaded}
%\verb#~AREXClient()#
%\end{shaded}

%This is the destructor. It does what destructors usually do, cleans up.


\subsubsection{Job submission}

\begin{shaded}
\verb#std::string submit(std::istream& jsdl_file)#
\end{shaded}

This method submits a job to the A-REX service corresponding to this
client instance.

\paragraph{Parameters}
\begin{description}
\item[\texttt{jsdl\_file}] An input stream from which the JSDL file for
the job can be read.
\end{description}

\paragraph{Returns}
The Job ID of the the submitted job.

\paragraph{Throws}
An \verb:AREXClientError: object if an error occurs.      


\subsubsection{Job status query}

\begin{shaded}
\verb#std::string stat(const std::string& jobid)#
\end{shaded}

This method queries the A-REX service about the status of a
job.

\paragraph{Parameters}
\begin{description}
\item[\texttt{jobid}] The Job ID of the job.
\end{description}

\paragraph{Returns}
The status of the job.

\paragraph{Throws}
An \verb:AREXClientError: object if an error occurs.      


\subsubsection{Job termination}

\begin{shaded}
\verb#void kill(const std::string& jobid)#
\end{shaded}

This method sends a request to the A-REX service to terminate a job.

\paragraph{Parameters}
\begin{description}
\item[\texttt{jobid}] The Job ID of the job.
\end{description}

\paragraph{Throws}
An \verb:AREXClientError: object if an error occurs.      


\subsection{The \texttt{AREXClientError} class}

This is an exception class that is used to handle runtime errors
discovered in the AREXClient class.

\paragraph{Inherits}
This class inherits \verb#std::runtime_error#


\subsubsection{Constructor}

\begin{shaded}
\verb#AREXClientError(const std::string& what="")#
\end{shaded}

This is the constructor of the AREXClientError class.

\paragraph{Parameters}
\begin{description}
\item[\texttt{what}] An explanation of the error.
\end{description}



\section{Configuration Example}
\label{app:config}
Figure~\ref{fig:config} shows an example configuration file, which is
an XML file. All configuration information is enclosed within an
\verb:ArcConfig: element. That element also specifies which XML
manespaces to use.

The \verb:ModuleManager: element contains paths to directories where
loadable plugins can be found. In this case, there is only one path --
\verb:/usr/lib/arc/: -- but there may be many.

The \verb:Plugins: elements specify which plugins to load -- in this
case message chain components (MCCs) for TCP, TLS, HTTP and SOAP.

The \verb:Chain: element, finally, defines how to combine the MCCs
that have been loaded into a communication chain. In this case, a TCP
MCC that connects to port 60000 on the local host is connected to a TLS
MCC, which is connected to a HTTP that uses the POST method, and
finally a SOAP MCC is connected to the HTTP MCC.

The client tools will use the SOAP MCC as an entry point when sending
messages. The messages will then propagate through the MCCs in the
chain as specified in the configuration file.

\begin{figure}
\begin{shaded}
\verbatiminput{arex_client.xml}
\end{shaded}
\caption{An examle of an \texttt{arex\_client.xml} configuration file.}
\label{fig:config}
\end{figure}


\clearpage
\bibliography{grid}


\end{document}
