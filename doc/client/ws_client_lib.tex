\section{The A-REX Client Library}
\label{app:library}

Besides the three command line tools described earlier in this
document, there is also a small library on which the command line
tools are based. The library consists of two classes; the
\verb:AREXClient: class and the \verb:AREXClientError: class.

\subsection{The \texttt{AREXClient} class}

This class is a client class for the A-REX service. It provides
methods for three operations on an A-REX service:
\begin{itemize}
\item Job submission
\item Job status query
\item Job termination
\end{itemize}


\subsubsection{Constructor}

\begin{shaded}
\verb#AREXClient(std::string configFile="")#
\end{shaded}

This is the constructor for the AREXClient class. It creates an A-REX
client that corresponds to a specific A-REX service, which is
specified in a configuration file. The configuration file also
specifies how to set up the communication chain for the client. The
location of the configuration file can be provided as a parameter to
the method. If no such parameter is given, the environment variable
\verb:ARC_AREX_CONFIG: is assumed to contain the location. If there is
no such environment variable, the configuration file is assumed to be
\verb:arex_client.xml: in the current working directory.

\paragraph{Parameters}
\begin{description}
\item[\texttt{configFile}] The location of the configuration file.
\end{description}

\paragraph{Throws}

An \verb:AREXClientError: object if an error occurs.


%\subsubsection{Destructor}

%\begin{shaded}
%\verb#~AREXClient()#
%\end{shaded}

%This is the destructor. It does what destructors usually do, cleans up.


\subsubsection{Job submission}

\begin{shaded}
\verb#std::string submit(std::istream& jsdl_file)#
\end{shaded}

This method submits a job to the A-REX service corresponding to this
client instance.

\paragraph{Parameters}
\begin{description}
\item[\texttt{jsdl\_file}] An input stream from which the JSDL file for
the job can be read.
\end{description}

\paragraph{Returns}
The Job ID of the the submitted job.

\paragraph{Throws}
An \verb:AREXClientError: object if an error occurs.      


\subsubsection{Job status query}

\begin{shaded}
\verb#std::string stat(const std::string& jobid)#
\end{shaded}

This method queries the A-REX service about the status of a
job.

\paragraph{Parameters}
\begin{description}
\item[\texttt{jobid}] The Job ID of the job.
\end{description}

\paragraph{Returns}
The status of the job.

\paragraph{Throws}
An \verb:AREXClientError: object if an error occurs.      


\subsubsection{Job termination}

\begin{shaded}
\verb#void kill(const std::string& jobid)#
\end{shaded}

This method sends a request to the A-REX service to terminate a job.

\paragraph{Parameters}
\begin{description}
\item[\texttt{jobid}] The Job ID of the job.
\end{description}

\paragraph{Throws}
An \verb:AREXClientError: object if an error occurs.      


\subsection{The \texttt{AREXClientError} class}

This is an exception class that is used to handle runtime errors
discovered in the AREXClient class.

\paragraph{Inherits}
This class inherits \verb#std::runtime_error#


\subsubsection{Constructor}

\begin{shaded}
\verb#AREXClientError(const std::string& what="")#
\end{shaded}

This is the constructor of the AREXClientError class.

\paragraph{Parameters}
\begin{description}
\item[\texttt{what}] An explanation of the error.
\end{description}
