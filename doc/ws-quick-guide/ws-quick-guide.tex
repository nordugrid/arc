%\documentclass{book}
\documentclass{article}                            %for shorter notes
\usepackage{graphicx}                              %for PNG images (pdflatex)
%\usepackage{graphics}                              %for EPS images (latex)
\usepackage[linkbordercolor={1.0 1.0 0.0}]{hyperref} %for \url tag
\usepackage{color}                                 %for defining custom colors
\usepackage{framed}                                %for shaded and framed paragraphs
\usepackage{textcomp}                              %for various symbols, e.g. Registered Mark
\usepackage{geometry}                              %for defining page size
\usepackage{longtable}                             %for breaking tables
\usepackage{listings}                                %for code listings
\lstloadlanguages{bash,XML}
%
\geometry{verbose,a4paper,tmargin=2.5cm,bmargin=2.5cm,lmargin=2.5cm,rmargin=2cm}
\hypersetup{
  pdfauthor = {NorduGrid},
  pdftitle = {ARC Web Services Quick Usage Guide},
  pdfsubject = {ARC Grid middleware},
  pdfkeywords = {Grid,middleware,ARC,KnowARC,NorduGrid},
  pdfcreator = {PDFLaTeX with hyperref package},
  pdfproducer = {PDFLaTeX}
}
%
\bibliographystyle{IEEEtran}                       %a nice bibliography style
%
\def\efill{\hfill\nopagebreak}%
\hyphenation{Nordu-Grid}
\setlength{\parindent}{0cm}
\setlength{\FrameRule}{1pt}
\setlength{\FrameSep}{8pt}
\addtolength{\parskip}{5pt}
\renewcommand{\thefootnote}{\fnsymbol{footnote}}
\renewcommand{\arraystretch}{1.3}
\newcommand{\dothis}{\colorbox{shadecolor}}
\newcommand{\globus}{Globus Toolkit\textsuperscript{\textregistered}~4~}
\newcommand{\GT}{Globus Toolkit\textsuperscript{\textregistered}}
\newcommand{\ngdl}{\url{http://download.nordugrid.org/}~}
\definecolor{shadecolor}{rgb}{1,1,0.6}
\definecolor{salmon}{rgb}{1,0.9,1}
\definecolor{bordeaux}{rgb}{0.75,0.,0.}
\definecolor{cyan}{rgb}{0,1,1}
\definecolor{lightblue}{cmyk}{0.05,0,0,0.05}
\lstset{
frame=single,
backgroundcolor=\color{shadecolor},
fillcolor=\color{shadecolor},
rulecolor=\color{lightblue},
basicstyle=\ttfamily,
xleftmargin=0.05\linewidth,
aboveskip=2\medskipamount
}
%
%----- DON'T CHANGE HEADER MATTER
\begin{document}
\def\today{\number\day/\number\month/\number\year}

\begin{titlepage}

\begin{tabular}{rl}
\resizebox*{3cm}{!}{\includegraphics{ng-logo.png}}
&\parbox[b]{2cm}{\textbf \it {\hspace*{-1.5cm}NORDUGRID\vspace*{0.5cm}}}
\end{tabular}

\hrulefill

%-------- Change this to NORDUGRID-XXXXXXX-NN

{\raggedleft NORDUGRID-XXXXXXX-NN\par}

{\raggedleft \today\par}

\vspace*{2cm}

%%%%---- The title ----
{\centering \textsc{\Large ARC Web Services Quick Usage Guide}\Large \par}
\vspace*{0.5cm}

%%%%---- A subtitle, if necessary ----
%{\centering \textit{\large Paper subtitle}\large \par}

\vspace*{1.5cm}
%%%%---- A list of authors ----
%    {\centering \large A. Konstantinov\footnote{const@takas.lt} \large \par}

%%%%---- An abstract - if style is article ----
%\begin{abstract}
%The abstract
%\end{abstract}
\end{titlepage}

\tableofcontents                          %Comment if use article style
\newpage
%\chapter{Preface}


\section{ARC Middleware}                    %Use Sections for articles
\label{sec:arc-intro}

The Advanced Resource Connector (ARC) middleware~\cite{arc}, introduced by
NorduGrid~\cite{nordugrid}, is an open source software solution
enabling production quality computational and data grids.
Since the first release (May 2002) the middleware has been deployed and
been used in production environments. Emphasis is put on scalability,
stability, reliability and performance of the middleware. A growing
number of grid projects, like NDGF~\cite{ndgf}, Swegrid~\cite{swegrid} or Swiss
National Grid Association~\cite{swing} have chosen ARC as their middleware.

This document provides a quick usage guide to the new Web Service (WS)
components of ARC, often referred to as ARC1 components. WS design of ARC is
based on a service container -- the \textit{Hosting Environment Daemon (HED)},
and various capabilities of the grid are implemented as distinct Web Services
that reside in HED.

At the time of writing, this comprises the following components:
\begin{itemize}
\item an OGSA BES~\cite{ogsa-bes} compliant execution service --
\textit{ARC Resource-coupled EXecution service (A-REX)}~\cite{arex}
\item an \textit{Echo} service (for testing purposes)
\end{itemize}

\section{Dependencies}
\label{sec:deps}

The core part of middleware is written in C/C++. To build the software
from source or installing a precompiled binary requires the prior installation
of multiple external packages. Client and server packages differ in their
dependencies to some degree. The following lists the explicit requirements:

\begin{itemize}
\item Mandatory (on client as well as server side):
\begin{itemize}
\item gnu make, autotools (autoconf $\geq$ 2.56, automake $\geq$ 1.8) (build)
\item C++ compiler and library (build)
\item libtool (build)
\item pkg-config (build)
\item gthread-2.0 $\geq$2 .4.7 (build, run)
\item glibmm-2.4 $\geq$ 2.4.7 (build, run)
\item libxml-2.0 $\geq$ 2.4.0 (build, run)
\item openssl $\geq$ 0.9.7a (build, run)
\item e2fsprogs (build, run)
\item doxygen (build)
\end{itemize}
\item Optional (mainly applicable on server side):
\begin{itemize}
\item Swig $\geq$ 1.3.28 (build)
\item Java SDK $\geq$ 1.4 for Java bindings (build, run)
\item Python for Python bindings (build, run)
\item Grid Packaging Tools (GPT)~\cite{gpt} (build)
\item \globus~\cite{globus} which contains (build, run):
\begin{itemize}
\item Globus RLS client
\item Globus FTP client
\item Globus RSL
\end{itemize}
\item LHC File Catalog (LFC)~\cite{lfc} (build, run)
\item CppUnit for unit testing (build)
\item Berkeley DB C++ interface (build, run)
\end{itemize}
\end{itemize}

The developers of ARC have prepared packages for several Linux
distributions\footnote{The nightly build system current comprises Debian,
Ubuntu, Fedora, RedHat and OpenSuSE}. As a user of any such operating
system, the external packages are all readily available.

\begin{framed}
Please note that depending on operating system distribution in order to build
ARC1 you may need to install development versions of mentioned packages.
\end{framed}

\section{Getting the software}
\label{sec:download}

The software is free to deploy anywhere by anybody. Pre-built binaries for a
dozen of Linux platforms can be downloaded from the NorduGrid software
repository at \ngdl (``arc1'' repository).

The software is released under the GNU General Public License (GPL) (see the
COPYING file distributed with the software).

You can get the latest source code for ARC1 from the Subversion repository.
For details, see:

\url{http://svn.nordugrid.org}

The NorduGrid software repository hosts the source code, and provides most of
the required external software which are not part of a standard Linux
distribution.

There are also nightly code snapshots available at:

\url{http://download.nordugrid.org/software/nordugrid-arc1/nightly/}

Choose the latest date available and download snapshot tarball -- for example\\
\verb|nordugrid-arc1-200802201038-snapshot.tar.gz|.

\section{Building and Installation}
\label{sec:build}

Building from source is currently the recommended way to install ARC Web
Services. If you downloaded the tarball, unpack it and traverse into the created
directory:

\begin{lstlisting}
tar -zxvf nordugrid-arc1-200802201038-snapshot.tar.gz
cd nordugrid-arc1-200802201038
\end{lstlisting}

If you obtained the code from the Subversion repository, use the ``trunk''
directory:

\begin{lstlisting}
cd trunk
\end{lstlisting}

Now configure the obtained code with:

\begin{lstlisting}
./autogen.sh
./configure --prefix=PLACE_TO_INSTALL_ARC
\end{lstlisting}

Choose installation prefix wisely and according to the requirements of your OS
and personal preferences. ARC1 services should function properly from any
location. By default installation goes into \texttt{/usr/local} if you omit the
``\texttt{--prefix}'' option. For some modules of ARC1 to work properly you may
need to set up the environment variable after installation:

\begin{lstlisting}
export ARC_LOCATION=PLACE_TO_INSTALL_ARC
\end{lstlisting}

On some systems ``\texttt{autogen.sh}'' may produce few warnings. Ignore them as
long as ``\texttt{configure}'' passes without errors. But in case of problems
during configure or compilation, collect them and present while reporting
problems.

If the previous commands finish without errors, compile and install ARC1
services:

\begin{lstlisting}
make
make install
\end{lstlisting}

Depending on chosen installation location you may need to run the last command
from \texttt{root} account. That should install the following components:

\begin{itemize}
\item \texttt{sbin/arched} - server executable
\item \texttt{bin/} - user tools and command line clients
\item \texttt{lib/} - common libraries used by clients, server and plugins
\item \texttt{lib/arc/} - plugins implementing Message Chain, Service and
Security components
\item \texttt{include/arc/} - C++ headers for application development
\item \texttt{libexec/} - additional modules used by ARC1 services - currently
only A-REX
\item \texttt{share/doc/arc} - configuration examples/templates and
documentation
\item \texttt{share/locale} - internationalization files - curently very limited
support
\item \texttt{share/man} - manual pages for various utilities
\end{itemize}

\section{Security and Authorization}
\label{sec:security}

ARC1 sevices implement security related features through set of Security Handler
and Policy Decision Point components. Security Handler components are attached
to message processing components. Each Security Handler takes care of processing
own part of security information. Currently ARC comes with the following
Security Handlers:

\begin{itemize}
 \item \texttt{identity.map} -- associates client's identity with local (UNIX)
identity. It uses PDP components to choose local isentity and/or identity
mapping algorithm.
\item \texttt{arc.authz} -- calls PDP components and combines
obtained authorization decisions.
\item \texttt{delegation.collector} -- parses proxy policy from remote proxy
certificate. This Security Handler should be configured under TLS MCC component.
\item \texttt{usernametoken.handler} -- implement the functioanlity
of WS-Security Usernametoken profile. It will generate Usernametoken
into SOAP header, or extract Usernametoken out of SOAP header and
do authentication based on the extracted Usernametoken.
\end{itemize}

Among available PDP components there are:
\begin{itemize}
\item \texttt{allow} -- always returns positive result
\item \texttt{deny} -- always returns negative result
\item  \texttt{simplelist.pdp} -- compares DN of user to those stored in a file.
\item  \texttt{arc.pdp} -- compares request information parsed from message and
policy information specified in this PDP.
\item  \texttt{pdpservice.invoker} -- composes the request, puts request into
SOAP message, and invokes the remote PDP service to get the response SOAP
which includes authorization decision. The PDP service has similar
functionality with \texttt{arc.pdp}.
 \item \texttt{delegation.pdp} -- compares request information parsed from
message and policy information specified in proxy certificate from remote side.
\end{itemize}

There are examples of A-REX service and Echo service with Security Handlers
being used. They may be found in:

\verb|$ARC_LOCATION/share/doc/arc/arex_secure.xml| \\and\\
\verb|$ARC_LOCATION/share/doc/arc/echo.xml|.

There is also a PDP service which implements the same functionality as
\texttt{arc.pdp}. See \texttt{src/service/pdp/README}.

Specifically for \texttt{arc.pdp} and \texttt{pdpservice}, a formatted policy
with specific schema should be managed, for details see:

\verb|$ARC_LOCATION/share/doc/arc/pdp_policy.xml.example| \\and\\
\verb|$ARC_LOCATION/share/doc/arc/Policy.xsd|

For usernametoken handler, there is example about configuration on service side
in \verb|$ARC_LOCATION/share/doc/arc/echo.xml|, you can run Echo service by
using this configuration file with usernametoken sechandler configuered. For
client side, the echo client (\texttt{src/client/echo})can use usernametoken
sechandler to authenticate against echo service (see README under
\verb|src/client/echo|); there is also a test program in
\verb|src/tests/echo/test_clientinterface.cpp| which can be compiled and
tested against Echo service with usernametoken sechandler configured.

\subsection{X509 Certificates}
\label{sec:x509}

Most of  planned and existing ARC1 services use HTTPS as transport protocol so
they require proper setup of X509 security infrastructure~\cite{x509}. Minimal
requirements are:

\begin{itemize}
\item Host certificate aka public key in PEM format
\item Corresponding private key
\item Certificate of the Certification Authority (CA) which was used to sign
the host certificate
\item Certificates of CAs of clients which are going to send requests to
services (unless of course clients use the same CA as the server).
\end{itemize}

More information about X509 certificates and their usage in Grid environement
can be found in:

\url{http://www.nordugrid.org/documents/certificate_howto.html}\\
\url{http://www.nordugrid.org/documents/ng-server-install.html#security}

For testing purposes you can use the pre-generated certificates and keys
available in the code repository:

\url{http://svn.nordugrid.org/trac/nordugrid/browser/arc1/trunk/doc/sec/TestCA}

Alternatively you may choose to use KnowARC Instant CA service available at:

\url{https://vls.grid.upjs.sk/CA/instantCA}

The latter is especially usefull if you want to test installation consisting of
multiple hosts.

Please remember that it is not safe to use these keys in publicly accessible
insallations of ARC. Make sure that even the CA certificate is removed before
making your services available to the outside world.

You can put host certificates and private keys anywhere, their location is
configurable. Common locations for servers running from root account are:

\verb|/etc/grid-security/hostcert.pem| \\and\\
\verb|/etc/grid-security/hostkey.pem|

respectively.

\begin{framed}
Since services have no way to ask for passwords, the content of private key
must not be encrypted, neither it should be protected by password. So make sure
it is properly protected by means of the file system.
\end{framed}

It is possible to configure ARC1 server to accept either a single CA certificate
or multiple CA certificates located in the specified directory. The latter
option is recommended. The common location is
\texttt{/etc/grid-security/certificates/} . In that case names of certificate
files have to follow hash values of the certificates. These are obtainable by
running the command:

\begin{lstlisting}
openssl x509 -hash -noout -in path_to_certificate
\end{lstlisting}

The corresponding file name for the certificate should be
\texttt{<hash\_value>.0}. The hash value for the pre-generated CA certificate is
\texttt{4457e417}.

Please make sure the chosen location of certificates is correctly configured in
the service configuration file. The configuration for the certificate for TLS
MCC should look like this:

\begin{lstlisting}
<KeyPath>/etc/grid-security/hostkey.pem</KeyPath>
<CertificatePath>/etc/grid-security/hostcert.pem</CertificatePath>
<CACertificatesDir>/etc/grid-security/certificates</CACertificatesDir>
\end{lstlisting}

or

\begin{lstlisting}
<CACertificatePath>/etc/grid-security/ca.pem</CACertificatePath>
\end{lstlisting}

The same requirements are valid for the client tools for ARC1. You may use the
pre-generated user certificate and key located at the same place. Locations of
the credentials are configurable, or can be provided to the client tools from
the command line.

The set of pre-generated keys and certificates also includes a user certificate
in PKCS12 format which you can import into your browser for accessing ARC1
services capable of producing HTML output.

ARC comes with utility \texttt{arcproxy} which generates proxy credentials from
certificate/private key pair. It provides only basic functionality and is meant
for testing purposes only.

\begin{framed}
\textbf{IMPORTANT}: If during configuration stage you see a message

 \textit{``OpenSSL contains no support for proxy credentials''}

then you won't be able to use proxy credentials generated by utilities
like \texttt{grid-proxy-init}, \texttt{voms-proxy-init} or \texttt{arcproxy}.
Because of that all user private keys will have to be kept unencrypted.
\end{framed}

To avoid providing credential information on command line it is possible to have
user configuration file with predefined values. Default location for this file
is \texttt{~/.arc/client.xml}. There is an example installed in
\verb|$ARC_LOCATION/share/doc/arc/client.xml|. Please edit before copying it
into your home directory.

\subsection{Proxy Certificate Generation and Usage}
\label{sec:proxy}

As metioned above, ARC comes with proxy generation utility
-- \texttt{arcproxy}, installed in \verb|$ARC_LOCATION/bin|. The usage
of \texttt{arcproxy} is like:

\begin{lstlisting}
$ARC_LOCATION/bin/arcproxy -P proxy.pem -C cert.pem -K key.pem
  -c validityStart=2008-05-29T10:20:30Z
  -c validityEnd=2008-06-29T10:20:30Z
  -c proxyPolicyFile=delegation_policy.xml
\end{lstlisting}

By using argument ``-c'', some constraints can be specified for proxy
certificate. Currently, the life-time can be specified by using ``\texttt{-c
validityStart=...}'' and ``\texttt{-c validityEnd=...}'', or ``\texttt{-c
validityStart=...}'' and ``\texttt{-c validityPeriod=...}''; the proxy policy
can be specified by using ``\texttt{-c proxyPolicyFile=...}'' or ``\texttt{-c
proxyPolicy=...}''.

If proxy certificate is used, in the configuration file for service side or
client side, the configuration for the certificate for TLS MCC should look like
this:

\begin{lstlisting}
<KeyPath>./proxy.pem</KeyPath>
<CertificatePath>./proxy.pem</CertificatePath>
<CACertificatePath>./ca.pem</CACertificatePath>
\end{lstlisting}

Because normally a proxy certificate file includes the proxy certificate and
private key corresponding to the proxy certificate, \texttt{<KeyPath/>} and
\texttt{<CertificatePath/>} are configured to be the same.

Proxy policy can be spefified as constraint. Proxy policy is for constraining
identity delegation. Currently, the supported policy is ARC specific policy.
Proxy policy is inserted into proxy certificate's ``proxy cert info'' extenstion
in RFC3820's policy language ``\verb|NID_id_ppl_anyLanguage|''.


\section{ARC Server Setup and Configuration}
\label{sec:server-setup}

The configuration of the ARC server is specified in an XML file, the location of
which is specified as a command line argument with the -c option of
``\texttt{arched}'' daemon. Examples of configuration files with comments
describing various elements are available in directory
\texttt{/usr/share/doc/nordugrid-arc<version>} corresponding to the ARC1
installation.


\section{The Echo Service}
\label{sec:echo-service}

The \textit{Echo} service is offered purely for testing purposes. It is
``atomic'' and has no additional dependencies other than what is provided by the
Hosting Environment Daemon (HED). An example of an Echo service configuration
can be found in \texttt{/usr/share/doc/nordugrid-arc<version>/echo.xml}.


\subsection{The Echo Client}
\label{sec:echo-client}

The configuration of the ARC Echo client is specified in an XML
file. The location of the configuration file is specified by the
environment variable \verb|$ARC_ECHO_CONFIG|. If there is no such
environment
variable, the configuration file is assumed to be \verb|echo_client.xml| in
the current working directory.

To use the Echo client, type

\begin{lstlisting}
$ARC_LOCATION/bin/arcecho <message>
\end{lstlisting}

where \verb|<message>| is the message which the Echo service should return.


\section{The A-REX Service}
\label{sec:a-rex}

ARC1 comes with OGSA BES compliant Grid job management service called A-REX.
To deploy A-REX use example configuration files available in
\verb|/usr/share/doc/nordugrid-arc<version>|:

\begin{itemize}
 \item \texttt{arex.xml} -- configuration for arched server. Read comments
inside this file.
 \item \texttt{arc-arex.conf} -- legacy configuration for Grid Manager part of
A-REX. This
file defines how jobs are managed by A-REX locally. Read and edit it.
and edit it to fit your installation. This file defines WS interface of A-REX.
\end{itemize}

For more detailed information please read Grid Manager
documentation~\cite{arex}. Grid Manager runs as part of A-REX service. There is
no need to run any additional executable. But you still need to setup its
infrastructure as long as you are going to have anything more sophisticated than
described in the example configuration. For more information read the previously
mentioned document.

Currently, a proper functioning A-REX requires environment variable
\verb|$ARC_LOCATION| to be set to the installation prefix of ARC.

A-REX uses HTTPS as transport protocol (although one can reconfigure it to use
plain HTTP) so it requires proper setup of X509 security infrastructure. See
Section~\ref{sec:x509} for instructions.

Copy example configuration files to some location and edit them. Make sure all
paths to X509 certificates and Grid Manager configuration are set correctly.
Start server with command:

\begin{lstlisting}
$ARC_LOCATION/sbin/arched -c path_to_edited_arex.xml
\end{lstlisting}

Look into log file specified in \texttt{arex.xml} for possible errors. You can
safely ignore messages like \textit{``Not a '...' type plugin''} and
\textit{``Unknown element ... - ignoring''}.

If you compiled ARC1 services with Globus support and you see complaints about
``libglobus...'' and that it cannot open a shared object file, try to add
\texttt{/opt/globus/lib} to your \verb|$LD_LIBRARY_PATH|:

\begin{lstlisting}
export LD_LIBRARY_PATH=$LD_LIBRARY_PATH:/opt/globus/lib
\end{lstlisting}

\subsection{Testing and Using A-REX (clients)}
\label{sec:testing}

Now you may use the command line utility ``\texttt{arcinfo}'' to obtain a service
description. Make sure user key and certificate and CA certificate(s) are
readable from account you use, and run something like:

\begin{lstlisting}
./arcinfo -d VERBOSE -C $HOME/usercert.pem -K $HOME/userkey.pem
 -A /etc/grid-security/certificates https://localhost:60000/arex 2>error.log
\end{lstlisting}

This should produce XML output describing resource A-REX represents.
Listing~\ref{lst:arcinfo} shows an example of proper output. In case of error see
\texttt{error.log} file and report problems to ARC development team. Include
content of \texttt{error.log} and server log.

\begin{lstlisting}[caption={Example of a good arcinfo output},label=lst:arcinfo]
<n:nordugrid xmlns:n="urn:nordugrid" xmlns:M0="urn:Mds"
xmlns:nc0="urn:nordugrid-cluster">
  <nc0:name>mobile2.home.net</nc0:name>
  <nc0:aliasname>MINIMAL Computing Element</nc0:aliasname>
  <nc0:comment>This is a minimal out-of-box CE setup</nc0:comment>
  <nc0:issuerca>/O=Grid/O=Test/CN=CA</nc0:issuerca>
  <nc0:issuerca-hash>4457e417</nc0:issuerca-hash>
  <nc0:trustedca>/O=Grid/O=NorduGrid/CN=NorduGrid Certification
Authority</nc0:trustedca>
  <nc0:trustedca>/O=Grid/O=Test/CN=CA</nc0:trustedca>
  <nc0:contactstring>gsiftp://mobile2.home.net:2811/jobs</nc0:contactstring>
  <nc0:lrms-type>fork</nc0:lrms-type>
  <nc0:architecture>i686</nc0:architecture>
  <nc0:homogeneity>TRUE</nc0:homogeneity>
  <nc0:nodeaccess>inbound</nc0:nodeaccess>
  <nc0:nodeaccess>outbound</nc0:nodeaccess>
  <nc0:totalcpus>1</nc0:totalcpus>
  <nc0:usedcpus>0</nc0:usedcpus>
  <nc0:cpudistribution>1cpu:1</nc0:cpudistribution>
  <nc0:prelrmsqueued>0</nc0:prelrmsqueued>
  <nc0:totaljobs>0</nc0:totaljobs>
  <nc0:sessiondir-free>8469</nc0:sessiondir-free>
  <nc0:sessiondir-total>43865</nc0:sessiondir-total>
  <nc0:sessiondir-lifetime>10080</nc0:sessiondir-lifetime>
  <M0:validfrom>20080220142836Z</M0:validfrom>
  <M0:validto>20080220143036Z</M0:validto>
  <nq0:name xmlns:nq0="urn:nordugrid-queue" name="fork">
    <nq0:name>fork</nq0:name>
    <nq0:status>inactive, grid-manager is down</nq0:status>
    <nq0:comment>This queue is nothing more than a fork host</nq0:comment>
    <nq0:schedulingpolicy>FIFO</nq0:schedulingpolicy>
    <nq0:homogeneity>TRUE</nq0:homogeneity>
    <nq0:nodecpu>Intel(R) Celeron(R) M CPU        420  @ 1.60GHz @ 1596.048
MHz</nq0:nodecpu>
    <nq0:architecture>i686</nq0:architecture>
    <nq0:maxcputime>unlimited</nq0:maxcputime>
    <nq0:maxwalltime>unlimited</nq0:maxwalltime>
    <nq0:running>0</nq0:running>
    <nq0:gridrunning>0</nq0:gridrunning>
    <nq0:localqueued>0</nq0:localqueued>
    <nq0:gridqueued>0</nq0:gridqueued>
    <nq0:prelrmsqueued>0</nq0:prelrmsqueued>
    <nq0:totalcpus>1</nq0:totalcpus>
    <IM0:validfrom>20080220142836Z</M0:validfrom>
    <M0:validto>20080220143036Z</M0:validto>
    <nig0:name xmlns:nig0="urn:nordugrid-info-group" name="jobs">
      <nig0:name>jobs</nig0:name>
      <M0:validfrom>20080220142836Z</M0:validfrom>
      <M0:validto>20080220143036Z</M0:validto>
      <nj0:globalid xmlns:nj0="urn:nordugrid-job"
name="gsiftp://mobile2.home.net:2811/jobs/636312034687631804
289383">

<nj0:globalid>gsiftp://mobile2.home.net:2811/jobs/636312034687631804289383</nj0:
globalid>
        <nj0:globalowner>/O=Grid/O=Test/CN=user</nj0:globalowner>
        <nj0:jobname>TEST</nj0:jobname>
        <nj0:execcluster>mobile2.home.net</nj0:execcluster>
        <nj0:execqueue>fork</nj0:execqueue>
        <nj0:submissionui>127.0.0.1:59790</nj0:submissionui>
        <nj0:submissiontime>20080220005243Z</nj0:submissiontime>
        <nj0:sessiondirerasetime>20080227005843Z</nj0:sessiondirerasetime>
        <nj0:completiontime>20080220005843Z</nj0:completiontime>
        <nj0:cpucount>1</nj0:cpucount>
        <nj0:comment>GM: The grid-manager is down</nj0:comment>
        <nj0:usedcputime>0</nj0:usedcputime>
        <nj0:usedwalltime>0</nj0:usedwalltime>
        <nj0:errors>Job submission to LRMS failed</nj0:errors>
        <nj0:status>FAILED</nj0:status>
        <nj0:rerunable>SUBMIT</nj0:rerunable>
        <M0:validfrom>20080220142836Z</M0:validfrom>
        <M0:validto>20080220143036Z</M0:validto>
      </nj0:globalid>
    </nig0:name>
    <nig0:name xmlns:nig0="urn:nordugrid-info-group" name="users">
      <nig0:name>users</nig0:name>
      <M0:validfrom>20080220142836Z</M0:validfrom>
      <M0:validto>20080220143036Z</M0:validto>
    </nig0:name>
  </nq0:name>
</n:nordugrid>
\end{lstlisting}

Please note that you can run similar \texttt{arcinfo} request against any ARC1
service except Echo service. Output should always be XML description of the service.

A-REX accepts jobs described in JSDL language. Example JSDL jobs are provided in
\verb|$ARC_LOCATION/share/doc/| in files \verb|jsdl_simple.xml| and
\verb|jsdl_stage.xml|. To submit job to A-REX service one may use
\texttt{arcsub} command:

\begin{lstlisting}
$ARC_LOCATION/bin/arcsub https://localhost:60000/arex
 /usr/local/share/doc/arc/jsdl_simple.xml id.xml
\end{lstlisting}

If everything goes properly somewhere in it's output there should be a message
``Submitted the job!''. Obtained job identifier is an XML document and is stored
in id.xml file. It may then be used to query job state with \texttt{arcstat}
utility:

\begin{lstlisting}
$ARC_LOCATION/bin/arcstat id.xml 2>/dev/null
Job status: Running/Submiting

$ARC_LOCATION/bin/arcstat id.xml 2>/dev/null
Job status: Running/Finishing

$ARC_LOCATION/bin/arcstat id.xml 2>/dev/null
Job status: Finished/Finished
\end{lstlisting}

Full set of A-REX client tools consists of \texttt{arcsub}, \texttt{arcstat},
\texttt{arckill}, \texttt{arcget} and \texttt{arcclean} commands.
For more information please see man pages ot those utilities.

\section{Finding more information}
\label{sec:more-info}

Many information about functionality and configuration of various components
may be found inside the corresponding configuration XML schemas.

There are several API documents available as well.

\section{Contributing}
\label{sec:contributing}

The open source development of the ARC middleware is coordinated by the
NorduGrid collaboration. Currently, the main contributor is the KnowARC
project~\cite{knowarc}, but the Collaboration is open to new members.
Contributions from the community to the software and the documentation is
welcomed. Sources can be downloaded from the software repository at:

\url{http://download.nordugrid.org}

or the Subversion code repository at

\url{http://svn.nordugrid.org}.

The technical coordination group defines outstanding issues that have
to be addressed in the framework of the ARC development. Feature
requests and enhancement proposals are recorded in the Bugzilla problem
tracking system at:

 \url{http://bugzilla.nordugrid.org}

For a more detailed description, write access to the code repository and
further questions, write to the \texttt{nordugrid-discuss} mailing list (see
NorduGrid website~\cite{nordugrid} for details). Ongoing and completed Grid
research projects and student assignments related to the middleware are listed
on the NorduGrid website as well.


\section{Support, documentation, mailing lists, contact}
\label{sec:support}

User support and site installation assistance is provided via the
request tracking system available by e-mail to
\texttt{nordugrid-support@nordugrid.org}.
In addition, NorduGrid runs a couple of mailing lists, among which the
nordugrid-discuss mailing list is a general forum for all kind of
issues related to the ARC middleware.

Feature and enhancement requests, as well as discovered problems, should be
reported in the Bugzilla problem tracking system
(\url{http://bugzilla.nordugrid.org}).

Research papers, overview talks, reference manuals, user guides,
installation instructions, conference presentations, FAQ and even
tutorial materials can be fetched from the documentation section of the
NorduGrid website.

Contact information is kept updated on the NorduGrid website.

\bibliography{grid}
\end{document}

