%% LyX 1.4.5.1 created this file.  For more info, see http://www.lyx.org/.
%% Do not edit unless you really know what you are doing.
\documentclass[english]{book}
\usepackage[T1]{fontenc}
\usepackage[latin1]{inputenc}
\setcounter{secnumdepth}{3}
\setcounter{tocdepth}{3}
\usepackage{calc}
\usepackage{graphicx}

\makeatletter

%%%%%%%%%%%%%%%%%%%%%%%%%%%%%% LyX specific LaTeX commands.
%% Because html converters don't know tabularnewline
\providecommand{\tabularnewline}{\\}

%%%%%%%%%%%%%%%%%%%%%%%%%%%%%% User specified LaTeX commands.

%\documentclass{article}                            %for shorter notes
                              %for PNG images (pdflatex)
%\usepackage{graphics}                              %for EPS images (latex)
\usepackage[linkbordercolor={1.0 1.0 0.0}]{hyperref}
 %for \url tag
\usepackage{color}
                                 %for defining custom colors
\usepackage{framed}
                                %for shaded and framed paragraphs
\usepackage{textcomp}
                              %for various symbols, e.g. Registered Mark
\usepackage{geometry}
                              %for defining page size
\usepackage{longtable}
                             %for breaking tables
%
\geometry{verbose,a4paper,tmargin=2.5cm,bmargin=2.5cm,lmargin=2.5cm,rmargin=2cm}
\hypersetup{
  pdfauthor = {Author Name},
  pdftitle = {Paper title},
  pdfsubject = {Paper subject},
  pdfkeywords = {Paper,keyword,comma-separated},
  pdfcreator = {PDFLaTeX with hyperref package},
  pdfproducer = {PDFLaTeX}
}
%
\bibliographystyle{IEEEtran}                       %a nice bibliography style
%
\def\efill{\hfill\nopagebreak}%
\hyphenation{Nordu-Grid}
\setlength{\parindent}{0cm}
\setlength{\FrameRule}{1pt}
\setlength{\FrameSep}{8pt}
\addtolength{\parskip}{5pt}
\renewcommand{\thefootnote}{\fnsymbol{footnote}}
\renewcommand{\arraystretch}{1.3}
\newcommand{\dothis}{\colorbox{shadecolor}}
\newcommand{\globus}{Globus Toolkit\textsuperscript{\textregistered}~2~}
\newcommand{\GT}{Globus Toolkit\textsuperscript{\textregistered}}
\newcommand{\ngdl}{\url{http://ftp.nordugrid.org/download}~}
\definecolor{shadecolor}{rgb}{1,1,0.6}
\definecolor{salmon}{rgb}{1,0.9,1}
\definecolor{bordeaux}{rgb}{0.75,0.,0.}
\definecolor{cyan}{rgb}{0,1,1}
%
%----- DON'T CHANGE HEADER MATTER


\usepackage{babel}
\makeatother
\begin{document}
\def\today{\number\day/\number\month/\number\year}

\begin{titlepage}

\begin{tabular}{rl}
\resizebox*{3cm}{!}{\includegraphics{tex/ng-logo}} &
%
\parbox[b][0pt]{2cm}{%
\textit{{\hspace*{-1.5cm}NORDUGRID\vspace*{0.5cm}
}}%
}%
 \tabularnewline
\end{tabular}

\hrulefill

%-------- Change this to NORDUGRID-XXXXXXX-NN


{\raggedleft NORDUGRID-XXXXXXX-NN

}

{\raggedleft \today

}

\vspace*{2cm}


%%%%---- The title ----
{\centering \textsc{\Large Security Infrastructure of ARC1}{\Large \par}

} \vspace*{0.5cm}


%%%%---- A subtitle, if necessary ----
{\centering \textit{\large Paper subtitle}{\large \par}

}

\vspace*{1.5cm}
 {\centering {\large Weizhong Qiang}%
\footnote{weizhongqiang@gmail.com%
}{\large , Aleksandr Konstantinov}%
\footnote{aleks@fys.uio.no%
} {\large }{\large \par}

}

%%%%---- An abstract - if style is article ----
%\begin{abstract}
%The abstract
%\end{abstract}
\end{titlepage}

\tableofcontents{}%Comment if use article style
\newpage{} 


\chapter{Introduction}

\label{sec:intro}

This document describes Security Infractructure of ARC1 middleware.


\chapter{Security architecture in HED: SecHandlers and PDP}


\chapter{Policy Evaluation engine}


\section{ArcPDP Policy Evaluator}


\chapter{Secuirty Attributes}


\section{Infrastructure}

Security Attributes represent security related information inside

HED framework and store information representing various aspects 

needed to perform authorization decison - identity of client, requested

action, targeted resource, constraint policies. Each kind of Security 

Attribute is represented by own class inherited from parent SecAttr
class.

Each Security Attribute stores it's information in internal format
and 

provides it to external code in one of predefined formats. Currently

only supported format is Arc Policy/Request.

Collectors of Security Attributes instantiate corresponding classes
and

link them to Secuirity Attributes containers - MessageAuth and 

MessageAuthContext representing attributes per request and per session 

correspondingly. Current implementations of Security Attributes Collectors

are either integratd into existing MCCs or implemented as separate
SecHandler

plugins. See Appendix A for available collectors and corresponding
Security

Attributes. 

Note for service developers: Services may implement own authorization 

algorithms. But they may use Security Atributes as well by providing 

instances of classes inherited from SecAttr and running them through

either configured or hardcoded processors/PDPs (see below).

Processors of Security Attributes are implemented as Policy Decision
Point 

components. Currently there are 2 PDP components available:

{*} Arc PDP makes use of Security Attributes containing identities
of client,

resource and requested action. It evaluates either all or selected
set 

of attributes against specified Policy documents thus making it possible

to enforce policies defined/selected by service providers. 

{*} Delegation PDP is similar to Arc PDP except that it takes it's
Policy

documents directly from Security Attributes. Differently from Arc
PDP it 

is meant to be used for enforcing policies defined by client.

Typical service example - TODO


\section{Available collectors}

{*} TCP

Information is collected inside TCP MCC. Security Attribute exports 

ARC Request with following attributes

Element AttributeId Content

Resource http://www.nordugrid.org/schemas/policy-arc/types/localendpoint
service\_ip{[}:service\_port] 

SubjectAttribute http://www.nordugrid.org/schemas/policy-arc/types/remoteendpoint
client\_ip{[}:client\_port] 

{*} TLS

Information is collected inside TLS MCC. Security Attribute exports 

ARC Request with following attributes

Element AttributeId Content

SubjectAttribute http://www.nordugrid.org/schemas/policy-arc/types/tls/ca
Subject of signer of first certificate in client's chain

SubjectAttribute http://www.nordugrid.org/schemas/policy-arc/types/tls/chain
Subject of certificate in client's chain - multiple items

SubjectAttribute http://www.nordugrid.org/schemas/policy-arc/types/tls/subject
Subject of last certificate in client's chain

SubjectAttribute http://www.nordugrid.org/schemas/policy-arc/types/tls/identity
Subject of last non-proxy certificate in client's chain

{*} HTTP

Information is collected inside TLS MCC. Security Attribute exports 

ARC Request with following attributes

Element AttributeId Content

Resource http://www.nordugrid.org/schemas/policy-arc/types/http/path
HTTP path without host and port part

Action http://www.nordugrid.org/schemas/policy-arc/types/http/method
HTTP method 

{*} SOAP

Information is collected inside TLS MCC. Security Attribute exports 

ARC Request with following attributes

Element AttributeId Content

Resource http://www.nordugrid.org/schemas/policy-arc/types/soap/endpoint
WS-Addr To element

Action http://www.nordugrid.org/schemas/policy-arc/types/soap/operation
SOAP top level element name without namespace prefix

Context http://www.nordugrid.org/schemas/policy-arc/types/soap/namespace
Namespace of SOAP top level element

{*} Delegation

This Security Attribute is collected by dedicated Security Handler
plugin.

It extracts policy document stored inside X509 certificate proxy extension

as defined in RFC3820. All certificates in a chain provided by client
are 

examined. Proxy certificates with id-ppl-inheritAll property are passed 

through and no policy is generated for them. From those with id-ppl-anyLanguage 

string content of ProxyPolicy element is extracted and converted into
XML 

document. Then document is check to be of Arc Policy kind. If policy
is

not recognized as Arc Policy procedure fails and that causes failure
of

comunication. Proxies with other type of policies including id-ppl-independent

are not accepted and generate immediate failure.


\chapter{Delegation restrictions}


\section{Delegation architecture}


\section{Delegation PDP}


\chapter*{Schemas}


\section*{Policy}


\section*{Request}


\chapter*{Examples:}


\section*{Policy}


\section*{Request}


\section*{Delegation Policy}


\chapter*{Template examples}

In expressions, the following operands are allowed: \begin{shaded}
\verb#=   !=   >   <   >=   <=# \end{shaded}

\begin{framed} Examples of URLs are:\\
 \\
 \verb#http://grid.domain.org/dir/script.sh#\\
 \verb#gsiftp://grid.domain.org:2811;threads=10/dir/input_12378.dat#\\
 \verb#ldap://grid.domain.org:389/lc=collection1,rc=Nordugrid,dc=nordugrid,dc=org#\\
 \verb#rc://grid.domain.org/lc=collection1,rc=Nordugrid,dc=nordugrid,dc=org/zebra/f1.zebra#
\verb#file:///home/auser/griddir/steer.cra#\\
 \end{framed}

%\subsection{Subsection}
%\label{sec:subsection}


%
\begin{figure}[ht]
 \centering{{\scalebox{0.9}{\includegraphics{tex/ng-logo}}} 


\caption{\label{fig:myfigure1}The figure shows a logo.}

} 
\end{figure}


\bibliography{grid}
 
\end{document}
