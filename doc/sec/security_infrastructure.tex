%% LyX 1.4.5.1 created this file.  For more info, see http://www.lyx.org/.
%% Do not edit unless you really know what you are doing.
\documentclass[english]{book}
\usepackage[T1]{fontenc}
\usepackage[latin1]{inputenc}
\setcounter{secnumdepth}{3}
\setcounter{tocdepth}{3}
\usepackage{array}
\usepackage{calc}
\usepackage{graphicx}

\makeatletter

%%%%%%%%%%%%%%%%%%%%%%%%%%%%%% LyX specific LaTeX commands.
%% Because html converters don't know tabularnewline
\providecommand{\tabularnewline}{\\}

%%%%%%%%%%%%%%%%%%%%%%%%%%%%%% User specified LaTeX commands.

%\documentclass{article}                            %for shorter notes
                              %for PNG images (pdflatex)
%\usepackage{graphics}                              %for EPS images (latex)
\usepackage[linkbordercolor={1.0 1.0 0.0}]{hyperref}
 %for \url tag
\usepackage{color}
                                 %for defining custom colors
\usepackage{framed}
                                %for shaded and framed paragraphs
\usepackage{textcomp}
                              %for various symbols, e.g. Registered Mark
\usepackage{geometry}
                              %for defining page size
\usepackage{longtable}
                             %for breaking tables
%
\geometry{verbose,a4paper,tmargin=2.5cm,bmargin=2.5cm,lmargin=2.5cm,rmargin=2cm}
\hypersetup{
  pdfauthor = {Author Name},
  pdftitle = {Paper title},
  pdfsubject = {Paper subject},
  pdfkeywords = {Paper,keyword,comma-separated},
  pdfcreator = {PDFLaTeX with hyperref package},
  pdfproducer = {PDFLaTeX}
}
%
\bibliographystyle{IEEEtran}                       %a nice bibliography style
%
\def\efill{\hfill\nopagebreak}%
\hyphenation{Nordu-Grid}
\setlength{\parindent}{0cm}
\setlength{\FrameRule}{1pt}
\setlength{\FrameSep}{8pt}
\addtolength{\parskip}{5pt}
\renewcommand{\thefootnote}{\fnsymbol{footnote}}
\renewcommand{\arraystretch}{1.3}
\newcommand{\dothis}{\colorbox{shadecolor}}
\newcommand{\globus}{Globus Toolkit\textsuperscript{\textregistered}~2~}
\newcommand{\GT}{Globus Toolkit\textsuperscript{\textregistered}}
\newcommand{\ngdl}{\url{http://ftp.nordugrid.org/download}~}
\definecolor{shadecolor}{rgb}{1,1,0.6}
\definecolor{salmon}{rgb}{1,0.9,1}
\definecolor{bordeaux}{rgb}{0.75,0.,0.}
\definecolor{cyan}{rgb}{0,1,1}
%
%----- DON'T CHANGE HEADER MATTER

\usepackage{babel}
\makeatother
\begin{document}
\def\today{\number\day/\number\month/\number\year}

\begin{titlepage}

\begin{tabular}{rl}
\resizebox*{3cm}{!}{\includegraphics{tex/ng-logo}} &
%
\parbox[b][0pt]{2cm}{%
\textit{{\hspace*{-1.5cm}NORDUGRID\vspace*{0.5cm}
}}%
}%
 \tabularnewline
\end{tabular}

\hrulefill

%-------- Change this to NORDUGRID-XXXXXXX-NN


{\raggedleft NORDUGRID-TECH-16

}

{\raggedleft \today

}

\vspace*{2cm}


%%%%---- The title ----
{\centering \textsc{\Large Security Infrastructure of ARC1}{\Large \par}

} \vspace*{0.5cm}


%%%%---- A subtitle, if necessary ----
{\centering \textit{\large Paper subtitle}{\large \par}

}

\vspace*{1.5cm}
 {\centering {\large Weizhong Qiang}%
\footnote{weizhongqiang@gmail.com%
}{\large , Aleksandr Konstantinov}%
\footnote{aleks@fys.uio.no%
} {\large }{\large \par}

}

%%%%---- An abstract - if style is article ----
%\begin{abstract}
%The abstract
%\end{abstract}
\end{titlepage}

\tableofcontents{}%Comment if use article style
\newpage{} 


\chapter{Introduction\label{sec:introduction}}

This document describes Security Infractructure of ARC1 middleware.


\chapter{Security architecture in HED: SecHandlers and PDP\label{sec:architecture}}


\chapter{Policy Evaluation engine\label{sec:policy_engine}}


\section{ArcPDP Policy Evaluator\label{sec:policy_evaluator}}


\chapter{Secuirty Attributes\label{sec:sec_attr}}


\section{Infrastructure\label{sec:sec_attr_infrastructure}}

Security Attributes represent security related information inside
HED framework and store information representing various aspects needed
to perform authorization decison - identity of client, requested action,
targeted resource, constraint policies. 

Each kind of Security Attribute is represented by own class inherited
from parent SecAttr class <arc/message/SecAttr.h>. Each Security Attribute
stores it's information in internal format and is capable to export
it to one of predefined formats using Export() method. Currently only
supported format is Arc Policy/Request XML document.

Collectors of Security Attributes instantiate corresponding classes
and link them to Secuirity Attributes containers - MessageAuth <arc/message/MessageAuth.h>
and MessageAuthContext <arc/message/Message.h> storing collected attributes
per request and per session correspondingly. Each attribute is assigned
a name. Current implementations of Security Attributes Collectors
are either integrated into existing MCCs or implemented as separate
SecHandler plugins. See Section \ref{sec:sec_attr_collectors} for
available Collectors and corresponding Security Attributes. 

\begin{framed}Note for service developers: Services may implement
own authorization algorithms. But they may use Security Atributes
as well by providing instances of classes inherited from SecAttr and
running them through either configured or hardcoded processors/PDPs.\end{framed}

Processors of Security Attributes are implemented as Policy Decision
Point components. Currently there are 2 PDP components available:

\begin{itemize}
\item ArcPDP makes use of Security Attributes containing identities of client,
resource and requested action. It evaluates either all or selected
set of attributes against specified Policy documents thus making it
possible to enforce policies defined/selected by service providers. 
\item DelegationPDP is described below in Section \ref{sec:delegation_pdp}.
\end{itemize}

\section{Available collectors\label{sec:sec_attr_collectors}}

Here Security Attribute collectors distributed as part of ARC1 are
described except those used for Delegation Restrictions. Those are
described in Section \ref{sec:delegation}.


\subsection{TCP}

Information is collected inside TCP MCC. Security Attribute is stored
under name 'TCP' and exports ARC Request with following attributes:

\begin{tabular}{|>{\centering}p{0.2\textwidth}|c|>{\centering}p{0.2\textwidth}|}
\hline 
Element&
AttributeId&
Content\tabularnewline
\hline
\hline 
Resource&
http://www.nordugrid.org/schemas/policy-arc/types/localendpoint&
service\_ip{[}:service\_port] \tabularnewline
\hline 
SubjectAttribute&
http://www.nordugrid.org/schemas/policy-arc/types/remoteendpoint&
client\_ip{[}:client\_port] \tabularnewline
\hline
\end{tabular}


\subsection{TLS}

Information is collected inside TLS MCC. Generated Security Attribute
class exports ARC Request with following attributes:

\begin{tabular}{|>{\centering}p{0.2\textwidth}|c|>{\centering}p{0.2\textwidth}|}
\hline 
Element&
AttributeId&
Content\tabularnewline
\hline
\hline 
SubjectAttribute&
http://www.nordugrid.org/schemas/policy-arc/types/tls/ca&
Subject of signer of first certificate in client's chain\tabularnewline
\hline 
SubjectAttribute&
http://www.nordugrid.org/schemas/policy-arc/types/tls/chain&
Subject of certificate in client's chain - multiple items\tabularnewline
\hline 
SubjectAttribute&
http://www.nordugrid.org/schemas/policy-arc/types/tls/subject&
Subject of last certificate in client's chain\tabularnewline
\hline 
SubjectAttribute&
http://www.nordugrid.org/schemas/policy-arc/types/tls/identity&
Subject of last non-proxy certificate in client's chain\tabularnewline
\hline
\end{tabular}


\subsection{HTTP}

Information is collected inside TLS MCC. Security Attribute is stored
under name 'HTTP' and exports ARC Request with following attributes:

\begin{tabular}{|>{\centering}p{0.2\textwidth}|c|>{\centering}p{0.2\textwidth}|}
\hline 
Element&
AttributeId&
Content\tabularnewline
\hline
\hline 
Resource&
http://www.nordugrid.org/schemas/policy-arc/types/http/path&
HTTP path without host and port part\tabularnewline
\hline 
Action&
http://www.nordugrid.org/schemas/policy-arc/types/http/method&
HTTP method\tabularnewline
\hline
\end{tabular}


\subsection{SOAP}

Information is collected inside TLS MCC. Security Attribute is stored
under name 'SOAP' and exports ARC Request with following attributes:

\begin{tabular}{|>{\centering}p{0.2\textwidth}|c|>{\centering}p{0.2\textwidth}|}
\hline 
Element&
AttributeId&
Content\tabularnewline
\hline
\hline 
Resource&
http://www.nordugrid.org/schemas/policy-arc/types/soap/endpoint&
To element of WS-Addressing structure\tabularnewline
\hline 
Action&
http://www.nordugrid.org/schemas/policy-arc/types/soap/operation&
SOAP top level element name without namespace prefix\tabularnewline
\hline 
Context&
http://www.nordugrid.org/schemas/policy-arc/types/soap/namespace&
Namespace of SOAP top level element\tabularnewline
\hline
\end{tabular}


\chapter{Delegation Restrictions\label{sec:delegation}}


\section{Delegation Architecture\label{sec:delegation_architectire}}

In current implementation delegation is achieved through Identity
Delegation implemented using X509 Proxy Certificates as defined in
RFC 3820. Client wishing to allow service to act on it's behalf provides
Proxy Certificate to the service using Web Service based Delegation
interface described in \ref{sec:delegation_interface}. 

For limitinng the scope of delegated credentials along with usually
used time constraints it is possible to attach Policy document to
Proxy Certificate. According to RFC 3820 Policy is stored in \emph{ProxyPolicy}
extension. In order not to introduce new type of object Policy is
assigned \emph{id-ppl-anyLanguage} identifier. RFC 3820 allows any
octet string associated with such object. We are using textual representation
of ARC Policy XML document.

Each deployment implementing Delegation Restrictions must use dedicated
Security Handler plugin (see section \ref{sec:delegation_collector})
to collect all Policy documents from Proxy Certificates used for establishing
secure connection. Then those documents must be processed by dedicated
Policy Decision Point plugin (see section \ref{sec:delegation_pdp})
to make a final decision based on collected Policies and various information
about client's identity and requested operation. Service or MCC chain
supporting Delegation Restrictions must accept negative decision of
this PDP as final and do not override it with any other decision based
on other policies.


\section{Delegation Collector\label{sec:delegation_collector}}

This Security Attribute is collected by dedicated Security Handler
plugin named {}``\emph{delegation.pdp}'' avaialble in \emph{arcpdc}
plugins' library. It extracts policy document stored inside X509 certificate
proxy extension as defined in RFC3820 and described in Section \ref{sec:delegation_architectire}.
All proxy certificates in a chain provided by client are examined
and all available policies are extracted. 

Extracted content is converted into XML document. Then document is
checked to be of Arc Policy kind. If policy is not recognized as Arc
Policy procedure fails and that causes failure of comunication.

Proxy certificates with \emph{id-ppl-inheritAll} property are passed
through and no policy document is generated for them. Proxies with
other type of policies including \emph{id-ppl-independent} are not
accepted and generate immediate failure.


\section{Delegation PDP\label{sec:delegation_pdp}}

DelegationPDP is similar to ArcPDP described above except that it
takes it's Policy documents directly from Security Attributes. Differently
from Arc PDP it is meant to be used for enforcing policies defined
by client.


\chapter{Schemas and descriptions\label{sec:schemas}}


\section{Policy}


\section{Request}


\section{ArcPDP configuration}

\begin{verbatim}
<?xml version="1.0" encoding="UTF-8"?>
<xsd:schema
  xmlns:xsd="http://www.w3.org/2001/XMLSchema"
  xmlns="http://www.nordugrid.org/schemas/ArcPDP"
  targetNamespace="http://www.nordugrid.org/schemas/ArcPDP"
  elementFormDefault="qualified">

    <xsd:complexType name="FilterType">
        <!-- This element defines Security Attributes to select and reject.
            If there are no Select elements all Attributes are used except
            those listed in Reject elements.  -->
        <xsd:sequence>
            <xsd:element name="Select" type="xsd:string" minOccurs="0" maxOccurs="unbounded"/>
            <xsd:element name="Reject" type="xsd:string" minOccurs="0" maxOccurs="unbounded"/>
        </xsd:sequence>
    </xsd:complexType>
    <xsd:element name="Filter" type="FilterType"/>

    <!--
        This element specified file containing policy document.
        There may be multiple such elements.
    -->
    <xsd:element name="PolicyStore" type="xsd:string"/>

</xsd:schema>
\end{verbatim}


\section{DelegationPDP configuration}

\begin{verbatim}
<?xml version="1.0" encoding="UTF-8"?>
<xsd:schema
  xmlns:xsd="http://www.w3.org/2001/XMLSchema"
  xmlns="http://www.nordugrid.org/schemas/DelegationPDP"
  targetNamespace="http://www.nordugrid.org/schemas/DelegationPDP"
  elementFormDefault="qualified">

    <xsd:complexType name="FilterType">
        <!-- This element defines Security Attributes to select and reject.
            If there are no Select elements all Attributes are used except
            those listed in Reject elements.  -->
        <xsd:sequence>
            <xsd:element name="Select" type="xsd:string" minOccurs="0" maxOccurs="unbounded"/>
            <xsd:element name="Reject" type="xsd:string" minOccurs="0" maxOccurs="unbounded"/>
        </xsd:sequence>
    </xsd:complexType>
    <xsd:element name="Filter" type="FilterType"/>

</xsd:schema>
\end{verbatim}


\section{Delegation interface\label{sec:delegation_interface}}

Web Service delegation interface. Each ARC1 service wishing to accept
delegated credentials implements this interface. Here is how delegation
procedure works:

\begin{itemize}
\item Step 1

\begin{itemize}
\item Client contacts service requesting operation DelegateCredentialsInit.
This operation has no arguments.
\item Service responds with \emph{DelegateCredentialsInitResponse} message
with element \emph{TokenRequest}. That element contains credentials
request generated by service in \emph{Value}. Type of request is defined
by attribute \emph{Format}. Currently only supported format is \emph{x509}.
Along with \emph{Value} service provides identifier \emph{Id} which
is used in second step.
\end{itemize}
\item Step 2

\begin{itemize}
\item Client requests \emph{UpdateCredentials} operation with \emph{DelegatedToken}
argument. This element contains \emph{Value} with serialized delegated
credentials and \emph{Id} which links it to first step. Delegated
token element may also contain multiple \emph{Reference} elements.
\emph{Reference} refers to the object which these credentials should
be applied to in a way specific to the service. \\
The \emph{DelegatedToken} element must also be used for delegating
credentials when Step 2 is combined with other operations on service.
\item Service responds with empty \emph{UpdateCredentialsResponse} message.
\end{itemize}
\end{itemize}
\begin{verbatim}<?xml version="1.0" encoding="UTF-8"?>
<wsdl:definitions targetNamespace="http://www.nordugrid.org/schemas/delegation"
 xmlns:SOAP-ENV="http://schemas.xmlsoap.org/soap/envelope/"
 xmlns:SOAP-ENC="http://schemas.xmlsoap.org/soap/encoding/"
 xmlns:xsi="http://www.w3.org/2001/XMLSchema-instance"
 xmlns:xsd="http://www.w3.org/2001/XMLSchema"
 xmlns:soap="http://schemas.xmlsoap.org/wsdl/soap/"
 xmlns:wsdl="http://schemas.xmlsoap.org/wsdl/"
 xmlns:wsa="http://www.w3.org/2005/08/addressing"
 xmlns:deleg="http://www.nordugrid.org/schemas/delegation">

  <wsdl:types>
    <xsd:schema targetNamespace="http://www.nordugrid.org/schemas/delegation">

      <!-- Common types -->

      <xsd:simpleType name="TokenFormatType">
        <xsd:restriction base="xsd:string">
          <xsd:enumeration value="x509"/>
        </xsd:restriction>
      </xsd:simpleType>

      <xsd:complexType name="ReferenceType">
        <xsd:sequence>
          <xsd:any namespace="##other" processContents="lax" minOccurs="0" maxOccurs="unbounded"/>
        </xsd:sequence>
      </xsd:complexType>

      <xsd:complexType name="DelegatedTokenType">
        <xsd:sequence>
          <xsd:element name="Id" type="xsd:string"/>
          <xsd:element name="Value" type="xsd:string"/>
          <xsd:element name="Reference" type="deleg:ReferenceType" minOccurs="0" maxOccurs="unbounded"/>
        </xsd:sequence>
        <xsd:attribute name="Format" type="deleg:TokenFormatType" use="required"/>
      </xsd:complexType>
      <xsd:element name="DelegatedToken" type="deleg:DelegatedTokenType"/>

      <xsd:complexType name="TokenRequestType">
        <xsd:sequence>
          <xsd:element name="Id" type="xsd:string"/>
          <xsd:element name="Value" type="xsd:string"/>
        </xsd:sequence>
        <xsd:attribute name="Format" type="deleg:TokenFormatType" use="required"/>
      </xsd:complexType>
      <xsd:element name="TokenRequest" type="deleg:TokenRequestType"/>

      <!-- Types for messages -->

      <xsd:complexType name="DelegateCredentialsInitRequestType">
        <xsd:sequence>
        </xsd:sequence>
      </xsd:complexType>
      <xsd:element name="DelegateCredentialsInit" type="deleg:DelegateCredentialsInitRequestType"/>

      <xsd:complexType name="DelegateCredentialsInitResponseType">
        <xsd:sequence>
          <xsd:element name="TokenRequest" type="deleg:TokenRequestType"/>
        </xsd:sequence>
      </xsd:complexType>
      <xsd:element name="DelegateCredentialsInitResponse" type="deleg:DelegateCredentialsInitResponseType"/>

      <xsd:complexType name="UpdateCredentialsRequestType">
        <xsd:sequence>
          <xsd:element name="DelegatedToken" type="deleg:DelegatedTokenType"/>
        </xsd:sequence>
      </xsd:complexType>
      <xsd:element name="UpdateCredentials" type="deleg:UpdateCredentialsRequestType"/>

      <xsd:complexType name="UpdateCredentialsResponseType">
        <xsd:sequence>
        </xsd:sequence>
      </xsd:complexType>
      <xsd:element name="UpdateCredentialsResponse" type="deleg:UpdateCredentialsResponseType"/>

      <!-- Faults -->

      <xsd:complexType name="UnsupportedFaultType">
        <xsd:sequence>
          <xsd:element name="Description" type="xsd:string" minOccurs="0"/>
        </xsd:sequence>
      </xsd:complexType>
      <xsd:element name="UnsupportedFault" type="deleg:UnsupportedFaultType"/>

      <xsd:complexType name="ProcessingFaultType">
        <xsd:sequence>
          <xsd:element name="Description" type="xsd:string" minOccurs="0"/>
        </xsd:sequence>
      </xsd:complexType>
      <xsd:element name="ProcessingFault" type="deleg:ProcessingFaultType"/>

      <xsd:complexType name="WrongReferenceFaultType">
        <xsd:sequence>
          <xsd:element name="Description" type="xsd:string" minOccurs="0"/>
        </xsd:sequence>
      </xsd:complexType>
      <xsd:element name="WrongReferenceFault" type="deleg:WrongReferenceFaultType"/>

    </xsd:schema>
  </wsdl:types>

  <wsdl:message name="DelegateCredentialsInitRequest">
    <wsdl:part name="DelegateCredentialsInitRequest" element="deleg:DelegateCredentialsInit"/>
  </wsdl:message>

  <wsdl:message name="DelegateCredentialsInitResponse">
    <wsdl:part name="DelegateCredentialsInitResponse" element="deleg:DelegateCredentialsInitResponse"/>
  </wsdl:message>

  <wsdl:message name="UpdateCredentialsRequest">
    <wsdl:part name="UpdateCredentialsRequest" element="deleg:UpdateCredentials"/>
  </wsdl:message>

  <wsdl:message name="UpdateCredentialsResponse">
    <wsdl:part name="UpdateCredentialsResponse" element="deleg:UpdateCredentialsResponse"/>
  </wsdl:message>

  <wsdl:message name="UnsupportedFault">
    <wsdl:part name="Detail" element="deleg:UnsupportedFault"/>
  </wsdl:message>

  <wsdl:message name="ProcessingFault">
    <wsdl:part name="Detail" element="deleg:ProcessingFault"/>
  </wsdl:message>

  <wsdl:message name="WrongReferenceFault">
    <wsdl:part name="Detail" element="deleg:WrongReferenceFault"/>
  </wsdl:message>

  <wsdl:portType name="DelegationPortType">

   <wsdl:operation name="DelegateCredentialsInit">
      <wsdl:documentation>
      </wsdl:documentation>
      <wsdl:input name="DelegateCredentialsInitRequest"
        message="deleg:DelegateCredentialsInitRequest"/>
      <wsdl:output name="DelegateCredentialsInitResponse"
        message="deleg:DelegateCredentialsInitResponse"/>
      <wsdl:fault name="UnsupportedFault"
        message="deleg:UnsupportedFault"/>
      <wsdl:fault name="ProcessingFault"
        message="deleg:ProcessingFault"/>
    </wsdl:operation>

   <wsdl:operation name="UpdateCredentials">
      <wsdl:documentation>
      </wsdl:documentation>
      <wsdl:input name="UpdateCredentialsRequest"
        message="deleg:UpdateCredentialsRequest"/>
      <wsdl:output name="UpdateCredentialsResponse"
        message="deleg:UpdateCredentialsResponse"/>
      <wsdl:fault name="UnsupportedFault"
        message="deleg:UnsupportedFault"/>
      <wsdl:fault name="ProcessingFault"
        message="deleg:ProcessingFault"/>
      <wsdl:fault name="WrongReferenceFault"
        message="deleg:WrongReferenceFault"/>
    </wsdl:operation>

  </wsdl:portType>

  <wsdl:binding name="DelegationBinding" type="deleg:DelegationPortType">

    <soap:binding style="document" transport="http://schemas.xmlsoap.org/soap/http"/>

    <wsdl:operation name="DelegateCredentialsInit">
      <soap:operation soapAction="DelegateCredentialsInit"/>
      <wsdl:input name="DelegateCredentialsInitRequest">
         <soap:body use="literal"/>
      </wsdl:input>
      <wsdl:output name="DelegateCredentialsInitResponse">
         <soap:body use="literal"/>
      </wsdl:output>
    </wsdl:operation>

    <wsdl:operation name="UpdateCredentials">
      <soap:operation soapAction="UpdateCredentials"/>
      <wsdl:input name="UpdateCredentialsRequest">
        <soap:body use="literal"/>
      </wsdl:input>
      <wsdl:output name="UpdateCredentialsResponse">
        <soap:body use="literal"/>
      </wsdl:output>
    </wsdl:operation>

  </wsdl:binding>

</wsdl:definitions>
\end{verbatim}


\chapter*{Examples:}


\section*{Policy}


\section*{Request}


\section*{Delegation Policy}


\chapter*{Template examples}

In expressions, the following operands are allowed: \begin{shaded}
\verb#=   !=   >   <   >=   <=# \end{shaded}

\begin{framed} Examples of URLs are:\\
 \\
 \verb#http://grid.domain.org/dir/script.sh#\\
 \verb#gsiftp://grid.domain.org:2811;threads=10/dir/input_12378.dat#\\
 \verb#ldap://grid.domain.org:389/lc=collection1,rc=Nordugrid,dc=nordugrid,dc=org#\\
 \verb#rc://grid.domain.org/lc=collection1,rc=Nordugrid,dc=nordugrid,dc=org/zebra/f1.zebra#
\verb#file:///home/auser/griddir/steer.cra#\\
 \end{framed}

%\subsection{Subsection}
%\label{sec:subsection}


%
\begin{figure}[ht]
 \centering{{\scalebox{0.9}{\includegraphics{tex/ng-logo}}} 


\caption{\label{fig:myfigure1}The figure shows a logo.}

} 
\end{figure}


\bibliography{grid}
 
\end{document}
