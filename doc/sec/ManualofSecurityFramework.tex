% This file was converted to LaTeX by Writer2LaTeX ver. 0.5
% see http://www.hj-gym.dk/~hj/writer2latex for more info
\documentclass{article}
\usepackage[ascii]{inputenc}
\usepackage[T1]{fontenc}
\usepackage[english]{babel}
\usepackage{amsmath,amssymb,amsfonts,textcomp}
\usepackage{color}
\usepackage{array}
\usepackage{supertabular}
\usepackage{hhline}
\usepackage{hyperref}
\hypersetup{pdftex, colorlinks=true, linkcolor=blue, citecolor=blue, filecolor=blue, pagecolor=blue, urlcolor=blue, pdftitle=, pdfauthor=qiangwz, pdfsubject=, pdfkeywords=}
\usepackage[pdftex]{graphicx}
% Outline numbering
\setcounter{secnumdepth}{3}
\renewcommand\thesection{\arabic{section}.}
\renewcommand\thesubsection{\arabic{subsection}.}
\renewcommand\thesubsubsection{\arabic{subsection}.\arabic{subsubsection}.}
\makeatletter
\newcommand\arraybslash{\let\\\@arraycr}
\makeatother
% List styles
\newcounter{saveenum}
\newcommand\liststyleWWviiiNumxi{%
\renewcommand\theenumi{\arabic{enumi}}
\renewcommand\theenumii{\arabic{enumii}}
\renewcommand\theenumiii{\arabic{enumiii}}
\renewcommand\theenumiv{\arabic{enumiv}}
\renewcommand\labelenumi{\theenumi.}
\renewcommand\labelenumii{\theenumii.}
\renewcommand\labelenumiii{\theenumiii.}
\renewcommand\labelenumiv{\theenumiv.}
}
\newcommand\liststyleWWviiiNumxii{%
\renewcommand\labelitemi{[F0B7?]}
\renewcommand\labelitemii{[F0B7?]}
\renewcommand\labelitemiii{[F0B7?]}
\renewcommand\labelitemiv{[F0B7?]}
}
\newcommand\liststyleWWviiiNumxv{%
\renewcommand\theenumi{\arabic{enumi}}
\renewcommand\theenumii{\arabic{enumii}}
\renewcommand\theenumiii{\arabic{enumiii}}
\renewcommand\theenumiv{\arabic{enumiv}}
\renewcommand\labelenumi{\theenumi.}
\renewcommand\labelenumii{\theenumii.}
\renewcommand\labelenumiii{\theenumiii.}
\renewcommand\labelenumiv{\theenumiv.}
}
\newcommand\liststyleWWviiiNumxvi{%
\renewcommand\theenumi{\arabic{enumi}}
\renewcommand\theenumii{\arabic{enumii}}
\renewcommand\theenumiii{\arabic{enumiii}}
\renewcommand\theenumiv{\arabic{enumiv}}
\renewcommand\labelenumi{\theenumi.}
\renewcommand\labelenumii{\theenumii.}
\renewcommand\labelenumiii{\theenumiii.}
\renewcommand\labelenumiv{\theenumiv.}
}
\newcommand\liststyleWWviiiNumxxiv{%
\renewcommand\theenumi{\arabic{enumi}}
\renewcommand\theenumii{\arabic{enumii}}
\renewcommand\theenumiii{\arabic{enumiii}}
\renewcommand\theenumiv{\arabic{enumiv}}
\renewcommand\labelenumi{\theenumi.}
\renewcommand\labelenumii{\theenumii.}
\renewcommand\labelenumiii{\theenumiii.}
\renewcommand\labelenumiv{\theenumiv.}
}
% Page layout (geometry)
\setlength\paperwidth{8.2673in}
\setlength\paperheight{11.6925in}
\setlength\voffset{-1in}
\setlength\hoffset{-1in}
\setlength\topmargin{1in}
\setlength\oddsidemargin{1.25in}
\setlength\textheight{9.0315in}
\setlength\textwidth{5.7672997in}
\setlength\footskip{0.661in}
\setlength\headheight{0cm}
\setlength\headsep{0cm}
% Footnote rule
\setlength{\skip\footins}{0.0469in}
\renewcommand\footnoterule{\vspace*{-0.0071in}\setlength\leftskip{0pt}\setlength\rightskip{0pt plus 1fil}\noindent\textcolor{black}{\rule{0.25\columnwidth}{0.0071in}}\vspace*{0.0398in}}
% Pages styles
\makeatletter
\newcommand\ps@Convertv{
  \renewcommand\@oddhead{}
  \renewcommand\@evenhead{}
  \renewcommand\@oddfoot{}
  \renewcommand\@evenfoot{\@oddfoot}
  \renewcommand\thepage{\arabic{page}}
}
\newcommand\ps@Standard{
  \renewcommand\@oddhead{}
  \renewcommand\@evenhead{}
  \renewcommand\@oddfoot{\thepage{}}
  \renewcommand\@evenfoot{\@oddfoot}
  \renewcommand\thepage{\arabic{page}}
}
\makeatother
\pagestyle{Standard}
\setlength\tabcolsep{1mm}
\renewcommand\arraystretch{1.3}
\title{}
\begin{document}
\begin{flushleft}
\tablehead{}\begin{supertabular}{m{1.3809599in}m{2.0976598in}}
\begin{center}
\includegraphics[width=1.422in,height=0.9091in]{ManualofSecurityFramework-img1.jpg}
\end{center}
 &
\selectlanguage{english}\scshape\color{black} \newline
\newline
NORDUGRID\\
\end{supertabular}
\end{flushleft}

\bigskip


\bigskip

\begin{flushleft}
\tablehead{}\begin{supertabular}{m{0.83445984in}m{4.78586in}m{0.83655983in}}
\multicolumn{3}{m{6.61436in}}{\raggedleft
\selectlanguage{english}\color{black} NORDUGRID-TECH-17\newline
15/11/08}\\
\multicolumn{3}{m{6.61436in}}{\centering
{\selectlanguage{english}\bfseries\scshape\color{black} Manual of
Security Framework}\par

~
}\\
\multicolumn{3}{m{6.61436in}}{\centering
\selectlanguage{english}\color{black} W.Qiang, A.Konstantinov}\\
\multicolumn{3}{m{6.61436in}}{\centering
\selectlanguage{english}\itshape\color{black} Abstract}\\
~
 &
\selectlanguage{english}\color{black} This document is about user manual
of security framework, as well as development manual and administration
manual. &
~
\\
\end{supertabular}
\end{flushleft}
\clearpage{\centering\selectlanguage{english}\bfseries\scshape\color{black}
Table of Contents
\par}

\setcounter{tocdepth}{9}
\renewcommand\contentsname{}
\tableofcontents

\bigskip

\bigskip

\bigskip
\clearpage\clearpage\setcounter{page}{1}\pagestyle{Convertv}

\bigskip

\subsection{1. User Manual}
{\selectlanguage{english}\color{black}
This section describes how to configure and use SecHandler and PDP
elements included in the ARC1 and provides few examples of the ARC
Policy documents. The target readers are those users who will use the
ARC1 middleware. Currently this section is very short on details. It is
going to be continuously extended. Especially taking user feedback into
account.}

\subsubsection[Authorization SecHandler and PDPs]{Authorization
SecHandler and PDPs}
{\selectlanguage{english}\color{black}
There is a specific Authorization SecHandler (arc.authz) which is
implemented for calling the Policy Decision Points (PDP) and serves as
their container. }

{\selectlanguage{english}\color{black}
Usually the Authorization SecHandler and included PDPs are used on the
service side of communication channel. Although it is also possible to
use them on the client side. All possibilities are achieved by
modifying the configuration file (hereafter mentioned as service.xml)
and possibly providing the authorization policy in a separate file.}

{\selectlanguage{english}\color{black}
Here the {\textquotedblleft}echo{\textquotedblright} test service is
used to explaining the usage, but the explanation applies to other
services as well.}

{\selectlanguage{english}\color{black}
The procedure for configuring Authorization SecHandler in service.xml is
following:}

\liststyleWWviiiNumxi
\begin{enumerate}
\item {\selectlanguage{english}\color{black}
Add the Authorization SecHandler as child element
{\textless}SecHandler/{\textgreater} of
{\textless}Service/{\textgreater} element. }
\end{enumerate}
{\selectlanguage{english}\color{black}
The {\textquotedblleft}name{\textquotedblright} and
{\textquotedblleft}event{\textquotedblright} attribute of
{\textless}SecHandler/{\textgreater} element are both important. The
{\textquotedblleft}name{\textquotedblright} attribute is used for
distinguishing betwenn loaded SecHandler objects. The
{\textquotedblleft}event{\textquotedblright} attribute defines for
which message authorization would be enforced. Usually and reasonably
it is done for {\textquotedblleft}incoming{\textquotedblright}
messages. But some services and other Message Chain components may
define other internal types of messages. For possible values please
refer to documentation of particular Service or MCC. In our particular
case {\textquotedblleft}echo{\textquotedblright} service only supports
{\textquotedblleft}incoming{\textquotedblright} messages for this
purpose.}

\liststyleWWviiiNumxi
\setcounter{saveenum}{\value{enumi}}
\begin{enumerate}
\setcounter{enumi}{\value{saveenum}}
\item {\selectlanguage{english}\color{black}
Add the PDP configuration as \ child element
{\textless}PDP/{\textgreater} under
{\textless}SecHandler/{\textgreater}. Currently there are four usable
PDPs distributed as part of the ARC1 middleware:}
\end{enumerate}
\liststyleWWviiiNumxii
\begin{itemize}
\item {\selectlanguage{english}\color{black}
simplelist.pdp -- compares Subject of user{\textquotesingle}s X509
certificate to those stored in a file.}
\item {\selectlanguage{english}\color{black}
arc.pdp -- compares authorization related information parsed from
message at various processing steps \ to Policy document specified in
configuration of this PDP.}
\item {\selectlanguage{english}\color{black}
pdpservice.invoker - composes the ARC Request, puts request into SOAP
message, and invokes the remote PDP service to get the response SOAP
which includes authorization decision. The PDP service functionality is
similar to arc.pdp.}
\item {\selectlanguage{english}\color{black}
delegation.pdp -- compares authorization related information parsed from
message at various processing steps and Policy document embedded in
proxy certificate used by remote side.}
\end{itemize}
{\selectlanguage{english}\color{black}
Default behavior of Authorization SecHandler is to execute all
configured PDPs sequentially till eithe r one of them fails or all
produced positive results. This behavior may be modified by attribute
{\textquotedblleft}action{\textquotedblright} of
{\textless}PDP/{\textgreater} element.}

{\selectlanguage{english}\upshape\color{black}
The description of PDP configuration and ARC Policy example are
available in Section }

\subsubsection[Delegation SecHandler, Delegation PDP and Proxy
Certificate Generation]{Delegation SecHandler, Delegation PDP and Proxy
Certificate Generation}
{\selectlanguage{english}\upshape\color{black}
Delegation SecHandler and Delegation PDP in their current state provide
an infrastructure for limiting capabilities of delegated credentials.
Their collect and process policies attached to X509 Proxy Certificates
respectively. Hence to have delegation restriction working both must be
enabled in configuration of service. Configuration of Delegation
SecHandler is described in section . }

{\selectlanguage{english}\color{black}
The possible location for Delegation PDP is inside Authorization
SecHandler (arc.authz). Depending on how fine grained policy of
delegated credentials is supposed to be corresponding Authrorization
SecHandler may be attached to different MCCs or directly to Service
component. However, the precondition for using Delegation PDP is that
there must be Delegation SecHandler instantiated earlier in chain.}

{\selectlanguage{english}\color{black}
On the client side, command line utility
{\textquotedblleft}approxy{\textquotedblright} utility can be used to
generate Proxy Certificate with Delegation Policy embedded.}

{\selectlanguage{english}\color{black}
Normally approxy appears in \$ARC\_LOCATION/bin. The usage of approxy is
like:}

{\selectlanguage{english}\color{black}
approxy -P proxy.pem -C cert.pem -K key.pem -c constraints}

{\selectlanguage{english}\color{black}
By using argument {\textquotedbl}-c{\textquotedbl}, some constraints can
be specified for proxy certificate. Currently, the life time can be
limited by using {\textquotedbl}-c validityStart=...{\textquotedbl} and
{\textquotedbl}-c validityEnd=...{\textquotedbl}, {\textquotedbl}-c
validityStart=...{\textquotedbl} and {\textquotedbl}-c
validityPeriod=...{\textquotedbl}. Like for example }

{\selectlanguage{english}\color{black}
{}-c validityStart=2008-05-29T10:20:30Z}

{\selectlanguage{english}\color{black}
{}-c validityEnd=2008-06-29T10:20:30Z}

{\selectlanguage{english}\color{black}
The Delegation Policy can be specified by using {\textquotedbl}-c
proxyPolicyFile=...{\textquotedbl} or {\textquotedbl}-c
proxyPolicy=...{\textquotedbl}. Like }

{\selectlanguage{english}\color{black}
{}-c proxyPolicyFile=delegation\_policy.xml}

{\selectlanguage{english}\upshape\color{black}
The Delegation Policy is the same as the ARC Policy explained in section
. Simple example below renders delegated credentials usable only for
contacting service attached to HTTP communication channel under path
/arex (line 6) \ and allows HTTP operation POST (line 9) on it. }

\liststyleWWviiiNumxv
\begin{enumerate}
\item {\selectlanguage{english}\ttfamily\color{black}
{\textless}?xml version={\textquotedbl}1.0{\textquotedbl}
encoding={\textquotedbl}UTF-8{\textquotedbl}?{\textgreater}}
\item {\selectlanguage{english}\color{black}
\texttt{{\textless}Policy
xmlns={\textquotedbl}}\url{http://www.nordugrid.org/schemas/policy-arc}\texttt{{\textquotedbl}}}
\item {\selectlanguage{english}\ttfamily\color{black}
\ \ \ \ \ \ \ \ PolicyId={\textquotedbl}sm-example:policy1{\textquotedbl}
CombiningAlg={\textquotedbl}Deny-Overrides{\textquotedbl}{\textgreater}}
\item {\selectlanguage{english}\ttfamily\color{black}
\ \ \ {\textless}Rule RuleId={\textquotedbl}rule1{\textquotedbl}
Effect={\textquotedbl}Permit{\textquotedbl}{\textgreater}}
\item {\selectlanguage{english}\ttfamily\color{black}
\ \ \ \ \ \ {\textless}Resources{\textgreater}}
\item {\selectlanguage{english}\ttfamily\color{black}
\ \ \ \ \ \ \ \ \ {\textless}Resource
Type={\textquotedbl}string{\textquotedbl}
AttributeId={\textquotedbl}http://www.nordugrid.org/schemas/policy-arc/types/http/path{\textquotedbl}{\textgreater}/arex{\textless}/Resource{\textgreater}}
\item {\selectlanguage{english}\ttfamily\color{black}
\ \ \ \ \ \ {\textless}/Resources{\textgreater}}
\item {\selectlanguage{english}\ttfamily\color{black}
\ \ \ \ \ \ {\textless}Actions{\textgreater}}
\item {\selectlanguage{english}\ttfamily\color{black}
\ \ \ \ \ \ \ \ \ {\textless}Action
Type={\textquotedbl}string{\textquotedbl}
AttributeId={\textquotedbl}http://www.nordugrid.org/schemas/policy-arc/types/http/method{\textquotedbl}{\textgreater}POST{\textless}/Action{\textgreater}}
\item {\selectlanguage{english}\ttfamily\color{black}
\ \ \ \ \ \ {\textless}/Actions{\textgreater}}
\item {\selectlanguage{english}\ttfamily\color{black}
\ \ \ {\textless}/Rule{\textgreater}}
\item {\selectlanguage{english}\ttfamily\color{black}
{\textless}/Policy{\textgreater}}
\end{enumerate}
{\selectlanguage{english}\color{black}
Another example of delegation policy is presented below. This policy
restricts usage of delegated credentials to SOAP operation
CreateActivity (line 5) of Basic Execution Service (BES) [4] namespace
(line 9). Such policy could be embedded into credentials delegated to
high level Brokering service performing Grid job submission to low
level BES on behalf of user.}

\liststyleWWviiiNumxvi
\begin{enumerate}
\item {\selectlanguage{english}\ttfamily\color{black}
{\textless}?xml version={\textquotedbl}1.0{\textquotedbl}
encoding={\textquotedbl}UTF-8{\textquotedbl}?{\textgreater}}
\item {\selectlanguage{english}\ttfamily\color{black}
{\textless}Policy
xmlns={\textquotedbl}http://www.nordugrid.org/schemas/policy-arc{\textquotedbl}
PolicyId={\textquotedbl}sm-example:policy1{\textquotedbl}
CombiningAlg={\textquotedbl}Deny-Overrides{\textquotedbl}{\textgreater}}
\item {\selectlanguage{english}\ttfamily\color{black}
\ \ \ {\textless}Rule RuleId={\textquotedbl}rule1{\textquotedbl}
Effect={\textquotedbl}Permit{\textquotedbl}{\textgreater}}
\item {\selectlanguage{english}\ttfamily\color{black}
\ \ \ \ \ \ {\textless}Actions{\textgreater}}
\item {\selectlanguage{english}\ttfamily\color{black}
\ \ \ \ \ \ \ \ \ {\textless}Action
Type={\textquotedbl}string{\textquotedbl}
AttributeId={\textquotedbl}http://www.nordugrid.org/schemas/policy-arc/types/soap/operation{\textquotedbl}{\textgreater}CreateActivity{\textless}/Action{\textgreater}}
\item {\selectlanguage{english}\ttfamily\color{black}
\ \ \ \ \ \ {\textless}/Actions{\textgreater}}
\item {\selectlanguage{english}\ttfamily\color{black}
\ \ \ \ \ \ {\textless}Conditions{\textgreater}}
\item {\selectlanguage{english}\ttfamily\color{black}
\ \ \ \ \ \ \ \ {\textless}Condition{\textgreater}}
\item {\selectlanguage{english}\ttfamily\color{black}
\ \ \ \ \ \ \ \ \ \ {\textless}Attribute
Type={\textquotedbl}string{\textquotedbl}
AttributeId={\textquotedbl}http://www.nordugrid.org/schemas/policy-arc/types/soap/namespace{\textquotedbl}{\textgreater}http://schemas.ggf.org/bes/2006/08/bes-factory{\textless}/Attribute{\textgreater}}
\item {\selectlanguage{english}\ttfamily\color{black}
\ \ \ \ \ \ \ \ {\textless}/Condition{\textgreater}}
\item {\selectlanguage{english}\ttfamily\color{black}
\ \ \ \ \ \ {\textless}/Conditions{\textgreater}}
\item {\selectlanguage{english}\ttfamily\color{black}
\ \ \ {\textless}/Rule{\textgreater}}
\item {\selectlanguage{english}\ttfamily\color{black}
{\textless}/Policy{\textgreater}}
\end{enumerate}
\subsubsection{VOMS Proxy Certificate}
{\selectlanguage{english}\color{black}
The commonly used voms proxy certificate can be used in ARC1 for
authentication as normal proxy certificate, and attribute acquiring in
order to make authorization decision based on attributes.}

{\selectlanguage{english}\color{black}
1.How to create voms proxy certificate}

{\selectlanguage{english}\color{black}
Currently the proxy creation utility in ARC1---arcproxy can not be used
for creating voms proxy certificate because the GSI-based communication
requirement from voms server side has not been supported in ARC1 (ARC1
uses standard SSL/TLS communication). So the way to create voms proxy
certificate is still to use the
{\textquotedblleft}voms-proxy-init{\textquotedblright} utility.}

{\selectlanguage{english}\color{black}
2.How to use voms proxy certificate in ARC1}

{\selectlanguage{english}\color{black}
The attribute certificate (AC, which is created by voms server and then
embedded in voms proxy certificate \ as one of the certificate
extension by {\textquotedblleft}voms-proxy-init{\textquotedblright}
client) \ will be parsed by TLS handling plugin (called MCCTLS) of
ARC1, and saved in the message context in the format
{\textquotedblleft}grantor=knowarc://testvoms.knowarc.eu:50000/knowarc:role=guest{\textquotedblright}.}

{\selectlanguage{english}\color{black}
Service administrator needs to configure the trusted DN chain for
verifying the AC under tls MCC{\textquotesingle}s configurationwith the
following format. You should specify one node
({\textless}tls:VOMSCertTrustDNChain/{\textgreater}) for each vo which
you trust. The reason why using trusted DN chain is for service to
restrict which vo server is trusted. }

{\selectlanguage{english}\color{black}
In the following table, the first
{\textless}tls:VOMSCertTrustDNChain{\textgreater} in the following
table defines two items: the first one is the DN of voms server, the
second line is the DN of the corresponding CA. The second
{\textless}tls:VOMSCertTrustDNChain{\textgreater} defines regular
expression for matching, the DN of voms server and DN of the
corresponding CA should both match it. The third one defines the
external file which includes the
{\textless}tls:VOMSCertTrustDNChain{\textgreater}.}

{\selectlanguage{english}\color{black}
\ \ \ \ \ \ \ \ \ \ \ \ {\textless}tls:VOMSCertTrustDNChain{\textgreater}}


\bigskip

{\selectlanguage{english}\color{black}
\ \ \ \ \ \ \ \ \ \ \ \ \ \ {\textless}tls:VOMSCertTrustDN{\textgreater}/O=Grid/O=NorduGrid/OU=fys.uio.no/CN=ABCDE{\textless}/tls:VOMSCertTrustDN{\textgreater}}


\bigskip

{\selectlanguage{english}\color{black}
\ \ \ \ \ \ \ \ \ \ \ \ \ \ {\textless}tls:VOMSCertTrustDN{\textgreater}/O=Grid/O=NorduGrid/CN=NorduGrid
Certification Authority{\textless}/tls:VOMSCertTrustDN{\textgreater}}


\bigskip

{\selectlanguage{english}\color{black}
\ \ \ \ \ \ \ \ \ \ \ \ {\textless}/tls:VOMSCertTrustDNChain{\textgreater}}


\bigskip

{\selectlanguage{english}\color{black}
\ \ \ \ \ \ \ \ \ \ \ \ {\textless}tls:VOMSCertTrustDNChain{\textgreater}}


\bigskip

{\selectlanguage{english}\color{black}
\ \ \ \ \ \ \ \ \ \ \ \ \ \ {\textless}tls:VOMSCertTrustRegex{\textgreater}\^{}/O=Grid/O=NorduGrid{\textless}/tls:VOMSCertTrustRegex{\textgreater}}


\bigskip

{\selectlanguage{english}\color{black}
\ \ \ \ \ \ \ \ \ \ \ \ {\textless}/tls:VOMSCertTrustDNChain{\textgreater}}


\bigskip

{\selectlanguage{english}\color{black}
\ \ \ \ \ \ \ \ \ \ \ \ {\textless}tls:VOMSCertTrustDNChainsLocation{\textgreater}./pathto/external/file{\textless}/tls:VOMSCertTrustDNChainsLocation{\textgreater}}


\bigskip

{\selectlanguage{english}\color{black}
Service administrator can then use the attribute into access control
policy:}

\liststyleWWviiiNumxxiv
\begin{enumerate}
\item {\selectlanguage{english}\ttfamily\color{black}
\ \ \ \ \ \ {\textless}Subjects{\textgreater}}
\item {\selectlanguage{english}\ttfamily\color{black}
\ \ \ \ \ \ \ \ \ {\textless}Subject{\textgreater}}
\item {\selectlanguage{english}\ttfamily\color{black}
\ \ \ \ \ \ \ \ \ \ \ {\textless}Attribute
AttributeId={\textquotedbl}http://www.nordugrid.org/schemas/policy-arc/types/tls/vomsattribute{\textquotedbl}
Type={\textquotedbl}string{\textquotedbl}{\textgreater}grantor=knowarc://testvoms.knowarc.eu:50000/knowarc:role=guest{\textless}/Attribute{\textgreater}}
\item {\selectlanguage{english}\ttfamily\color{black}
\ \ \ \ \ \ {\textless}Attribute
AttributeId={\textquotedbl}http://www.nordugrid.org/schemas/policy-arc/types/tls/ca{\textquotedbl}
Type={\textquotedbl}string{\textquotedbl}{\textgreater}/C=NO/ST=Oslo/O=UiO/CN=CA{\textless}/Attribute{\textgreater}}
\item {\selectlanguage{english}\ttfamily\color{black}
\ \ \ \ \ \ \ \ \ \ \ \ {\textless}Attribute
AttributeId={\textquotedbl}http://www.nordugrid.org/schemas/policy-arc/types/tls/identity{\textquotedbl}
Type={\textquotedbl}string{\textquotedbl}{\textgreater}/C=NO/ST=Oslo/O=UiO/CN=test{\textless}/Attribute{\textgreater}}
\item {\selectlanguage{english}\ttfamily\color{black}
\ \ \ \ \ \ \ \ \ {\textless}/Subject{\textgreater}}
\item {\selectlanguage{english}\ttfamily\color{black}
\ \ \ \ \ \ {\textless}/Subjects{\textgreater}}
\end{enumerate}

\bigskip

\subsubsection[UsernameToken SecHandler]{UsernameToken SecHandler}
{\selectlanguage{english}\color{black}
The UsernameToken SecHandler is meant for processing - generating and
extracting - WS-Security [5] UsernameToken from SOAP header. Hence it
must be attached to SOAP MCC of service or/and client communication
channel. }

{\selectlanguage{english}\upshape\color{black}
On the service side, the functionality of extracting UsernameToken may
be configured as described in section . }

{\selectlanguage{english}\color{black}
On the client side, the UsernameToken SecHandler may be configured
either by using client specific methods (for example see
test\_clientinterface.cpp src/tests/echo directory of source tree) or
through generic client configuration file as shown in example below
(also can see share/doc/arc/echo\_client.xml.example under the
installation directory). This example will generate token with username
{\textquotedblleft}user{\textquotedblright} and password
{\textquotedblleft}pass{\textquotedblright} inside any SOAP message
sent by client tools of the ARC1.}

{\selectlanguage{english}\ttfamily\color{black}
{\textless}ArcConfig{\textgreater}}

{\selectlanguage{english}\ttfamily\color{black}
\ {\textless}Plugins
overlay={\textquotedbl}add{\textquotedbl}{\textgreater}{\textless}Name{\textgreater}arcpdc{\textless}/Name{\textgreater}{\textless}/Plugins{\textgreater}}

{\selectlanguage{english}\ttfamily\color{black}
\ {\textless}Chain{\textgreater}}

{\selectlanguage{english}\ttfamily\color{black}
\ \ {\textless}Component
name={\textquotedbl}soap.client{\textquotedbl}{\textgreater}}

{\selectlanguage{english}\ttfamily\color{black}
\ \ \ {\textless}SecHandler
name={\textquotedbl}usernametoken.handler{\textquotedbl}
event={\textquotedbl}outgoing{\textquotedbl}
overlay={\textquotedbl}add{\textquotedbl}{\textgreater}}

{\selectlanguage{english}\ttfamily\color{black}
\ \ \ \ {\textless}Process{\textgreater}generate{\textless}/Process{\textgreater}}

{\selectlanguage{english}\ttfamily\color{black}
\ \ \ \ {\textless}Username{\textgreater}user{\textless}/Username{\textgreater}}

{\selectlanguage{english}\ttfamily\color{black}
\ \ \ \ {\textless}Password{\textgreater}pass{\textless}/Password{\textgreater}}

{\selectlanguage{english}\ttfamily\color{black}
\ \ \ \ {\textless}PasswordEncoding{\textgreater}digest{\textless}/PasswordEncoding{\textgreater}}

{\selectlanguage{english}\ttfamily\color{black}
\ \ \ {\textless}/SecHandler{\textgreater}}

{\selectlanguage{english}\ttfamily\color{black}
\ \ {\textless}/Component{\textgreater}}

{\selectlanguage{english}\ttfamily\color{black}
\ {\textless}/Chain{\textgreater}}

{\selectlanguage{english}\ttfamily\color{black}
{\textless}/ArcConfig{\textgreater}}

\subsubsection[X509Token SecHandler]{X509Token SecHandler}
{\selectlanguage{english}\color{black}
The X509Token SecHandler is meant for processing - generating and
extracting - WS-Security [5] X509Token from SOAP header. Hence it must
be attached to SOAP MCC of service or/and client communication channel.
}

{\selectlanguage{english}\color{black}
On the service side, the functionality of extracting X509Token may be
configured as described in section 7.12. }

{\selectlanguage{english}\color{black}
On the client side, the X509Token SecHandler may be configured either by
using client specific methods (for example see
test\_clientinterface.cpp src/tests/echo directory of source tree) or
through generic client configuration file as shown in example below.
This example will generate x509 token (by using the specified
certificate and key file) inside any SOAP message sent by client tools
of the ARC1.}

{\selectlanguage{english}\ttfamily\color{black}
{\textless}ArcConfig{\textgreater}}

{\selectlanguage{english}\ttfamily\color{black}
\ {\textless}Plugins
overlay={\textquotedbl}add{\textquotedbl}{\textgreater}{\textless}Name{\textgreater}arcpdc{\textless}/Name{\textgreater}{\textless}/Plugins{\textgreater}}

{\selectlanguage{english}\ttfamily\color{black}
\ {\textless}Chain{\textgreater}}

{\selectlanguage{english}\ttfamily\color{black}
\ \ {\textless}Component
name={\textquotedbl}soap.client{\textquotedbl}{\textgreater}}

{\selectlanguage{english}\ttfamily\color{black}
\ \ \ {\textless}SecHandler
name={\textquotedbl}x509token.handler{\textquotedbl}
event={\textquotedbl}outgoing{\textquotedbl}
overlay={\textquotedbl}add{\textquotedbl}{\textgreater}}

{\selectlanguage{english}\ttfamily\color{black}
\ \ \ \ {\textless}Process{\textgreater}generate{\textless}/Process{\textgreater}}

{\selectlanguage{english}\ttfamily\color{black}
\ \ \ \ {\textless}KeyPath{\textgreater}/path/to/key.pem{\textless}/KeyPath{\textgreater}}

{\selectlanguage{english}\ttfamily\color{black}
\ \ \ \ {\textless}CertificatePath{\textgreater}/path/to/cert.pem{\textless}/CertificatePath{\textgreater}}

{\selectlanguage{english}\ttfamily\color{black}
\ \ \ {\textless}/SecHandler{\textgreater}}

{\selectlanguage{english}\ttfamily\color{black}
\ \ {\textless}/Component{\textgreater}}

{\selectlanguage{english}\ttfamily\color{black}
\ {\textless}/Chain{\textgreater}{\textless}/ArcConfig{\textgreater}}


\bigskip

\subsubsection{SP Service and SAML2SSO Service Provider Handler}
{\selectlanguage{english}\color{black}
\ \ \ \ Service Provider servicer (SP service) implements the
functionality of Service Provider in SAML2 Web SSO profile.}

{\selectlanguage{english}\color{black}
\ \ \ \ Normally SP service will not be deployed independently, instead,
it should be deployed together with other services. The SLCS service is
the typical deployment about SP service. On the client side, client
developer should use the ClientSAML2Interface instead of
ClientInterface to call the client functionality. The SLCS client
(arcslcs) is the typical usage of \ ClientSAML2Interface.}

{\selectlanguage{english}\color{black}
\ \ \ \ Once the SP Service has cooperated with user agent and Identity
Provider (external) and succeeded to accomplish the SAML2 SSO profile,
\ the SP service will get the saml authentication assertion which
asserts that the authentication has succeeded, and then SP service will
store this assertion into session context.}

{\selectlanguage{english}\color{black}
\ \ \ \ SAML2 assertion consumer handler
(saml2ssoassertionconsumer.handler) is the security handler which will
understand the authentication assertion and attribute assertion from
IdP, and make authorization decision according to the attribute values
inside these assertions. Currently the SAML2 assertion consumer handler
is an empty security handler and does not effect the services.}

{\selectlanguage{english}\color{black}
\ \ \ \ The example configuration about using SP service and SAML2SSO
assertion consumer handler can be seen in the configuration of SLCS
service. }

{\selectlanguage{english}\color{black}
\ \ \ \ SP service is not only supposed to work together with SLCS
service. If it is used to work together with other services, the client
should be developed on ClientSAML2Interface instead of ClientInterface,
which is the same as that in SLCS client (\textit{arcslcs}).}


\bigskip

\subsubsection{SLCS service and client}
{\selectlanguage{english}\color{black}
\ \ \ \ Short-lived credential service (SLCS) is for signing short-lived
x509 credential for user based user{\textquotesingle}s
username/password credential. Then the user can use the short-lived
x509 credential to access grid services/resources where x509 credential
is required by default. \ SLCS service should depend on the SP (Service
provider) service which is one of the participants of SAML2 SSO profile
(SAML2 SSO profile is used for authenticating based on
username/password credential and getting SAML authentication assertion;
and SAML authentication assertion is then used as basis for signing
short-lived x509 credential).}

{\selectlanguage{english}\color{black}
a. \textbf{SLCS Service} }

{\selectlanguage{english}\color{black}
On the SLCS service side, a typical configuration should be like the
following.}

{\selectlanguage{english}\ttfamily\color{black}
{\textless}?xml version={\textquotedbl}1.0{\textquotedbl}?{\textgreater}
}

{\selectlanguage{english}\ttfamily\color{black}
{\textless}ArcConfig }

{\selectlanguage{english}\ttfamily\color{black}
\ \ xmlns={\textquotedbl}http://www.nordugrid.org/schemas/ArcConfig/2007{\textquotedbl}
}

{\selectlanguage{english}\ttfamily\color{black}
\ \ xmlns:tcp={\textquotedbl}http://www.nordugrid.org/schemas/ArcMCCTCP/2007{\textquotedbl}
}

{\selectlanguage{english}\ttfamily\color{black}
\ \ xmlns:tls={\textquotedbl}http://www.nordugrid.org/schemas/ArcMCCTLS/2007{\textquotedbl}
}

{\selectlanguage{english}\ttfamily\color{black}
\ \ xmlns:arcpdp={\textquotedbl}http://www.nordugrid.org/schemas/ArcPDP{\textquotedbl}
}

{\selectlanguage{english}\ttfamily\color{black}
\ \ xmlns:slcs={\textquotedbl}http://www.nordugrid.org/schemas/ArcConfig/2007/slcs{\textquotedbl}
}

{\selectlanguage{english}\ttfamily\color{black}
{\textgreater} }

{\selectlanguage{english}\ttfamily\color{black}
\ \ \ {\textless}ModuleManager{\textgreater} }

{\selectlanguage{english}\ttfamily\color{black}
\ \ \ \ \ \ \ \ {\textless}Path{\textgreater}.libs/{\textless}/Path{\textgreater}
}

{\selectlanguage{english}\ttfamily\color{black}
\ \ \ \ \ \ \ \ {\textless}Path{\textgreater}../../hed/mcc/http/.libs/{\textless}/Path{\textgreater}
}

{\selectlanguage{english}\ttfamily\color{black}
\ \ \ \ \ \ \ \ {\textless}Path{\textgreater}../../hed/mcc/tls/.libs/{\textless}/Path{\textgreater}
}

{\selectlanguage{english}\ttfamily\color{black}
\ \ \ \ \ \ \ \ {\textless}Path{\textgreater}../../hed/mcc/soap/.libs/{\textless}/Path{\textgreater}
}

{\selectlanguage{english}\ttfamily\color{black}
\ \ \ \ \ \ \ \ {\textless}Path{\textgreater}../../hed/mcc/tcp/.libs/{\textless}/Path{\textgreater}
}

{\selectlanguage{english}\ttfamily\color{black}
\ \ \ \ \ \ \ \ {\textless}Path{\textgreater}../../hed/pdc/.libs/{\textless}/Path{\textgreater}
}

{\selectlanguage{english}\ttfamily\color{black}
\ \ \ \ \ \ \ \ {\textless}Path{\textgreater}../../services/saml/.libs/{\textless}/Path{\textgreater}
}

{\selectlanguage{english}\ttfamily\color{black}
\ \ \ \ \ \ \ \ {\textless}Path{\textgreater}../../services/slcs/.libs/{\textless}/Path{\textgreater}
}

{\selectlanguage{english}\ttfamily\color{black}
\ \ \ \ {\textless}/ModuleManager{\textgreater} }

{\selectlanguage{english}\ttfamily\color{black}
\ \ \ \ {\textless}Plugins{\textgreater}{\textless}Name{\textgreater}mcctcp{\textless}/Name{\textgreater}{\textless}/Plugins{\textgreater}
}

{\selectlanguage{english}\ttfamily\color{black}
\ \ \ \ {\textless}Plugins{\textgreater}{\textless}Name{\textgreater}mcctls{\textless}/Name{\textgreater}{\textless}/Plugins{\textgreater}
}

{\selectlanguage{english}\ttfamily\color{black}
\ \ \ \ {\textless}Plugins{\textgreater}{\textless}Name{\textgreater}mcchttp{\textless}/Name{\textgreater}{\textless}/Plugins{\textgreater}
}

{\selectlanguage{english}\ttfamily\color{black}
\ \ \ \ {\textless}Plugins{\textgreater}{\textless}Name{\textgreater}mccsoap{\textless}/Name{\textgreater}{\textless}/Plugins{\textgreater}
}

{\selectlanguage{english}\ttfamily\color{black}
\ \ \ \ {\textless}Plugins{\textgreater}{\textless}Name{\textgreater}arcpdc{\textless}/Name{\textgreater}{\textless}/Plugins{\textgreater}
}

{\selectlanguage{english}\ttfamily\color{black}
\ \ \ \ {\textless}Plugins{\textgreater}{\textless}Name{\textgreater}saml2sp{\textless}/Name{\textgreater}{\textless}/Plugins{\textgreater}
}

{\selectlanguage{english}\ttfamily\color{black}
\ \ \ \ {\textless}Plugins{\textgreater}{\textless}Name{\textgreater}slcs{\textless}/Name{\textgreater}{\textless}/Plugins{\textgreater}
}

{\selectlanguage{english}\ttfamily\color{black}
\ \ \ \ {\textless}Chain{\textgreater} }

{\selectlanguage{english}\ttfamily\color{black}
\ \ \ \ \ \ \ \ {\textless}Component
name={\textquotedbl}tcp.service{\textquotedbl}
id={\textquotedbl}tcp{\textquotedbl}{\textgreater} }

{\selectlanguage{english}\ttfamily\color{black}
\ \ \ \ \ \ \ \ \ \ \ \ {\textless}next
id={\textquotedbl}tls{\textquotedbl}/{\textgreater} }

{\selectlanguage{english}\ttfamily\color{black}
\ \ \ \ \ \ \ \ \ \ \ \ {\textless}tcp:Listen{\textgreater}{\textless}tcp:Port{\textgreater}60000{\textless}/tcp:Port{\textgreater}{\textless}/tcp:Listen{\textgreater}
}

{\selectlanguage{english}\ttfamily\color{black}
\ \ \ \ \ \ \ \ {\textless}/Component{\textgreater} }

{\selectlanguage{english}\ttfamily\color{black}
\ \ \ \ \ \ \ \ {\textless}Component
name={\textquotedbl}tls.service{\textquotedbl}
id={\textquotedbl}tls{\textquotedbl}{\textgreater} }

{\selectlanguage{english}\ttfamily\color{black}
\ \ \ \ \ \ \ \ \ \ \ \ {\textless}next
id={\textquotedbl}http{\textquotedbl}/{\textgreater} }

{\selectlanguage{english}\ttfamily\color{black}
\ \ \ \ \ \ \ \ \ \ \ \ {\textless}tls:KeyPath{\textgreater}./testkey-nopass.pem{\textless}/tls:KeyPath{\textgreater}
}

{\selectlanguage{english}\ttfamily\color{black}
\ \ \ \ \ \ \ \ \ \ \ \ {\textless}tls:CertificatePath{\textgreater}./testcert.pem{\textless}/tls:CertificatePath{\textgreater}
}

{\selectlanguage{english}\ttfamily\color{black}
\ \ \ \ \ \ \ \ \ \ \ \ {\textless}!-{}-tls:CACertificatePath{\textgreater}./cacert.pem{\textless}/tls:CACertificatePath-{}-{\textgreater}
}

{\selectlanguage{english}\ttfamily\color{black}
\ \ \ \ \ \ \ \ \ \ \ \ {\textless}tls:ClientAuthn{\textgreater}false{\textless}/tls:ClientAuthn{\textgreater}
}

{\selectlanguage{english}\ttfamily\color{black}
\ \ \ \ \ \ \ \ {\textless}/Component{\textgreater} }

{\selectlanguage{english}\ttfamily\color{black}
\ \ \ \ \ \ \ \ {\textless}Component
name={\textquotedbl}http.service{\textquotedbl}
id={\textquotedbl}http{\textquotedbl}{\textgreater} }

{\selectlanguage{english}\ttfamily\color{black}
\ \ \ \ \ \ \ \ \ \ \ \ {\textless}next
id={\textquotedbl}plexer{\textquotedbl}{\textgreater}POST{\textless}/next{\textgreater}
}

{\selectlanguage{english}\ttfamily\color{black}
\ \ \ \ \ \ \ \ {\textless}/Component{\textgreater} }

{\selectlanguage{english}\ttfamily\color{black}
\ \ \ \ \ \ \ \ {\textless}Plexer
name={\textquotedbl}plexer.service{\textquotedbl}
id={\textquotedbl}plexer{\textquotedbl}{\textgreater} }

{\selectlanguage{english}\ttfamily\color{black}
\ \ \ \ \ \ \ \ \ \ \ \ {\textless}next
id={\textquotedbl}samlsp{\textquotedbl}{\textgreater}/saml2sp{\textless}/next{\textgreater}
}

{\selectlanguage{english}\ttfamily\color{black}
\ \ \ \ \ \ \ \ \ \ \ \ {\textless}next
id={\textquotedbl}soap{\textquotedbl}{\textgreater}/slcs{\textless}/next{\textgreater}
}

{\selectlanguage{english}\ttfamily\color{black}
\ \ \ \ \ \ \ \ {\textless}/Plexer{\textgreater} }

{\selectlanguage{english}\ttfamily\color{black}
\ \ \ \ \ \ \ \ {\textless}Component
name={\textquotedbl}soap.service{\textquotedbl}
id={\textquotedbl}soap{\textquotedbl}{\textgreater} }

{\selectlanguage{english}\ttfamily\color{black}
\ \ \ \ \ \ \ \ \ \ \ \ {\textless}next
id={\textquotedbl}slcs{\textquotedbl}/{\textgreater} }

{\selectlanguage{english}\ttfamily\color{black}
\ \ \ \ {\textless}SecHandler
name={\textquotedbl}\textcolor[rgb]{0.8627451,0.13725491,0.0}{saml2ssoassertionconsumer.handler}{\textquotedbl}
id={\textquotedbl}saml2ssosp{\textquotedbl}
event={\textquotedbl}incoming{\textquotedbl}/{\textgreater}}

{\selectlanguage{english}\ttfamily\color{black}
\ \ \ \ \ \ \ \ \ {\textless}/Component{\textgreater} }

{\selectlanguage{english}\ttfamily\color{black}
\ \ \ \ \ \ \ \ {\textless}Service
name={\textquotesingle}\textcolor[rgb]{0.77254903,0.0,0.043137256}{saml.sp}{\textquotesingle}
id={\textquotesingle}samlsp{\textquotesingle}{\textgreater} }

{\selectlanguage{english}\ttfamily\color{black}
\ \ \ \ \ \ \ \ \ \ \ \ {\textless}MetaDataLocation{\textgreater}./test\_metadata.xml{\textless}/MetaDataLocation{\textgreater}
}

{\selectlanguage{english}\ttfamily\color{black}
\ \ \ \ \ \ \ \ \ \ \ \ {\textless}ServiceProviderName{\textgreater}https://squark.uio.no/shibboleth-sp{\textless}/ServiceProviderName{\textgreater}
}

{\selectlanguage{english}\ttfamily\color{black}
\ \ \ \ \ \ \ \ {\textless}/Service{\textgreater} }

{\selectlanguage{english}\ttfamily\color{black}
\ \ \ \ \ \ \ \ {\textless}Service
name={\textquotedbl}\textcolor[rgb]{0.8627451,0.13725491,0.0}{slcs.service}{\textquotedbl}
id={\textquotedbl}slcs{\textquotedbl}{\textgreater} }

{\selectlanguage{english}\ttfamily\color{black}
\ \ \ \ \ \ \ \ \ \ \ \ {\textless}next
id={\textquotedbl}slcs{\textquotedbl}/{\textgreater} }

{\selectlanguage{english}\ttfamily\color{black}
\ \ \ \ \ \ \ \ \ \ \ \ {\textless}slcs:CACertificate{\textgreater}./CAcert.pem{\textless}/slcs:CACertificate{\textgreater}
}

{\selectlanguage{english}\ttfamily\color{black}
\ \ \ \ \ \ \ \ \ \ \ \ {\textless}slcs:CAKey{\textgreater}./CAkey.pem{\textless}/slcs:CAKey{\textgreater}
}

{\selectlanguage{english}\ttfamily\color{black}
\ \ \ \ \ \ \ \ \ \ \ \ {\textless}slcs:CASerial{\textgreater}./CAserial{\textless}/slcs:CASerial{\textgreater}
}

{\selectlanguage{english}\ttfamily\color{black}
\ \ \ \ \ \ \ \ {\textless}/Service{\textgreater} }

{\selectlanguage{english}\ttfamily\color{black}
\ \ \ \ {\textless}/Chain{\textgreater} }

{\selectlanguage{english}\ttfamily\color{black}
{\textless}/ArcConfig{\textgreater} }

{\selectlanguage{english}\color{black}
\ \ \ \ SLCS service and SP service should together be configured by
using the Plexer. SP service is directly based on http and SLCS is
directly based on soap. }

{\selectlanguage{english}\color{black}
\ \ \ \ SLCS service should specifically include the CA credential
(certificate and key file) and the serial number file (which includes
the serial number of each signed certificate). Therefore, the service
administrator should firstly create a CA credential (note for
interoperability purpose in the production grid deployment, the CA
credential should be trusted by others). }

{\selectlanguage{english}\color{black}
\textbf{Note}: When deploying the SLCS service, service administrator
should deploy a dedicated IdP (Identity Provider) which can be assigned
as authentication/attribute authority, or the users should have already
had his own IdP. And the IdP information (authentication URL and
attribute authority URL) should have already been included into the
metadata of above configuration (e.g. test\_metadata.xml). On the other
hand, the SP (SP service) information (assertion consuming URL) should
also have been included into the metadata of the IdP.}

{\selectlanguage{english}\color{black}
After the whole process, user only needs to authentication with his
existing IdP and can get back the x509 credential.}

{\selectlanguage{english}\color{black}
Shibboleth IdP (http://shibboleth.internet2.edu) is used for the current
solution of IdP, since it is widely deployed for other AAI
(Authentication and Authority Infrastructure). There is a test IdP
deployed in
\href{https://squark.uio.no:8443/}{https://squark.uio.no:8443}
(idpname: https://squark.uio.no/idp/shibboleth), you can use the SP
from \url{https://sp.testshib.org/} to test the validity of the test
IdP. And also you can installed your own IdP which is supposed to
authenticates the users from your own organization.}

{\selectlanguage{english}\color{black}
b. \textbf{SLCS client}}

{\selectlanguage{english}\color{black}
On the SLCS client side, there is a client utility called
{\textquotedblleft}arcslcs{\textquotedblright}. \ The command option
for arcslcs is as following:}

{\selectlanguage{english}\color{black}
Application Options: }

{\selectlanguage{english}\color{black}
\ \ {}-S, -{}-url=url \ \ \ \ \ \ \ \ \ \ \ \ \ \ \ URL of SLCS service
}

{\selectlanguage{english}\color{black}
\ \ {}-I, -{}-idp=string \ \ \ \ \ \ \ \ \ \ \ \ IdP name }

{\selectlanguage{english}\color{black}
\ \ {}-U, -{}-user=string \ \ \ \ \ \ \ \ \ \ \ User account to IdP }

{\selectlanguage{english}\color{black}
\ \ {}-P, -{}-password=string \ \ \ \ \ \ \ Password for user account to
IdP }

{\selectlanguage{english}\color{black}
\ \ {}-Z, -{}-keysize=number \ \ \ \ \ \ \ \ Key size of the private key
(512, 1024, 2048) }

{\selectlanguage{english}\color{black}
\ \ {}-K, -{}-keypass=passphrase \ \ \ \ Private key passphrase }

{\selectlanguage{english}\color{black}
\ \ {}-L, -{}-lifetime=period \ \ \ \ \ \ \ Lifetime of the certificate,
start with current time, hour as unit }

{\selectlanguage{english}\color{black}
\ \ {}-D, -{}-storedir=directory \ \ \ \ Store directory for key and
signed certificate}

{\selectlanguage{english}\color{black}
\ \ {}-c, -{}-conffile=filename \ \ \ \ \ configuration file (default
\~{}/.arc/client.xml) }

{\selectlanguage{english}\color{black}
An example is:}

{\selectlanguage{english}\color{black}
./arcslcs -S https://127.0.0.1:60000/slcs -I
https://squark.uio.no/idp/shibboleth -U root -P aa1122 -D
/home/wzqiang/arc-0.9/src/clients/credentials -c client.xml }

{\selectlanguage{english}\color{black}
\ \ \ \ Note user should input the {\textquotedblleft}IdP
name{\textquotedblright} is the corresponding Identity Provider (one of
the participants of SAML2 SSO profile) name to which the user would
authenticate against by using its username/password credential. And the
name is stored inside the metadata on both SP service and IdP provider,
and it is used by SP service to get the authentication URL for
username/password based authentication.}

{\selectlanguage{english}\color{black}
\ \ \ \ And the {\textquotedblleft}user{\textquotedblright} and
{\textquotedblleft}password{\textquotedblright} is the credential which
will be used to authentication against IdP.}

{\selectlanguage{english}\color{black}
The {\textquotedblleft}conffile{\textquotedblright} can be another
option for all of the above options, see the following as an example:}

{\selectlanguage{english}\ttfamily\color{black}
{\textless}ArcConfig{\textgreater} }

{\selectlanguage{english}\ttfamily\color{black}
\ {\textless}!-{}- change the paths below to the location of your user
certs -{}-{\textgreater} }

{\selectlanguage{english}\ttfamily\color{black}
\ {\textless}!-{}- here only trusted certificates are set. Because this
arcslcs is }

{\selectlanguage{english}\ttfamily\color{black}
\ not supposed to have x509 credential before the samlsso profile has
been }

{\selectlanguage{english}\ttfamily\color{black}
\ (by using client{\textquotesingle}s username/password credential)
passed and the SLCS }

{\selectlanguage{english}\ttfamily\color{black}
\ service succeeded to repond this client with the signed x509
credential. }

{\selectlanguage{english}\ttfamily\color{black}
\ On the SLCS service{\textquotesingle}s configuration: ClientAuthn
should be off.-{}-{\textgreater} }

{\selectlanguage{english}\ttfamily\color{black}
\ {\textless}!-{}-KeyPath{\textgreater}./testkey-nopass.pem{\textless}/KeyPath-{}-{\textgreater}
}

{\selectlanguage{english}\ttfamily\color{black}
\ {\textless}!-{}-CertificatePath{\textgreater}./testcert.pem{\textless}/CertificatePath-{}-{\textgreater}
}

{\selectlanguage{english}\ttfamily\color{black}
\ {\textless}CACertificatesDir{\textgreater}./certificates{\textless}/CACertificatesDir{\textgreater}
}

{\selectlanguage{english}\ttfamily\color{black}
\ {\textless}SLCSURL{\textgreater}https://127.0.0.1:60000/slcs{\textless}/SLCSURL{\textgreater}
}

{\selectlanguage{english}\ttfamily\color{black}
\ {\textless}IdPname{\textgreater}https://squark.uio.no/idp/shibboleth{\textless}/IdPname{\textgreater}
}

{\selectlanguage{english}\ttfamily\color{black}
\ {\textless}Username{\textgreater}root{\textless}/Username{\textgreater}
}

{\selectlanguage{english}\ttfamily\color{black}
\ {\textless}Password{\textgreater}aa1122{\textless}/Password{\textgreater}
}

{\selectlanguage{english}\ttfamily\color{black}
\ {\textless}Keysize{\textgreater}1024{\textless}/Keysize{\textgreater}
}

{\selectlanguage{english}\ttfamily\color{black}
\ {\textless}Keypass{\textgreater}123456{\textless}/Keypass{\textgreater}
}

{\selectlanguage{english}\ttfamily\color{black}
\ {\textless}CertLifetime{\textgreater}24{\textless}/CertLifetime{\textgreater}
}

{\selectlanguage{english}\ttfamily\color{black}
\ {\textless}StoreDir{\textgreater}./{\textless}/StoreDir{\textgreater}
}

{\selectlanguage{english}\ttfamily\color{black}
\ {\textless}Debug{\textgreater}INFO{\textless}/Debug{\textgreater} }

{\selectlanguage{english}\ttfamily\color{black}
{\textless}/ArcConfig{\textgreater} }

{\selectlanguage{english}\color{black}
\ \ \ \ There is a temporary Identity Provider (IdP) deployed on
\textit{squark.uio.no} for test with the following test username and
password: staff, researcher, librarian, binduser; with the same
password {\textquotedbl}123456{\textquotedbl}}

{\selectlanguage{english}\color{black}
./arcslcs -S https://127.0.0.1:60000/slcs -I
https://squark.uio.no/idp/shibboleth -U staff -P 123456 -D
/home/wzqiang/arc-0.9/src/clients/credentials -c client.xml}

{\selectlanguage{english}\upshape\color{black}
\ \ \ \ The short-lived credential issued by SLCS service will include
the SAML assertion as the extension of X.509 certificate as a proof of
passing SAML2 SSO profile.}
\end{document}
