\chapter{The echo service}

The next service to be presented is the echo service which returns the received message back to the client.
Furthermore the client has to choose between the operations ``ordinary'' or ``reverse''.
Depending on that operation the message will be returned directly or its letters will be reordered reverse. % without manipulation
The goal of this chapter is to offer a deeper knowledge about message processing. Furthermore the \textit{WSDL} (Web Services Description Language) shall be introduced along with this example.

\section{Web Services Description Language}

Web Services are often reachable for a large set of users in order to easily access complex information. In many cases the Web Service are capable to replace HTML pages i.e. to access to databases like Pubmed or the NLM catalog.
% http://www.ncbi.nlm.nih.gov/entrez/query/static/esoap_help.html
In contrary to HTML which describes a human readable document the interfaces of Web Services are well defined and easier to interpret for machines.
In general the client is not familiar with the functionality of the service.
Due to that reason, the access to a web service can be defined in a \textit{WSDL} (Web Services Description Language) file.
WSDL is a meta language which describes the structure of the messages, the intended use of the message, the supported protocols and the endpoint of the service (URL).
The WSDL file which specifies the echo service of this chapter is to be seen in Listing~\ref{lst:echo_wsdl}. 
It consisits of five main WSDL elements:
\newcommand{\parboxWidth}{13cm}
\begin{itemize}
	\item \textbf{Types} --- The element \textit{types} contains the description of the message structures which is written in the language XSD (XML Schema Definition). The applicable data types of the echo service are defined in  line~\ref{lst_code:echo_wsdl_types}.

	\item \textbf{Message} --- Within the element \textit{message} a subset of the previously defined types are assigned to be a message. In case of the echo service two messages are designated: echoRequest and echo Response, see lines line~\ref{lst_code:echo_wsdl_message1} and \ref{lst_code:echo_wsdl_message2}.

	\item \textbf{Port type} --- The element \textit{portType} is used to assign the messages to an interface. Four interface types can be distinguished: One-way (input), request-response (input, output), solicit-response (input, output, fault) and notification (output).
	The echo service realises a simple request-response interface which is declared subsequent line~\ref{lst_code:echo_wsdl_portType}. % in the lines followed by the line X

	\item \textbf{Binding} --- The protocol and the data format used for the message transmission are denoted in the element \textit{binding}. Possible style attributes are \textit{document} or \textit{rpc} (Remote Procedure Call). The transport attribute defines the protocol which is regularly HTTP. For each operation a corresponding SOAP action has to be defined. Additionally the input and output has to be specified to be either \textit{literal} or \textit{encoded}~\cite{BUTEK_2009}. The service implemented in this chapter will use the binding \textit{document}/\textit{literal} and the protocol HTTP which is to be seen in line~\ref{lst_code:echo_wsdl_binding}.

% The style attribute can be "rpc" or "document". In this case we use document. The transport attribute defines the SOAP protocol to use. In this case we use HTTP.
%The operation element defines each operation that the port exposes.
%For each operation the corresponding SOAP action has to be defined. You must also specify how the input and output are encoded. In this case we use "literal".
%  http://www.ibm.com/developerworks/webservices/library/ws-whichwsdl/
%   1. RPC/encoded    (Remote Procedure Call)
%   2. RPC/literal
%   3. Document/encoded
%   4. Document/literal
% The terminology here is very unfortunate: RPC versus document. These terms imply that the RPC style should be used for RPC programming models and that the document style should be used for document or messaging programming models. That is not the case at all. The style has nothing to do with a programming model. It merely dictates how to translate a WSDL binding to a SOAP message. Nothing more. You can use either style with any programming model.
%
% RPC/literal SOAP message for myMethod
%
%<soap:envelope>
%    <soap:body>
%        <myMethod>
%            <x>5</x>
%            <y>5.0</y>
%        </myMethod>
%    </soap:body>
%</soap:envelope>
%
%  RPC/encoded SOAP message for myMethod
% 
% <soap:envelope>
%     <soap:body>
%         <myMethod>
%             <x xsi:type="xsd:int">5</x>
%             <y xsi:type="xsd:float">5.0</y>
%         </myMethod>
%     </soap:body>
% </soap:envelope>

	\item \textbf{Service} --- Within the last WSDL element the endpoint of the service is specified. As to be seen in line~\ref{lst_code:echo_wsdl_service} the endpoint of the echo service is assigned to \textit{http://localhost:60000/echo}.

\end{itemize}
%\textcolor{white}{newline}
\lstsetJUSTXML
\lstinputlisting
	[
	label=lst:echo_wsdl,float=p,
	caption={[WSDL file describing the echo service. Filename: echo.wsdl]
	\textbf{WSDL file describing the echo service. Filename: echo.wsdl}}
	]
{../src/services/echoservice/echo.wsdl}


More information about WSDL may be found at \href{http://www.w3.org/TR/wsdl}{http://www.w3.org/TR/wsdl}.

% a valid service request and the  


% WSDL ist eine Metasprache, mit deren Hilfe die angebotenen Funktionen, Daten, Datentypen und Austauschprotokolle eines Web Service beschrieben werden können. Es werden im Wesentlichen die Operationen definiert, die von außen zugänglich sind, sowie die Parameter und Rückgabewerte dieser Operationen. Im Einzelnen beinhaltet ein WSDL-Dokument funktionelle Angaben zu:

%    * der Schnittstelle
%    * Zugangsprotokoll und Details zum Deployment
%    * Alle notwendigen Informationen zum Zugriff auf den Service, in maschinenlesbarem Format

%Nicht enthalten sind hingegen:

%    * Quality-of-Service-Informationen
%    * Taxonomien/Ontologien zur semantischen Einordnung des Services
\clearpage

\section{Service}

The implementation of the service defined in the WSDL file of the previous section will blub. The source code of the C++ file is shown in Listing~\ref{lst:echo_service_cpp}. Again the header file will be set aside for it contains to much redundant information.

\lstsetCPP
\lstinputlisting
	[
	label=lst:echo_service_cpp,
	caption={[C++ implementation of the echo service. Filename: echoservice.cpp]
	\textbf{C++ implementation of the echo service. Filename: echoservice.cpp}}
	]
{../src/services/echoservice/echoservice.cpp}


\lstsetARCHEDXML

\lstinputlisting
	[
	label=lst:arcecho_arched_xml, float=htb,
	caption={[HED configuration file for the Arc intern echo service. Filename: arcecho\_no\_ssl.xml]
	\textbf{HED configuration file for the Arc intern echo service. Filename: arcecho\_no\_ssl.xml\textcolor{white}{hmf}}}
	]
{../src/services/echoservice/arched_echoservice.xml}





\lstsetKSH
\begin{lstlisting}[
label=lst:invokation_arched_timeservice,float=htb,
caption={[Transformation in eine uniforme konzentrische Verteilung.]
         \textbf{Transformation in eine uniforme konzentrische Verteilung.\textcolor{white}{hmf}}}]
$ arched -c arched_echoservice.xml  && echo jo ||echo n
\end{lstlisting}



\section{Client}


\lstsetCPP
\lstinputlisting
	[
	label=lst:arcecho_arched_xml,
	caption={[HED configuration file for the Arc intern echo service. Filename: arcecho\_no\_ssl.xml]
	\textbf{HED configuration file for the Arc intern echo service. Filename: arcecho\_no\_ssl.xml\textcolor{white}{hmf}}}
	]
{../src/clients/echoclient/echoclient.cpp}

\lstsetCPP




\lstsetKSH
\begin{lstlisting}[
label=lst:invokation_arched_timeservice, float=htb,
caption={[Transformation in eine uniforme konzentrische Verteilung.]
         \textbf{Transformation in eine uniforme konzentrische Verteilung.\textcolor{white}{hmf}}}]
$ ./echoclient http://localhost:60000/echo ordinary text_to_be_transmitted
[ text_to_be_transmitted ]
$ ./echoclient http://localhost:60000/echo reverse text_to_be_transmitted
[ dettimsnart_eb_ot_txet ]
\end{lstlisting}





\lstsetJUSTXML
\begin{lstlisting}[
label=lst:timeservice_cpp_source, float=htb,
caption={[Transformation in eine uniforme konzentrische Verteilung.]
         \textbf{Transformation in eine uniforme konzentrische Verteilung.\textcolor{white}{hmf}}}]
<soap-env:Envelope xmlns:echo="urn:echo" xmlns:soap-enc="http://schemas.xmlsoap.org/soap/encoding/" xmlns:soap-env="http://schemas.xmlsoap.org/soap/envelope/" xmlns:xsd="http://www.w3.org/2001/XMLSchema" xmlns:xsi="http://www.w3.org/2001/XMLSchema-instance">
  <soap-env:Body>
    <echo:echoRequest>
      <echo:say operation="reverse">text_to_be_transmitted</echo:say>
    </echo:echoRequest>
  </soap-env:Body>
</soap-env:Envelope>
\end{lstlisting}



\lstsetJUSTXML
\begin{lstlisting}[
label=lst:timeservice_cpp_source, float=htb,
caption={[Transformation in eine uniforme konzentrische Verteilung.]
         \textbf{Transformation in eine uniforme konzentrische Verteilung.\textcolor{white}{hmf}}}]
<soap-env:Envelope xmlns:echo="urn:echo" xmlns:soap-enc="http://schemas.xmlsoap.org/soap/encoding/" xmlns:soap-env="http://schemas.xmlsoap.org/soap/envelope/" xmlns:xsd="http://www.w3.org/2001/XMLSchema" xmlns:xsi="http://www.w3.org/2001/XMLSchema-instance">
  <soap-env:Body>
    <echo:echoResponse>
      <echo:hear>[ dettimsnart_eb_ot_txet ]</echo:hear>
    </echo:echoResponse>
  </soap-env:Body>
</soap-env:Envelope>
\end{lstlisting}









Folgender XML Aufruf und Antwort soll \textit{automatisch} generiert werden:

%[caption={[]WegDamit},language=XML,basicstyle=\scriptsize,breaklines=true,label=lst:request] 





Zertifikate
Zustände (Nutzer wiedererkennen, arbeit aufnehmen)
