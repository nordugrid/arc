\documentclass{book}
\usepackage{graphicx}                              %for PNG images (pdflatex)
\usepackage[linkbordercolor={1.0 1.0 0.0}]{hyperref} %for \url tag
\usepackage{color}                                 %for defining custom colors
\usepackage{framed}                                %for shaded and framed paragraphs
\usepackage{textcomp}                              %for various symbols, e.g. Registered Mark
\usepackage{geometry}                              %for defining page size
\usepackage{longtable}                             %for breaking tables
%
\geometry{verbose,a4paper,tmargin=2.5cm,bmargin=2.5cm,lmargin=2.5cm,rmargin=2cm}
\hypersetup{
  pdfauthor = {Zsombor Nagy},
  pdftitle = {User's manual of the ARC storage system},
  pdfsubject = {Paper subject},
  pdfkeywords = {Paper,keyword,comma-separated},
  pdfcreator = {PDFLaTeX with hyperref package},
  pdfproducer = {PDFLaTeX}
}
%
\bibliographystyle{IEEEtran}                       %a nice bibliography style
%
\def\efill{\hfill\nopagebreak}%
\hyphenation{Nordu-Grid}
\setlength{\parindent}{0cm}
\setlength{\FrameRule}{1pt}
\setlength{\FrameSep}{8pt}
\addtolength{\parskip}{5pt}
\renewcommand{\thefootnote}{\fnsymbol{footnote}}
\renewcommand{\arraystretch}{1.3}
\newcommand{\dothis}{\colorbox{shadecolor}}
\newcommand{\ngdl}{\url{http://ftp.nordugrid.org/download}~}
\definecolor{shadecolor}{rgb}{1,1,0.6}
\definecolor{salmon}{rgb}{1,0.9,1}
\definecolor{bordeaux}{rgb}{0.75,0.,0.}
\definecolor{cyan}{rgb}{0,1,1}
%
%----- DON'T CHANGE HEADER MATTER
\hyphenation{preserve-Original}
\begin{document}
\def\today{\number\day/\number\month/\number\year}

\begin{titlepage}

\begin{tabular}{rl}
\resizebox*{3cm}{!}{\includegraphics{ng-logo.png}}
&\parbox[b]{2cm}{\textbf \it {\hspace*{-1.5cm}NORDUGRID\vspace*{0.5cm}}}
\end{tabular}

\hrulefill

%-------- Change this to NORDUGRID-XXXXXXX-NN

{\raggedleft NORDUGRID-XXXXXXX-NN\par}

{\raggedleft \today\par}

\vspace*{2cm}

%%%%---- The title ----
{\centering \textsc{\Large Administrator's and user's manual of the ARC storage system}\Large \par}
\vspace*{0.5cm}
    
%%%%---- A subtitle, if necessary ----
%{\centering \textit{\large First prototype status and plans}\large \par}
    
\vspace*{1.5cm}
%%%%---- A list of authors ----
    {\centering \large Zsombor Nagy\footnote{zsombor@niif.hu} \large \par}
    {\centering \large Jon Nilsen\footnote{j.k.nilsen@usit.uio.no} \large \par}
    {\centering \large Salman Zubair Toor \footnote{salman.toor@it.uu.se} \large \par}
\end{titlepage}

\tableofcontents                          %Comment if use article style
\begin{verbatim}
    Administrator's and user's manual
    ---------------------------------

    Administrator's manual
    - Quick start guide
        - Checking the code out from the subversion
        - Compiling and installing
        - Modifying the configuration template
        - Running the daemon
        - Using the CLI tool
    - Installation:
        - Getting the source code from subversion
        - Compilation details
        - Installation
    - Configuration:
        - HED service config
        - Centralized A-Hash
        - Distributed A-Hash
        - Librarian
        - Shepherd
        - Hopi
        - Bartender
        - Configuring inter-service trust: local listing of DNs, or using an A-Hash to store them
        - Virtual organizations: how to configure HED to trust a VO
    - Deployment scenarios:
        - Single node: putting every service on one single node (including a storage node with Hopi)
        - Centralized management: using seperate multiple storage nodes with Hopi, but leaving a central A-Hash, a Librarian and a Bartender on the central node
        - Simple distribution: split the central node to three nodes, each of them has an A-Hash, a Librarian and a Bartender, and we have separate storage nodes
        - One service per node: we have seperate nodes for all the services, any number of A-Hash nodes, any number of Librarian nodes, Bartender nodes, storage nodes

    User's manual
    - Overview: what do you need to access a running ARC storage system
    - Prototype command line tool
        - Configuring the command line tool
        - Creating the root collection
        - Uploading files
        - Creating collections
        - Moving files
        - Creating hardlinks
        - Removing files
        - Removing collections
        - Close collections
        - Modify access policies
        - Working with mount points
    - Grid jobs
        - Configuring the ARC storage DMC
        - Specifying files in job descriptions
    - Accessing the system with FUSE
        - Configuring the FUSE module
        - Mounting the storage system
        - Supported operations
        - Limitations
    
\end{verbatim}

\newpage

\renewcommand{\thefootnote}{\arabic{footnote}}


\chapter{Administrator's manual} % (fold)
\label{cha:administrator_s_manual}

\section{Quick start guide} % (fold)
\label{sec:quick_start_guide}

\begin{verbatim}
http://svn.nordugrid.org/
svn co http://svn.nordugrid.org/repos/nordugrid/arc1/trunk arc1   
README file for dependencies
$ ./autogen.sh
$ ./configure --disable-java --disable-a-rex-service --disable-isi-service --disable-charon-service --disable-compiler-service --disable-paul-service --disable-sched-service
--prefix

Unit testing:       yes
Java binding:       no
Python binding:     yes (2.5)

Available third-party features:

RLS:                yes
GridFTP:            yes
LFC:                no
RSL:                yes
SAML:               yes
MYSQL CLIENT LIB:   no
gSOAP:              no

Included components:
A-Rex service:      no
ISI service:        no
CHARON service:     no
HOPI service:       yes
SCHED service:      no
STORAGE service:    yes
PAUL service:       no
SRM client (DMC):   no
GSI channel (MCC):  yes

$ make

$ sudo make install


\end{verbatim}


% section quick_start_guide (end)

\section{Installation} % (fold)
\label{sec:installation}

% section installation (end)

\section{Configuration} % (fold)
\label{sec:configuration}

% section configuration (end)

\section{Deployment scenarios} % (fold)
\label{sec:deployment_scenarios}

% section deployment_scenarios (end)

% chapter administrator_s_manual (end)

\chapter{User's manual} % (fold)
\label{cha:user_s_manual}

\section{Overview} % (fold)
\label{sec:overview}

% section overview (end)

\section{Prototype command line tool} % (fold)
\label{sec:prototype_command_line_tool}

% section prototype_command_line_tool (end)

\section{Grid jobs} % (fold)
\label{sec:grid_jobs}

% section grid_jobs (end)

\section{Accessing the system with FUSE} % (fold)
\label{sec:accessing_the_system_with_fuse}

% section accessing_the_system_with_fuse (end)

% chapter user_s_manual (end)

% \bibliography{grid}
\end{document}
