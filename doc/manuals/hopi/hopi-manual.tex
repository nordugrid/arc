\documentclass{article}
%\documentclass{book}
\usepackage{graphicx}                              %for PNG images (pdflatex)
\usepackage[linkbordercolor={1.0 1.0 0.0}]{hyperref} %for \url tag
\usepackage{color}                                 %for defining custom colors
\usepackage{framed}                                %for shaded and framed paragraphs
\usepackage{textcomp}                              %for various symbols, e.g. Registered Mark
\usepackage{geometry}                              %for defining page size
\usepackage{longtable}                             %for breaking tables
%
\geometry{verbose,a4paper,tmargin=2.5cm,bmargin=2.5cm,lmargin=2.5cm,rmargin=2cm}
\hypersetup{
  pdfauthor = {Zsombor Nagy},
  pdftitle = {The Hopi service},
  pdfsubject = {Paper subject},
  pdfkeywords = {Paper,keyword,comma-separated},
  pdfcreator = {PDFLaTeX with hyperref package},
  pdfproducer = {PDFLaTeX}
}
%
\bibliographystyle{IEEEtran}                       %a nice bibliography style
%
\def\efill{\hfill\nopagebreak}%
\hyphenation{Nordu-Grid}
\setlength{\parindent}{0cm}
\setlength{\FrameRule}{1pt}
\setlength{\FrameSep}{8pt}
\addtolength{\parskip}{5pt}
\renewcommand{\thefootnote}{\fnsymbol{footnote}}
\renewcommand{\arraystretch}{1.3}
\newcommand{\dothis}{\colorbox{shadecolor}}
\newcommand{\ngdl}{\url{http://ftp.nordugrid.org/download}~}
\definecolor{shadecolor}{rgb}{1,1,0.6}
\definecolor{salmon}{rgb}{1,0.9,1}
\definecolor{bordeaux}{rgb}{0.75,0.,0.}
\definecolor{cyan}{rgb}{0,1,1}
%
%----- DON'T CHANGE HEADER MATTER
\hyphenation{preserve-Original}
\begin{document}
\def\today{\number\day/\number\month/\number\year}

\begin{titlepage}

\begin{tabular}{rl}
\resizebox*{3cm}{!}{\includegraphics{ng-logo.png}}
&\parbox[b]{2cm}{\textbf \it {\hspace*{-1.5cm}NORDUGRID\vspace*{0.5cm}}}
\end{tabular}

\hrulefill

%-------- Change this to NORDUGRID-XXXXXXX-NN

{\raggedleft NORDUGRID-MANUAL-15\par}

{\raggedleft \today\par}

\vspace*{2cm}

%%%%---- The title ----
{\centering \textsc{\Large The Hopi manual}\Large \par}
\vspace*{0.5cm}
    
%%%%---- A subtitle, if necessary ----
%{\centering \textit{\large First prototype status and plans}\large \par}
    
\vspace*{1.5cm}
%%%%---- A list of authors ----
    {\centering \large Zsombor Nagy\footnote{zsombor@niif.hu} \large \par}
\end{titlepage}

\tableofcontents                          %Comment if use article style

\newpage

\renewcommand{\thefootnote}{\arabic{footnote}}


%\chapter{Quick start guide} % (fold)

\section{Introduction} % (fold)
\label{sec:introduction}
The Hopi service is a HTTP(S) server supporting GET and PUT methods and capable of generating a HTML listing of files. It's main configuration parameter is the document root directory.

It also has a slave mode which is used in the Chelonia storage system tied to the Shepherd service. In the slave mode: 1. there is no HTML listing generated; 2. the files are removed after download; 3. upload is only possible to an existing file, which will be removed after the upload. (The Shepherd service keeps a second hardlink of each file in a separate directory, so that's why it makes sense to remove files after transfer.)

The Hopi service can be the only service in a message chain, in that case it does not need to have a Plexer, and the URLs will not contain any extra path; or the Hopi could be after the Plexer, in that case all the URLs contains a fixed path (e.g. \verb!https://localhost:60000/hopi/!).
% section introduction (end)

\section{Clients} % (fold)
\label{sec:clients}
Because Hopi is a HTTP(S) server, any HTTP client can be used (for HTTPS maybe needs more configuration)
\begin{itemize}
	\item web browser (needs importing the client credentials into the browser)
	\item curl/wget (needs specifying the client credentials as command line argument)
	\item arccp (the client credentials comes from the userconfig)
\end{itemize}
% section clients (end)

\section{Configuration} % (fold)
\label{sec:configuration}
Here is an example configuration of a secure Hopi with Plexer:

\begin{verbatim}
<?xml version="1.0"?>
<cfg:ArcConfig xmlns="http://www.nordugrid.org/schemas/loader/2009/08"
	xmlns:cfg="http://www.nordugrid.org/schemas/arcconfig/2009/08"
	xmlns:tcp="http://www.nordugrid.org/schemas/tcp/2009/08"
	xmlns:tls="http://www.nordugrid.org/schemas/tls/2009/08">
    <cfg:Server>
        <cfg:PidFile>tmp/arched.pid</cfg:PidFile>
        <cfg:Logger>
            <cfg:File>/var/log/arched.log</cfg:File>
            <cfg:Level>ERROR</cfg:Level>
        </cfg:Logger>
    </cfg:Server>
    <ModuleManager>
        <Path>/usr/local/lib/arc/</Path>
    </ModuleManager>
    <Plugins>
        <Name>mcctls</Name>
        <Name>mcchttp</Name>
        <Name>mccsoap</Name>
        <Name>mcctcp</Name>
    </Plugins>
    <Chain>
        <Component name="tcp.service" id="tcp">
            <next id="tls"/>
            <tcp:Listen>
                <tcp:Interface>0.0.0.0</tcp:Interface>
                <tcp:Port>50000</tcp:Port>
                <tcp:Version>4</tcp:Version>
            </tcp:Listen>
        </Component>
        <Component name="tls.service" id="tls">
            <next id="http"/>
            <tls:KeyPath>/etc/grid-security/hostkey.pem</tls:KeyPath>
            <tls:CertificatePath>/etc/grid-security/hostcert.pem</tls:CertificatePath>
            <tls:CACertificatesDir>/etc/grid-security/certificates</tls:CACertificatesDir>
        </Component>
        <Component name="http.service" id="http">
            <next id="soap">POST</next>
            <next id="plexer">GET</next>
            <next id="plexer">PUT</next>
        </Component>
        <Component name="soap.service" id="soap">
            <next id="plexer"/>
        </Component>
        <Plexer name="plexer.service" id="plexer">
            <next id="hopi">^/hopi/</next>
        </Plexer>
        <Service name="hopi" id="hopi">
            <DocumentRoot>/tmp/hopi</DocumentRoot>
            <SlaveMode>0</SlaveMode>
        </Service>
    </Chain>
</cfg:ArcConfig>
\end{verbatim}
% section configuration (end)


\end{document}
