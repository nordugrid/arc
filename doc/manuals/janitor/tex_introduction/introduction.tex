\chapter{Introduction}

The Janitor is a service for an automated installation of runtime
environments (RTEs) for grid computing elements.  It is addressed
transparently via the A-REX service for job submission but can also be
used as a standalone tool.

\section{Motivation}

A major motivation for grid projects is to stimulate new communities
to adopt the technology to start sharing their resources. From the
current grid user's viewpoint, the admission of users of a very different
education will suddenly impose difficulties in the communication between
site maintainers. One will not even understand the respective other side's
research aims. Hence, the proper installation of non-standard software
(runtime environments, RTEs) is not guaranteed. And the available time for
manual labour plus self-education is scarse.

A core problem remains to distribute a locally working solution, the
Know-How, quickly across all contributing sites, i.e., without manual
interference. Every scientific discipline has its respective own set
of technologies for the distribution of work load. E.g. research in
bioinformatics requires access to so many different tools and databases,
that few sites, if any, install them all. Instead, the use of web
services became a commodity, with all their intrinsic problems as there
are bottlenecks and restrictions of repeated access.  The EU project
KnowARC\footnote{\href{http://www.knowarc.eu}{http://www.knowarc.eu}}
amongst other challenges with the here presented work
extends the NorduGrid's Advanced Research Connector (ARC) grid
middleware~\cite{ELLERT_2007} towards an infrastructure for the automated
installation of software packages.

An automation of the software installation, referred to as dynamic
Runtime Environments (dRTEs), seems the only approach to use the
computational grid to its full potential. Components of workflows
shall be spawned as jobs in a computational grid using dRTEs rather
than accessing a web service at one particular machine that is shared
amongst all users.  The grid introduces an extra level of parallelism
that web services cannot provide. The demands for short response times
and the heterogeneous education of site-administrators on a grid demand
an automatism for the installation of software and databases without
manual interference~\cite{BAYER_2007}.

\section{Overview}

This document first dedicates a chapter on how to set-up the Janitor
locally. It is followed by a chapter that gives further instructions on
how to use the Janitor with A-REX and/or without A-REX. Afterwards, in the
third chapter, the maintenance of the program will be presented, which is
basically covering the method how to prepare new dRTEs. Deeper insights on
the design of the Janitor will be given by the forth chapter. The document
ends with an outlook to anticipated future developments and opportunities.

\section*{Abbreviations}

\begin{tabular}{l@{--}p{150mm}}
RTE&Runtime Environment\\
dRTE&dynamic Runtime Environment\\
RDF&Resource Description Framework (supporting the RTE Catalog)\\
\end{tabular}


% BASIC CONCEPT

% * Installation

% * Usage

% * Maintainance

% * Programming concept
 
% * Future work

% * APPENDIX
