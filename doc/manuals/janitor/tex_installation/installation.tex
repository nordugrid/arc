\chapter{Installation} 


% ** Dependencies
% *** Ordianary: log4perl and ``Redland RDF Library - Perl Interface''
% *** WebService: libperl-dev
% *** Protege: For maintaining the knowledge base


In order to use Janitor, the two perl packages listed in table~\ref{tab:install_dependencies} are required. To have the WebService interface for Janitor, the packages listed in the table~\ref{tab:install_dependencies_optional} 
have to be installed before the build process is.
\begin{table}[!h]
   \begin{center}
        \mycaption{Required perl packages for Janitor.}{Log4perl is used for the internal logging of Janitor, while the 
                   Redland RDF libraray is used for accessing the knowledge base of RTEs.}
        \label{tab:install_dependencies}
	\begin{tabular}{|p{3cm}|p{7cm}|}
	\hline
	   liblog-log4perl-perl & \textit{Log4perl is a port of the log4j logging package}\\
	\hline
	   librdf-perl          & \textit{Perl language bindings for the Redland RDF library}\\
	\hline
	\end{tabular} 
   \end{center}
\end{table}
\begin{table}[!h]
   \begin{center}
        \mycaption{Optional libraries for Janitor.}{The library libperl-dev provides header files which are needed to link the 
                   WebService to the Perl interpreter.}
        \label{tab:install_dependencies_optional}
	\begin{tabular}{|p{3cm}|p{7cm}|}
	\hline
	   libperl-dev & \textit{ Perl library: development files}\\
	\hline
	\end{tabular}
   \end{center}
\end{table}
\forcelinebreak

Janitor will be installed by default along with ARC1. If desired it is possible to disable the shipping of Janitor 
using the \textit{configure} flags \textit{--disable-janitor-service} for the complete janitor or 
\textit{--disable-janitor-webservice} for only the Web Service support.
Furthermore it is recommended to install the ontology editor 
Prot\`eg\`e\footnote{\href{http://protege.stanford.edu}{http://protege.stanford.edu}} in order to be to easily maintain the 
knowledge database of installable packages.







\section{Configuration}\label{sec:janitor_configuration}

The current version of Janitor can be configured using the common file \textit{arc.conf} which is to be found in the 
configuration directory \textit{etc}. Janitor is using the environment variable \textit{NORDUGRID\_CONFIG} to determine the location 
of the corresponding file. If the variable is not set, the default location \textit{/etc/arc.conf} will be used. 
The configuration is assigned by the section \lbrack janitor\rbrack. The table~\ref{tab:arc_conf_tags} contains the available tags
for janitor configuration.
\begin{landscape}
\begin{table}[!h]
   \begin{center}
        \mycaption{Tags usable in \textit{arc.conf} within the section janitor.}{Tags usable in \textit{arc.conf} within the section janitor.}
	\label{tab:arc_conf_tags}
	\begin{longtable}{|p{3cm}|p{10cm}||p{10cm}|}
	\hline
	   \textbf{tag}    & \textbf{example}                      & \textbf{description}\\
        \hline
           enabled         & "1"                                   & Boolean flag which enables or disables janitor in A-REX.\\
	   uid             & "root"                                & The effective uid. \\
	   gid             & "0"                                   & The effective gid. \\
	   registrationdir & "/var/spool/nordugrid/janitor"        & Directory where we the current states of jobs are kept. \\
	   catalog         & "/var/spool/nordugrid/janitor/catalog/knowarc.rdf"& URL of the catalog containg the package information.\\
	   downloaddir     & "/var/spool/nordugrid/janitor/download" & Directory for downloads \\
	   installationdir & "/var/spool/nordugrid/janitor/runtime"& Directory for installation of packages                   \\
	   jobexpirytime   & "7200"                                & If a job is older than this, it is considered dead and assigned to be removal pending.\\
	   rteexpirytime   & "36"                                  & If a runtime environment was not used for this time, it will be assigned to be removal pending.\\
	   allow\_base     & "*"                                   & Allow rule for base packages. \\
	   deny\_base      & "debian::etch"                        & Deny rule for base packages.\\
	   allow\_rte      & "*"                                   & Allow rule for base packages. \\
	   deny\_rte       & "APPS/MATH/ELMER-5.0.2"               & Deny rule for base packages. \\
	   logconf         & "/opt/nordugrid/etc/log.conf"         & Location of the logging configuration file for janitor.\\
	\hline
	\end{longtable}
   \end{center}
\end{table}
\end{landscape}

The parameter \texttt{enable} defines whether Janitor shall be used within A-REX or not. Use the value \texttt{"0"} to disable Janitor. The
\texttt{uid} and the \texttt{gid} are defining which effective the user and group id shall be used for Janitor. 
The \texttt{registrationdir} describes the directory in which the subdirectories \texttt{jobs} and \texttt{rtes} will be created.
In these directories the states of the jobs and the runtime environments will be stored. 
The knowledge base of installable packages is specified by the parmeter \texttt{catalog}. 
Its value can be any kind of URL pointing to an file written in the Resource Description Framework (RDF) format.
The specification of the RDF file will be explained in detail in section~\ref{sec:catalog}.
The parameter \texttt{downloaddir} assignes the directory in which the installation files will be saved after they have been 
downloaded or copied from the repository which was specified by the catalog. The \texttt{installationdir} finally specifies the
directory into which all packages will be installed. Instead of the other directories the \texttt{installationdir} should be 
available for all computing elements i.e. by using a shared volume.
If the configuration file furthermore contains the \texttt{runtimedir} tag within the section \texttt{grid-manager}, Janitor will also
create a symbolic link in the \texttt{runtimedir} pointing to the configuration script of the installation done by Janitor.
The tags \texttt{jobexpirytime} and \texttt{rteexpirytime} are used for automated cleanup and is defined in seconds.
The default value for the \texttt{jobexpirytime} is seven days and for the \texttt{rteexpirytime} three days. 
The additional tags \texttt{allow\_base} \texttt{deny\_base} \texttt{allow\_rte} and \texttt{deny\_rte} are used to include
or exclude certain base packages or runtime environments of the catalog. This feature is useful, if the catalog is maintain by
a higher organization.
The path to the log4perl configuration file is defined by the tag \texttt{logconf}.
An examples how to configure arc and log4perl is provided in the Listings~\ref{lst:arc_conf} and ~\ref{lst:log_conf} and.

\lstsetCONFIGURE
\begin{lstlisting}[
        label=lst:arc_conf,
        caption={ [Example \textit{arc.conf} settings for janitor.]
                  \textbf{Example \textit{arc.conf} settings for janitor.}}
        ]
[janitor]
enabled="1"
logconf="/opt/nordugrid/etc/log.conf"
registrationdir="/var/spool/nordugrid/janitor"
installationdir="/var/spool/nordugrid/janitor/runtime"
downloaddir="/var/spool/nordugrid/janitor/download"
jobexpirytime="7200"
rteexpirytime="36"
uid="root"
gid="0"
allow_base="*"
allow_rte="*"

[janitor/nordugrid]
catalog="/var/spool/nordugrid/janitor/catalog/knowarc.rdf"
\end{lstlisting}

\lstsetCONFIGURE
\begin{lstlisting}[
        label=lst:log_conf,
        caption={ [Example \textit{log.conf} settings for janitor.]
                  \textbf{Example \textit{log.conf} settings for janitor.}}
        ]
# Master Loglevel
# [OFF | DEBUG | INFO | WARN | ERROR | FATAL]
#log4perl.threshold = OFF

log4perl.rootLogger = WARN, DebugLog, MainLog, ErrorLog
log4perl.appender.DebugLog = Log::Log4perl::Appender::Screen
log4perl.appender.DebugLog.layout = PatternLayout
log4perl.appender.DebugLog.layout.ConversionPattern = [%C] %d %p> %m%n

log4perl.appender.MainLog = Log::Log4perl::Appender::File
log4perl.appender.MainLog.Threshold = VERBOSE
log4perl.appender.MainLog.filename = /var/log/janitor.log
log4perl.appender.MainLog.layout = PatternLayout
log4perl.appender.MainLog.layout.ConversionPattern = %d %p> %m%n

log4perl.appender.ErrorLog = Log::Log4perl::Appender::File
log4perl.appender.ErrorLog.Threshold = ERROR
log4perl.appender.ErrorLog.filename = /var/log/janitor_error.log

log4perl.appender.ErrorLog.layout = PatternLayout
log4perl.appender.ErrorLog.layout.ConversionPattern = %d %p> %m%n
\end{lstlisting}


% ** Configuring arc.conf
% *** Where to store the data of janitor
% ** Configuring log.conf
% *** Where to store the log of janitor

\section{Limitations}

Janitor is designed for the usage on Linux distributions.

% Is designed for linux distributions. Windows or Mac won't work (Reason wget! maybe also the Executer).
