\documentclass{book}
\usepackage{geometry}
\geometry{verbose,a4paper,tmargin=2cm,bmargin=3cm,lmargin=2cm,rmargin=3cm}
\usepackage{makeidx}
\makeindex
\usepackage{amsmath}
\usepackage{amssymb}
\usepackage{color,framed}
\usepackage{longtable}                             %for breaking tables

%%%%---- Use EPS figures and graphics with LaTeX ----
%\usepackage{graphics}
%%%%---- USE PNG or JPG figures and graphicx with PDFLaTeX ----
\usepackage{graphicx}
\usepackage[linkbordercolor={0 0.8 0.8}]{hyperref}
\hypersetup{
  pdfauthor = {ARC developers},
  pdftitle = {ARC Clients},
  pdfsubject = {User's manual},
  pdfkeywords = {Grid,KnowARC,ARC,NorduGrid,client},
  pdfcreator = {PDFLaTeX with hyperref package},
  pdfproducer = {PDFLaTeX}
}

\bibliographystyle{IEEEtran}

\def\efill{\hfill\nopagebreak}%
\hyphenation{Nordu-Grid}
\setlength{\parindent}{0cm}
\setlength{\FrameRule}{1pt}
\setlength{\FrameSep}{8pt}
\addtolength{\parskip}{5pt}
\renewcommand{\thefootnote}{\fnsymbol{footnote}}
\renewcommand{\arraystretch}{1.3}
\newcommand{\dothis}[1]{\colorbox{shadecolor}{\texttt #1}}
\newcommand{\xrsl}[1]{\textbf{\sffamily #1}}
\newcommand{\uicommand}[1]{\textbf{\sffamily #1}}
\newcommand{\versions}[1]{\vspace{-0.3cm}(#1)\par}
\newcommand{\GT}{Globus Toolkit$\mathrm{^{TM}}$}
\newcommand{\globus}{Globus\textsuperscript{\textregistered}}
\definecolor{shadecolor}{rgb}{1,1,0.6}
\definecolor{salmon}{rgb}{1,0.9,1}
\definecolor{cyan}{rgb}{0,1,1}

\begin{document}
\def\today{\number\day/\number\month/\number\year}

\begin{titlepage}
   
   \begin{tabular}{rl}
      \resizebox*{3cm}{!}{\includegraphics{ng-logo.png}}
      &\parbox[b]{2cm}{\textbf \it {\hspace*{-1.5cm}NORDUGRID\vspace*{0.5cm}}}
   \end{tabular}
   
   \hrulefill
   
   {\raggedleft NORDUGRID-MANUAL-1\par}
   
   {\raggedleft \today\par}
   
   \vspace*{2cm}
   
   %%%% ---- The title ----
   
   \begin{center}
      
      \textsc{\Large ARC Clients}
      \Large \par \textit{User's Manual}
      
      \vspace*{2cm}
      
      % {\centering \large Refers to ARC release series 0.4 and up \large \par}
      
   \end{center}
   
   \vspace*{2cm}
   
   % \abstract{
   %   This is the user's  manual to ARC User Interface.
   % }
   
\end{titlepage}
% \thispagestyle{empty} $ $
% \newpage
% $\ $
\tableofcontents
\newpage

\chapter{Introduction} \label{sec:intro} 

The command line user interface of ARC consists of a set of commands
necessary for job submission and manipulation and data management.
This manual replaces the older version of \texttt{NORDUGRID-MANUAL-1} and is
valid for ARC versions 0.9 and above. Command line tools semantics is the same
as in earlier versions of ARC, roughly following that of basic Linux commands
and most common batch system commands. One obvious difference is change of
the legacy prefix from ``ng'' to the more appropriate ``arc''. This is not only
a cosmetic change: \textbf{behaviour of the commands also have changed}, as did
their functionalities and options.
\begin{framed}
  Users are strongly discouraged from modifying their old scripts by
  simply replacing ``ng'' with ``arc'' -- results may be unpredictable.
\end{framed}

\chapter{Commands} \label{sec:cli} 

\section{Proxy utilities} \label{sec:proxy}
\index{security} ARC now comes complete with a set of utilities to create
temporary user credentials (proxies) used to access Grid services.

\subsection{arcproxy}
\label{sec:arcproxy}

\index{arcproxy}\index{commands:arcproxy}In order to contact Grid services
(submit jobs, copy data, check information etc),
one has to present valid credentials. The are commonly formalized as so-called
``proxy'' certificates. There are many different types of proxy certificates,
with different Grids and different services having own preferences. \texttt{arcproxy}
is a powerful tool that can be used to generate most commonly used proxies. It supports
the following types:
\begin{itemize}
  \item pre-RFC GSI proxy
  \item RFC-compliant proxy (default)
  \item VOMS-extended proxy
  \item MyProxy delegation
\end{itemize}

\texttt{arcproxy} requires presence of user's private key and public certificate,
as well as the public certificate of their issuer CA.

\hspace*{0.5cm}
\begin{shaded}
   \uicommand{arcproxy [options]}
\end{shaded}
\versions{ARC 0.9}
\begin{longtable}{llp{8cm}}
   Options:&&\\
   \texttt{-P, --proxy}& \textit{path} & path to the proxy file\\
   \texttt{-C, --cert}& \textit{path} & path to the certificate file\\
   \texttt{-K, --key}& \textit{path} & path to the key file\\
   \texttt{-T, --cadir}& \textit{path} & path to the trusted certificate directory, only needed for VOMS client functionality\\
   \texttt{-V, --vomses}& \textit{path} & path to the VOMS server configuration file\\
   \texttt{-S, --voms}& \textit{voms[:command]} & Specify VOMS server (more than one VOMS server can be specified like this:\\
   & &--voms VOa:command1 --voms VOb:command2)\\
   & &:command is optional, and is used to ask for specific attributes(e.g. roles). Command options are:\\
   & &all -- put all of this DN's attributes into AC;\\
   & &list -- list all of the DN's attribute,will not create AC extension;\\
   & &/Role=yourRole -- specify the role, if this DN has such a role, the role will be put into AC\\
   & &/voname/groupname/Role=yourRole -- specify the VO,group and role; if this DN has such a role, the role will be put into AC\\
   \texttt{-G, --gsicom}& & use GSI communication protocol for contacting VOMS services\\
   \texttt{-O, --old}& & use GSI proxy (default is RFC 3820 compliant proxy)\\
   \texttt{-I, --info}& & print all information about this proxy. In order to show the Identity (DN without CN as suffix for proxy) of the certificate, the 'trusted certdir' is needed.\\
   \texttt{-U, --user}& \textit{string} & username for MyProxy server\\
   \texttt{-L, --myproxysrv}& \textit{URL} & URL of MyProxy server\\
   \texttt{-M, --myproxycmd}& \textit{PUT$|$GET} & command to MyProxy server. The command can be PUT and GET.\\
   & &PUT/put -- put a delegated credential to MyProxy server;\\
   & &GET/get -- get a delegated credential from MyProxy server, credential (certificate and key) is not needed in this case.\\
   \texttt{-c, --constraint}& \textit{string} & proxy constraints\\
   \texttt{-t, --timeout}& \textit{seconds} & timeout in seconds (default 20 seconds)\\
   \texttt{-d, --debug}& \textit{debuglevel}&debug level is one of  FATAL, ERROR, WARNING, INFO, DEBUG or VERBOSE\\
   \texttt{-z, --conffile}&\textit{filename}& configuration file (default {\$}HOME/.arc/client.conf)\\
   \texttt{-v, --version}& & print version information\\
   \texttt{-h, --help}& & print help page\\
\end{longtable}

Supported constraints are:
\begin{itemize}
  \item \texttt{validityStart=time} -- e.g. 2008-05-29T10:20:30Z; time when certificate becomes valid. Default is now.
  \item \texttt{validityEnd=time} -- time when certificate becomes invalid. Default is 43200 (12 hours) from start.
  \item \texttt{validityPeriod=time} -- e.g. 43200 or 12h or 12H; for how long certificate is valid. If neither \texttt{validityPeriod} nor \texttt{validityEnd} are specified, default is 12 hours
  \item \texttt{vomsACvalidityPeriod=time} -- e.g. 43200 or 12h or 12H; for how long the AC is valid. Default is the same as \texttt{validityPeriod}.
  \item \texttt{proxyPolicy=policy content} -- assigns specified string to proxy prolicy to limit it's functionality.
  \item \texttt{proxyPolicyFile=policy file}
\end{itemize}

MyProxy functionality can be used together with VOMS functionality.


\subsection{arcslcs}
\label{sec:arcslcs}

\index{arcslcs}\index{commands:arcslcs}This utility generates  short-lived
credential  based  on the credential to IdP in SAML2SSO profile (normally the
username/password to Shibboleth IdP).

\hspace*{0.5cm}
\begin{shaded}
   \uicommand{arcslcs [options]}
\end{shaded}
\versions{ARC 0.9}
\begin{longtable}{llp{8cm}}
   Options:&&\\
   \texttt{-S, --ur;}& \textit{URL} & URL of SLCS Service (e.g. https://127.0.0.1:60000/slcs)\\
   \texttt{-I, --idp}& \textit{URL} & the name of IdP (e.g. https://idp.testshib.org/idp/shibboleth)\\
   \texttt{-U, --user}& \textit{string} & User account to IdP\\
   \texttt{-P, --password}& \textit{string} & password for user accoutn to IdP\\
   \texttt{-Z, --keysize}& \textit{integer} & size of the private key, default is 1024\\
   \texttt{-K, --keypass}& \textit{} & passphrase for protecting the private key; if not set, the private key file will not be protected by the passphrase.\\
   \texttt{-L, --lifetime}& \textit{hours} & life time of the credential (hours)), starting with current time\\
   \texttt{-D, --storedir}& \textit{path} & store directory of the credential\\
   \texttt{-t, --timeout}& \textit{seconds} & timeout in seconds (default 20 seconds)\\
   \texttt{-d, --debug}& \textit{debuglevel}&debug level is one of  FATAL, ERROR, WARNING, INFO, DEBUG or VERBOSE\\
   \texttt{-c, --conffile}&\textit{filename}& configuration file (default {\$}HOME/.arc/client.conf)\\
   \texttt{-v, --version}& & print version information\\
   \texttt{-h, --help}& & print help page\\
\end{longtable}


\section{Job submission and management} \label{sec:ui} 
\index{job management}
The following commands are used for job submission and management,
such as status check, results retrieval, cancellation, re-submission
and such. The jobs must be described using a job description language. ARC
supports the following languages: JSDL~\cite{jsdl}, xRSL~\cite{xrsl} and
JDL~\cite{jdl}.

\subsection{arcsub}\label{sec:arcsub}
The \texttt{arcsub}\index{arcsub}\index{commands:arcsub} command is the
most essential one, as it is used for submitting jobs to the Grid
resources. \index{submit job}\texttt{arcsub} matches user's job
description to the information collected from the Grid, and the
optimal site is being selected for job submission. The job description
is then being forwarded to that site, in order to be submitted to the
Local Resource Management System (LRMS), which can be, e.g., PBS or
Condor or SGE etc.

\hspace*{0.5cm}
\begin{shaded}
   \uicommand{arcsub [options] [filename]}
\end{shaded}
\versions{ARC 0.9}
\begin{longtable}{llp{8cm}}
   Options:&&\\
   \texttt{-c, --cluster}&\verb#[-]#\textit{url}&explicitly select or reject (-) a specific site\\
   \texttt{-i, --index}&\textit{url}&explicitly select or reject (-) a specific index server\\
   \texttt{-e, --jobdescrstring}&\textit{filename}&string describing the job to be submitted\\
   \texttt{-f, --jobdescrfile}&\textit{filename}&file describing the job to be submitted\\
   \texttt{-j, --joblist}&\textit{filename}&file where user's job information will be stored\\
%   \texttt{-D, --dryrun}&&add dryrun option to the job description\\
   \texttt{-x, --dumpdescription}&&do not submit -- dump transformed job description to stdout\\
%   \texttt{-U, --unknownattr}&&allow unknown attributes in the job description\\
   \texttt{-b, --broker}&\textit{string}&select broker method (default is Random)\\
   \texttt{-n, --dolocalsandbox}& &store job descriptions in local sandbox (useful for eventual resubmission/migration)\\
   \texttt{-t, --timeout}&\textit{seconds}&timeout in seconds (default 20)\\
   \texttt{-d, --debug}&\textit{debuglevel}&debug level, FATAL, ERROR, WARNING, INFO, VERBOSE or DEBUG - default WARNING\\
   \texttt{-z, --conffile}&\textit{filename}& configuration file (default {\$}HOME/.arc/client.conf)\\
   \texttt{-v, --version}&&print version information\\
   \texttt{-h, --help}&&print help page\\
   Arguments:&&\\
   \texttt{filename}&&file describing the job(s) to be submitted\\
\end{longtable}

\begin{framed}
The \verb#-c# and \verb#-i# arguments accept meta-URLs of the format \texttt{GRID:URL}, where \texttt{GRID} indicates a Grid middleware flavour. Possible flavours are \texttt{ARC0}, \texttt{ARC1}, \texttt{CREAM} and \texttt{UNICORE}. For example, for index servers:
\begin{verbatim}
 ARC0:ldap://index.ng.org:2135/mds-vo-name=sweden,O=grid
 CREAM:ldap://cream.glite.org:2170/o=grid
\end{verbatim}
or clusters:

\verb# ARC0:ldap://ce.ng.eu:2135/nordugrid-cluster-name=ce.ng.eu,Mds-Vo-name=local,o=grid#

It is strongly recommended to use aliases for these long URLs. Aliases are specified in the configuration file (see Section~\ref{sec:client.conf}).
\end{framed}

As a shorthand \texttt{-f} can be omitted if the job description file is put last on the commandline.

A simple \textit{"Hello World"} job can look like:

\begin{shaded}
 arcsub -c my-test-site job.jsdl
\end{shaded}

The \verb#-c# option can be repeated several times, for example:
\begin{verbatim}
    arcsub -c alias1 -c alias2 job.xrsl
\end{verbatim}
This will submit a job to either \verb#alias1# or \verb#alias2#. To submit a job to any site except
\verb#badsite#, use \verb#-# sign in front of the name:
\begin{verbatim}
    arcsub -c -badsite job.xrsl
\end{verbatim}

If option \verb#-c# is not given, the \verb#arcsub# command locates the available sites by querying the
Information System. Default index services for the Information System are specified in the
configuration template distributed with the middleware, and can be overwritten both in the user's
configuration (see Section~\ref{sec:client.conf}) and from the command line using option
\verb#-i#. Different Grids use different notation for such index services.

A user has to have valid credentials (see Section~\ref{sec:proxy}) and be authorised at the specified site. A test file \texttt{job.jsdl} is shown below.

\begin{lstlisting}[language=xml]
<?xml version="1.0" encoding="UTF-8"?>
<JobDefinition
 xmlns="http://schemas.ggf.org/jsdl/2005/11/jsdl"
 xmlns:posix="http://schemas.ggf.org/jsdl/2005/11/jsdl-posix">
 <JobDescription>
   <JobIdentification>
     <JobName>Hello World job</JobName>
   </JobIdentification>
   <Application>
     <posix:POSIXApplication>
       <posix:Executable>/bin/echo</posix:Executable>
       <posix:Argument>'Hello World'</posix:Argument>
       <posix:Output>out.txt</posix:Output>
       <posix:Error>err.txt</posix:Error>
     </posix:POSIXApplication>
   </Application>
   <DataStaging>
     <FileName>out.txt</FileName>
     <CreationFlag>overwrite</CreationFlag>
     <DeleteOnTermination>false</DeleteOnTermination>
   </DataStaging>
   <DataStaging>
     <FileName>err.txt</FileName>
     <CreationFlag>overwrite</CreationFlag>
     <DeleteOnTermination>false</DeleteOnTermination>
   </DataStaging>
 </JobDescription>
</JobDefinition>
\end{lstlisting}

\begin{framed}
   If a job is successfully submitted, a \textbf{job identifier}
   (\textit{job ID})\index{job ID} is printed to standard output.
\end{framed}

The job ID uniquely identifies the job while it is being executed. Job IDs
differ strongly between Grid flavours, but basically they have a form of a URL.
You should use Job ID as a handle to refer to the job when doing other
job manipulations, such as querying job status (\verb#arcstat#),
killing it (\verb#arckill#), re-submitting (\verb#arcresub#), or
retrieving the result (\verb#arcget#).

\begin{framed}
   Every job ID is a valid URL for the job session directory. You can
   always use it to access the files related to the job, by using data
   management tools (see Chapter~\ref{sec:dm}).
\end{framed}

The job description in xRSL or JSDL format can be given either as an
argument on the command line, or can be read from a file. Several jobs can be
requested at the same time by giving more than one filename
argument, or by repeating the \verb#-f# or \verb#-e# options. It is
possible to mix \verb#-e# and \verb#-f# options in the same
\texttt{arcsub} command.

In order to keep track of submitted jobs, ARC client stores information in a
dedicated file, by default located in \texttt{{\$}HOME/.arc/jobs.xml}. It is
sometimes convenient to keep separate lists (e.g., for different kinds of jobs),
to be used later with e.g. \verb#arcstat#. This is achieved with the help of
\verb#-j# command line option.

The user interface transforms input job description into a format
that can be understood by the Grid services to which it is being
submitted. By specifying the \verb#-dumpdescription# option, such transformed
description is written to stdout instead of being submitted to the remote site.

% Description of brokers in a separate file
% text about brokers; may need to be extended and possibly re-used in another
% document, or moved as a separate subsection

Possible \index{broker}broker values for the \texttt{arcsub} command line option \verb#-b# are:
\begin{itemize}
 \item[--] \texttt{Random} -- ranks targets randomly (default)
 \item[--] \texttt{FastestQueue} -- ranks targets according to their queue length
 \item[--] \texttt{Benchmark[:name]} -- ranks targets according to a given benchmark, as specified by the \texttt{name}. If no benchmark is specified, CINT2000~\footnote{http://www.spec.org/cpu2000/CINT2000/} is used
 \item[--] \texttt{Data} -- ranks targets according the amount of megabytes of the
requested input files that are already in the computing resource’s cache.
 \item[--] \texttt{Python:$<$module$>$.$<$class$>$[:arguments]} -- ranks targets using any user-supplied custom Python broker module, optionally with broker arguments. Such module can reside anywhere in user's \texttt{PYTHONPATH}
% Below path is OS dependend.
 \item[--] \texttt{$<$otherbroker$>$[:arguments]} -- ranks targets using any user-supplied custom C++ broker plugin, optionally with broker arguments. Default location for broker plugins is \texttt{/usr/lib/arc} (may depend on the operating system), or the one specified by the \texttt{ARC\_PLUGIN\_PATH}.
\end{itemize}

% Below paths and description are OS dependend.
To write a custom broker in C++ one has to write a new specialization of the \texttt{Broker} base class and implement the \texttt{SortTargets} method in the new class. The class should be compiled as a loadable
module that has the proper ARC plugin descriptor for the new broker. For example, to build a broker plugin ``MyBroker'' one executes:
\begin{lstlisting}[language=sh]
  g++ -I /arc-install/include \
      -L /arc-install/lib \
      `pkg-config --cflags glibmm-2.4 libxml-2.0` \
      -o libaccmybroker.so -shared MyBroker.cpp
\end{lstlisting}
For more details, refer to \textit{libarclib} documentation~\cite{libarcclient}.



If you plan to resubmit or migrate jobs, you will have to use command line
option \verb#-n#, which will instruct the client to store complete job descriptions
in a local sandbox, such that a resubmitted/migrated job will be identical to the
original one.

It often happens that some sites that \verb#arcsub# has to contact
are slow to answer, or are down altogether. This will not prevent
you from submitting a job, but will slow down the submission. To
speed it up, you may want to specify a shorter timeout (default is
20 seconds) with the \verb#-t# option:
\begin{verbatim}
    arcsub -t 5 myjob.jsdl
\end{verbatim}

Default value for the timeout can be set in the user's configuration file.

If you would like to get diagnostics of the process of resource
discovery and requirements matching, a very useful option is
\verb#-d#. The following command:
\begin{verbatim}
    arcsub -d VERBOSE myjob.xrsl
\end{verbatim}

will print out the steps taken by the ARC client to find the
best cluster satisfying your job requirements. Possible diagnostics degrees, in the
order of increasing verbosity, are: \texttt{FATAL}, \texttt{ERROR}, \texttt{WARNING},
\texttt{INFO}, \texttt{VERBOSE} and \texttt{DEBUG}. Default is \texttt{WARNING}, and
it can be set to another value in the user's configuration file.

Default configuration file is {\$}HOME/.arc/client.conf. However, a user can chose any
other pre-defined configuration through option \verb#-z#.

Command line option \verb#-v# prints out version of the installed ARC client, and option
\verb#-h# provides a short help text.


\subsection{arcstat}
\label{sec:arcstat}

\begin{shaded}
   \uicommand{arcstat [options] [job ...]}
\end{shaded}
\versions{ARC 0.9}
\begin{longtable}{llp{8cm}}
   Options:&&\\
   \texttt{-a, --all}& & all jobs\\
   \texttt{-j, --joblist}& \textit{filename}& file containing a list of jobIDs\\
   \texttt{-c, --cluster}&\verb#[-]#\textit{name}&explicitly select or reject a specific site\\
%   \texttt{-c, --cluster}& & show information about a site (cluster)\\
   \texttt{-s, --status}& \textit{statusstr} &only select jobs whose status is \textit{statusstr}\\
%   \texttt{-i, --index}& \textit{url} &URL of an index service\\
   \texttt{-l, --long}& & long format (extended information)\\
   \texttt{-t, --timeout}& \textit{time}& timeout for queries (default 20 sec)\\
   \texttt{-d, --debug}& \textit{debuglevel}&debug level is one of  FATAL, ERROR, WARNING, INFO, VERBOSE or DEBUG\\
   \texttt{-z, --conffile}&\textit{filename}& configuration file (default {\$}HOME/.arc/client.conf)\\
   \texttt{-v, --version}& & print version information\\
   \texttt{-h, --help}& & print help page\\
   Arguments:&&\\
   \texttt{job ...} && list of job IDs and/or jobnames\\
\end{longtable}

The arcstat command returns the status of jobs in the Grid, and is typically issued with a
job ID (as returned by \verb#arcsub#) as an argument. It is also possible to use job name instead of
ID, but if several jobs have identical names, information will be collected about all of them. 
More than one job ID and/or name can be given.

If the \verb#-l# option is given, extended information is printed.

Options \verb#-a#, \verb#-c#, \verb#-s# and \verb#-j# do not use job ID or names. By
specifying the \verb#-a# option, the status of all active jobs will be shown. If the \verb#-j# option
is used, the list of jobs is read from a file with the specified filename, instead of
the default one (\texttt{{\$}HOME/.arc/jobs.xml}).

Option \verb#-c# accepts arguments in the \texttt{GRID:URL} notation,
as explained in the description of \texttt{arcsub}, or their aliases as 
specified in the configuration file.

Different sites may report different job states, depending on
the installed grid middleware version. Typical values can be e.g. 
``Accepted'', ``Preparing'', ``Running'', ``Finished'' or ``Deleted''.
Please refer to the respective middleware documentation for job state model
description.

Command line option \verb#-s# will instruct the client to display information
of only those jobs which status matches the instruction. This option must be given
together with either \verb#-a# or \verb#-c# ones, e.g.:
\begin{verbatim}
    arcstat -as Finished
\end{verbatim}

Other command line options are identical to those of \verb#arcsub#.

\subsection{arccat}
\label{sec:arccat}

It is often useful to monitor the job progress by checking what it
prints on the standard output or error. The command \texttt{arccat}
\index{arccat}\index{commands:arccat} assists here, extracting the
corresponding information from the execution cluster and dumping it
on the user's screen. It works both for running tasks and for the
finished ones. This allows a user to check the output of the
finished task without actually retreiving it.

\hspace*{0.5cm}
\begin{shaded}
   \uicommand{arccat [options] [job ...]}
\end{shaded}
\versions{ARC 0.9}
\begin{longtable}{llp{8cm}}
   Options:&&\\
   \texttt{-a, --all}& & all jobs\\
   \texttt{-j, --joblist}& \textit{filename} & file containing a list of job IDs\\
   \texttt{-c, --cluster}&\verb#[-]#\textit{url}&explicitly select or reject (-) a specific site\\
   \texttt{-s, --status}& \textit{statusstr} &only select jobs whose status is \textit{statusstr}\\
   \texttt{-o, --stdout}& & show the stdout of the job (default)\\
   \texttt{-e, --stderr}& & show the stderr of the job\\
   \texttt{-l, --gmlog}& & show the grid manager's error log of the job\\
   \texttt{-t, --timeout}& \textit{time} & timeout for queries (default 20 sec)\\
   \texttt{-d, --debug}& \textit{debuglevel}&debug level is one of  FATAL, ERROR, WARNING, INFO, VERBOSE or DEBUG\\
   \texttt{-z, --conffile}&\textit{filename}& configuration file (default {\$}HOME/.arc/client.conf)\\
   \texttt{-v, --version}& & print version information\\
   \texttt{-h, --help}& & print help page\\
   Arguments:&&\\
   \texttt{job ...} && list of job IDs and/or jobnames\\
\end{longtable}

The \texttt{arccat} command returns the standard output of a job
(\texttt{-o} option), the standard error (\texttt{-e} option) or
errors reported by either Grid Manager or A-REX (\texttt{-l} option).

Other command line options have the same meaning as in \verb#arcstat#.


\subsection{arcget}
\label{sec:arcget}

To retrieve the results of a finished job, the \texttt{arcget}
\index{arcget}\index{commands:arcget} command should be used. It
will transfer the files specified for download in job description
to the user's computer.

\hspace*{0.5cm}
\begin{shaded}
   \uicommand{arcget [options] [job ...]}
\end{shaded}
\versions{ARC 0.9}
\begin{longtable}{llp{8cm}}
   Options:&&\\
   \texttt{-a, --all}& & all jobs\\
   \texttt{-j, --joblist}& \textit{filename} & file containing a list of jobIDs\\
   \texttt{-c, --cluster}&\verb#[-]#textem{name}&explicitly select or reject a specific site (cluster)\\
   \texttt{-s, --status}& \textit{statusstr} &only select jobs whose status is \textit{statusstr}\\
   \texttt{-D, --dir} & \textit{dirname} & download path (the job directory will be created in that location)\\
   \texttt{-k, --keep}& & keep files in the Grid (do not clean)\\
   \texttt{-t, --timeout}& \textit{time} & timeout for queries (default 20 sec)\\
   \texttt{-d, --debug}& \textit{debuglevel}&debug level is one of  FATAL, ERROR, WARNING, INFO, VERBOSE or DEBUG\\
   \texttt{-z, --conffile}&\textit{filename}& configuration file (default {\$}HOME/.arc/client.conf)\\
   \texttt{-v, --version}& & print version information\\
   \texttt{-h, --help}& & print help page\\
   Arguments:&&\\
   \texttt{job ...} && list of job IDs and/or jobnames\\
\end{longtable}

Only the results of jobs that have finished can be downloaded. Just like in \verb#arstat# 
and \verb#arccat# cases, the job can be referred to either by the \texttt{jobID} that was returned by
\texttt{arcsub} at submission time, or by its name, if the job
description contained a job name attribute.

By default, the job is downloaded into a newly created directory in the current path, with the
name typically being a large random number. In order to instruct \verb#arcget# to use another
path, use option \verb#-D# (note the capital ``D''), e.g.
\begin{verbatim}
    arcget -D /tmp/myjobs "Test job nr 1"
\end{verbatim}

\begin{framed}
 After downloading, your jobs will be erased from the execution site! Use command line option \verb#-k#
 to keep finished jobs in the Grid.
\end{framed}

Other command line options are identical to those of e.g. \verb#arcstat#.

\subsection{arcsync}
\label{sec:arcsync}

It is advised to start every grid session by running \texttt{arcsync},
especially when changing workstations. The reason is that your job submission
history is cached on your machine, and if you are using ARC client
installations on different machines, your local lists of submitted jobs will be different. To synchronise
these lists with the information in the Information System, use the
\texttt{arcsync} \index{arcsync}\index{commands:arcsync} command.

\hspace*{0.5cm}
\begin{shaded}
   \uicommand{arcsync [options]}
\end{shaded}
\versions{ARC 0.9}
\begin{longtable}{llp{8cm}}
   Options:&&\\
   \texttt{-c, --cluster}&\verb#[-]#\textit{name}&explicitly select or reject a specific site\\
   \texttt{-i, --index}&\textit{url}&explicitly select or reject (-) a specific index server\\
   \texttt{-j, --joblist}&\textit{filename}&file where user's job information will be stored\\
   \texttt{-f, --force}& &don't ask for confirmation\\
%   \texttt{-m, --merge}& &merge the found jobs with the jobs already in the joblist\\
   \texttt{-T, --truncate}& &truncate the job list before synchronising\\
   \texttt{-t, --timeout}&\textit{seconds}&timeout in seconds (default 20)\\
   \texttt{-d, --debug}&\textit{debuglevel}&debug level, FATAL, ERROR, WARNING, INFO, VERBOSE or DEBUG - default WARNING\\
   \texttt{-z, --conffile}&\textit{filename}& configuration file (default {\$}HOME/.arc/client.conf)\\
   \texttt{-v, --version}&&print version information\\
   \texttt{-h, --help}&&print help page\\
\end{longtable}

The ARC client keeps a local list of jobs in the user's home
directory. If this file is lost,
corrupt, or the user wants to recreate the file on a different
workstation, the \texttt{arcsync} command will recreate this file from
the information available in the Information System.

Since  the  information  about  a job retrieved from a cluster can be slightly out of date if the user very recently
submitted or removed a job, a warning is issued when this command is run. The \verb#-f# option disables this warning.

If the job list is not empty when invoking syncronisation, the old jobs will be merged with the new jobs, unless
the \verb#-T# option is given (note the capital ``T''), in which case the job list will first be truncated and then the new jobs will be added.


\subsection{arckill}
\label{sec:arckill}

It happens that a user may wish to cancel a job. This is done by using
the \texttt{arckill} \index{arckill}\index{commands:arckill} command. A
job can be killed almost at any stage of processing through the Grid.

\hspace*{0.5cm}
\begin{shaded}
   \uicommand{arckill [options] [job ...]}
\end{shaded}
\versions{ARC 0.9}
\begin{longtable}{llp{8cm}}
   Options:&&\\
   \texttt{-a, --all}& & all jobs\\
   \texttt{-j, --joblist}& \textit{filename} & file containing a list of jobIDs\\
   \texttt{-c, --cluster}&\verb#[-]#\textit{url}&explicitly select or reject (-) a specific site\\
   \texttt{-s, --status}& \textit{statusstr} &only select jobs whose status is \textit{statusstr}\\
   \texttt{-k, --keep}& & keep files in the Grid (do not clean)\\
   \texttt{-t, --timeout}& \textit{time} & timeout for queries (default 20 sec)\\
   \texttt{-d, --debug}& \textit{debuglevel}&debug level is one of  FATAL, ERROR, WARNING, INFO, VERBOSE or DEBUG\\
   \texttt{-z, --conffile}&\textit{filename}& configuration file (default {\$}HOME/.arc/client.conf)\\
   \texttt{-v, --version}& & print version information\\
   \texttt{-h, --help}& & print help page\\
   Arguments:&&\\
   \texttt{job ...} && list of job IDs and/or jobnames\\
\end{longtable}

If a job is killed, its traces are being cleaned from the Grid. If you wish to keep the killed job
in the system, e.g. for a post-mortem analysis, use the \verb#-k# option.

\begin{framed}
   Job cancellation is an asynchronous process, such that it
   may take a few minutes before the job is actually cancelled.
\end{framed}

Command line options have the same meaning as the corresponding ones of \verb#arcstat# and others.


\subsection{arcclean}
\label{sec:arcclean}

If a job fails or gets killed with \verb#-k# option, or when you are not willing 
to retrieve the results for some reasons, a good practice for users is not to wait 
for the system to clean up the job leftovers, but to use 
\texttt{arcclean}\index{arcclean}\index{commands:arcclean} to release the disk 
space and to remove the job ID from the list of submitted jobs and from the Information System. 

\hspace*{0.5cm}
\begin{shaded}
   \uicommand{arcclean [options] [job ...]}
\end{shaded}
\versions{ARC 0.9}
\begin{longtable}{llp{8cm}}
   Options:&&\\
   \texttt{-a, --all}& & all jobs\\
   \texttt{-j, --joblist}& \textit{filename} & file containing a list of jobIDs\\
   \texttt{-c, --cluster}&\verb#[-]#textem{name}&explicitly select or reject a specific site (cluster)\\
   \texttt{-s, --status}& \textit{statusstr} &only select jobs whose status is \textit{statusstr}\\
   \texttt{-f, --force} & & removes the job ID from the local list even if the job is not found on the Grid\\
   \texttt{-t, --timeout}& \textit{time} & timeout for queries (default 20 sec)\\
   \texttt{-d, --debug}& \textit{debuglevel}&debug level is one of  FATAL, ERROR, WARNING, INFO, VERBOSE or DEBUG\\
   \texttt{-z, --conffile}&\textit{filename}& configuration file (default {\$}HOME/.arc/client.conf)\\
   \texttt{-v, --version}& & print version information\\
   \texttt{-h, --help}& & print help page\\
   Arguments:&&\\
   \texttt{job ...} && list of job IDs and/or jobnames\\
\end{longtable}

Only jobs that have finished or were cancelled can be cleaned.

It happens ever so often that the job is cleaned by the system, or is otherwise unreachable, and yet your 
local job list file still has it listed. Use \verb#-f# option in this case to forcefully remove such stale job
information from the local list. 

Other command line options have the same meaning as the corresponding ones of \verb#arcstat# and others.


\subsection{arcrenew}
\label{sec:arcrenew}

Quite often, the user proxy expires while the job is still running (or
waiting in a queue). In case such job has to upload output files to a
Grid location (Storage Element), it will fail. By using the \texttt{arcrenew}
\index{arcrenew}\index{commands:arcrenew} command, users can upload
a new proxy to the job. This can be done while a job is still running,
thus preventing it from failing

If a job has failed in file upload due to expired proxy, \texttt{arcrenew}
can be issued whithin 24 hours (or whatever is
the expiration time set by the site) after the job
end, which must be followed by \texttt{arcresume}. The Grid
Manager or A-REX will then attempt to finalize
the job by uploading the output files to the desired location. 

\hspace*{0.5cm}
\begin{shaded}
   \uicommand{arcrenew [options] [job ...]}
\end{shaded}
\versions{ARC 0.9}
\begin{longtable}{llp{8cm}}
   Options:&&\\
   \texttt{-a, --all}& & all jobs\\
   \texttt{-j, --joblist}& \textit{filename} & file containing a list of jobIDs\\
   \texttt{-c, --cluster}&\verb#[-]#textem{name}&explicitly select or reject a specific site (cluster)\\
   \texttt{-s, --status}& \textit{statusstr} &only select jobs whose status is \textit{statusstr}\\
   \texttt{-t, --timeout}& \textit{time} & timeout for queries (default 20 sec)\\
   \texttt{-d, --debug}& \textit{debuglevel}&debug level is one of  FATAL, ERROR, WARNING, INFO, VERBOSE or DEBUG\\
   \texttt{-z, --conffile}&\textit{filename}& configuration file (default {\$}HOME/.arc/client.conf)\\
   \texttt{-v, --version}& & print version information\\
   \texttt{-h, --help}& & print help page\\
   Arguments:&&\\
   \texttt{job ...} && list of job IDs and/or jobnames\\
\end{longtable}

\begin{framed}
 Prior to using \texttt{arcrenew}, be sure to actually create the
new proxy by running \verb#arcproxy#!
\end{framed}

Command line options have the same meaning as the corresponding ones of \verb#arcstat# and others.


\subsection{arcresume}
\label{sec:arcresume}

In some cases a user may want to restart a failed job, for example, when input
files become available, or the storage element for the output files came back
online, or when a proxy is renewed with \texttt{arcrenew}. This can be done using
the \texttt{arcresume}\index{arcresume}\index{commands:arcresume} command.

\begin{framed}
Make sure your proxy is still valid, or when uncertain, run \verb#arcproxy# followed by 
\verb#arcrenew# before \verb#arcresume#. The job will be resumed from the state where it has failed.
\end{framed}

\hspace*{0.5cm}
\begin{shaded}
   \uicommand{arcresume [options] [job ...]}
\end{shaded}
\versions{ARC 0.9}
\begin{longtable}{llp{8cm}}
   \texttt{-a, --all}& & all jobs\\
   \texttt{-j, --joblist}& \textit{filename} & file containing a list of jobIDs\\
   \texttt{-c, --cluster}&\verb#[-]#textem{name}&explicitly select or reject a specific site (cluster)\\
   \texttt{-s, --status}& \textit{statusstr} &only select jobs whose status is \textit{statusstr}\\
   \texttt{-t, --timeout}& \textit{time} & timeout for queries (default 20 sec)\\
   \texttt{-d, --debug}& \textit{debuglevel}&debug level is one of  FATAL, ERROR, WARNING, INFO, VERBOSE or DEBUG\\
   \texttt{-z, --conffile}&\textit{filename}& configuration file (default {\$}HOME/.arc/client.conf)\\
   \texttt{-v, --version}& & print version information\\
   \texttt{-h, --help}& & print help page\\
   Arguments:&&\\
   \texttt{job ...} && list of job IDs and/or jobnames\\
\end{longtable}

Command line options have the same meaning as the corresponding ones of \verb#arcstat# and others.


\subsection{arcresub}
\label{sec:arcresub}

Quite often it happens that a user would like to re-submit a job, but
has difficulties recovering the original job description xRSL file.
This happens when xRSL files are created by scripts on-fly, and
matching of xRSL to the job ID is not straightforward. The utility
called \texttt{arcresub}\index{arcresub}\index{commands:arcresub} helps
in such situations, allowing users to resubmit jobs.

\hspace*{0.5cm}
\begin{shaded}
   \uicommand{arcresub [options] [job ...]}
\end{shaded}
\versions{ARC 0.9}
\begin{longtable}{llp{8cm}}
   Options:&&\\
   \texttt{-a, --all}& & all jobs\\
   \texttt{-i, --index}&\textit{url}&explicitly select or reject (-) a specific index server\\
   \texttt{-j, --joblist}& \textit{filename} & file containing a list of jobIDs\\
   \texttt{-c, --cluster}&\verb#[-]#textem{name}&explicitly select or reject a specific source site\\
   \texttt{-q, --qluster}&\verb#[-]#textem{name}&explicitly select or reject a specific site as re-submission target\\
   \texttt{-m, --same}& &re-submit to the same site\\
   \texttt{-s, --status}& \textit{statusstr} &only select jobs whose status is \textit{statusstr}\\
   \texttt{-x, --dumpdescription}&&do not submit -- dump transformed job description to stdout\\
   \texttt{-k, --keep}& & keep files in the Grid (do not clean)\\
   \texttt{-b, --broker}&\textit{string}&select broker method (default is Random)\\
   \texttt{-t, --timeout}& \textit{time} & timeout for queries (default 20 sec)\\
   \texttt{-d, --debug}& \textit{debuglevel}&debug level is one of  FATAL, ERROR, WARNING, INFO, VERBOSE or DEBUG\\
   \texttt{-z, --conffile}&\textit{filename}& configuration file (default {\$}HOME/.arc/client.conf)\\
   \texttt{-v, --version}& & print version information\\
   \texttt{-h, --help}& & print help page\\
   Arguments:&&\\
   \texttt{job ...} && list of job IDs and/or jobnames\\
\end{longtable}

\begin{framed}
   Only jobs where the \verb#gmlog#\index{gmlog} attribute was
   specified in the job description can be resubmitted.
\end{framed}

More  than  one  jobid and/or jobname can be given. If several 
jobs were submitted with the same jobname all those jobs will be resubmitted.

Upon resubmission of a job the corresponding job description will be 
fetched from the local job list file. If input files have changed since the
original job submission, the job no longer remains the same job
and will therefore not be resubmitted. To make sure the job is always resubmittable,
submit it with \verb#arcsub -n#.

In case the job description is not found in the joblist, an attempt will be made to
retrieve it from the cluster holding the orignal job.  This however
may fail, since both the submission client and
the cluster can have made modifications to the job description.

Upon resubmision the job will receive a new job ID. The old job ID will be kept
in the local job list file, enabling future back tracing of the resubmitted job.

Regarding command line options, \verb#arcresub# behaves much like \verb#arcsub#, except that
\verb#-c# in this case indicates not the submission target site, but on the contrary, the \textbf{site 
from which the jobs will be resubmitted}. Submission target site is specified with
option \verb#-q#. If you wish to re-submit each job to the same site, use option \verb#-m#.

If the original job was successfully killed, its traces will be removed from the execution site,
unless the \verb#-k# option is specified.


\subsection{arcmigrate}
\label{sec:arcmigrate}

Quite often jobs end up stuck in long queues, and users wish to migrate them to a better
resource. Command \texttt{arcmigrate}\index{arcmigrate}\index{commands:arcmigrate} is
triggering this migration. It applies only to jobs submitted to A-REX, as other Grid
execution services do not support this functionality.

\hspace*{0.5cm}
\begin{shaded}
   \uicommand{arcmigrate [options] [job ...]}
\end{shaded}
\versions{ARC 0.9}
\begin{longtable}{llp{8cm}}
   Options:&&\\
   \texttt{-a, --all}& & all jobs\\
   \texttt{-i, --index}&\textit{url}&explicitly select or reject (-) a specific index server\\
   \texttt{-j, --joblist}& \textit{filename} & file containing a list of jobIDs\\
   \texttt{-c, --cluster}&\verb#[-]#textem{name}&explicitly select or reject a specific site (cluster)\\
   \texttt{-q, --qluster}&\verb#[-]#textem{name}&explicitly select or reject a specific site as re-submission target\\
   \texttt{-f, --forcemigration}& & force migration, ignoring kill failure\\
   \texttt{-b, --broker}&\textit{string}&select broker method (default is Random)\\
   \texttt{-t, --timeout}& \textit{time} & timeout for queries (default 20 sec)\\
   \texttt{-d, --debug}& \textit{debuglevel}&debug level is one of  FATAL, ERROR, WARNING, INFO, VERBOSE or DEBUG\\
   \texttt{-z, --conffile}&\textit{filename}& configuration file (default {\$}HOME/.arc/client.conf)\\
   \texttt{-v, --version}& & print version information\\
   \texttt{-h, --help}& & print help page\\
   Arguments:&&\\
   \texttt{job ...} && list of job IDs and/or jobnames\\
\end{longtable}

\begin{framed}
Currently only jobs having the A-REX status ``Running'', ``Executing'' or ``Queuing'' can be migrated
\end{framed}

Command line options \verb#-c# and \verb #-q# are interpreted in the same way as in \verb#arcresub#, namely, 
\verb#-c# indicates ``from'' and \verb #-q# -- ``to'' which site the job will be migrated.

If the job(s) is successfully migrated, a new job ID(s) is printed out. This jobID uniquely 
identifies the job while it is being executed.

\section{Data manipulation} \label{sec:dm} 
\index{data management} ARC provides basic data management tools,
which are simple commands for file copy and removal, with eventual use
of data indexing services.

\subsection{arcls}\label{sec:arcls}
\texttt{arcls}\index{arcls}\index{commands:arcls} is a simple
utility that allows to list contents and view some attributes of
objects of a specified (by a URL) remote directory.

\hspace*{0.5cm}
\begin{shaded}
   \uicommand{arcls [options] $<$URL$>$}
\end{shaded}
\versions{ARC 0.9}
\begin{longtable}{llp{8cm}}
    Options:&&\\
   \texttt{-l, --long} &  & detailed listing\\
   \texttt{-L, --locations} &  & detailed listing including URLs from which files can be downloaded\\
   \texttt{-m, --metadata} && display all available metadata\\
   \texttt{-r, --recursive} & \textit{recursion\_level} & operate recursively (if possible) up to specified level (0 - no recursion)\\
   \texttt{-t, --timeout}&\textit{seconds}&timeout in seconds (default 20)\\
   \texttt{-d, --debug}&\textit{debuglevel}&debug level, FATAL, ERROR, WARNING, INFO, VERBOSE or DEBUG - default WARNING\\
   \texttt{-z, --conffile}&\textit{filename}& configuration file (default {\$}HOME/.arc/client.conf)\\
   \texttt{-v, --version}&&print version information\\
   \texttt{-h, --help}&&print help page\\
    Arguments:&&\\
    \texttt{URL} && file or directory URL\\
\end{longtable}

This tool is very convenient not only because it allows to list files
at a Storage Element or records in an indexing service, but also
because it can give a quick overview of a job's working directory,
which is explicitly given by job ID.

Usage examples can be as follows:

\begin{verbatim}
    arcls -L rls://rls.nordugrid.org:38203/logical_file_name
    arcls -l gsiftp://lscf.nbi.dk:2811/jobs/1323842831451666535
    arcls srm://grid.uio.no:8446/srm/managerv2?SFN=/johndoe/log2
\end{verbatim}

Examples of URLs accepted by this tool can be found in
Section~\ref{sec:urls}, though \texttt{arcls} won't be able to list a
directory at an HTTP server, as they normally do not return directory
listings.

\subsection{arccp}\label{sec:arccp}

\texttt{arccp}\index{arccp}\index{commands:arccp} is a powerful
tool to copy files over the Grid. It is a part of the A-REX,
but can be used by the User Interface as well.
\hspace*{0.5cm}
\begin{shaded}
   \uicommand{arccp [options] $<$source$>$ $<$destination$>$}
\end{shaded}
\versions{ARC 0.9}
\begin{longtable}{llp{8cm}}
    Options:&&\\
   \texttt{-p, --passive} && use passive transfer (does not work if secure is on, default if secure is not requested)\\
   \texttt{-n, --nopassive} && do not try to force passive transfer\\
   \texttt{-f, --force} && if the destination is an indexing service and not the same as the source and the destination is already registered, then the copy is normally not done. However, if this option is specified the source is assumed to be a replica of the destination created in an uncontrolled way and the copy is done like in case of replication\\
   \texttt{-i, --indicate} && show progress indicator\\
   \texttt{-T, --notransfer} && do not transfer file, just register it - destination must be non-existing meta-url\\
   \texttt{-u, --secure} && use secure transfer (insecure by default)\\
   \texttt{-y, --cache} &\textit{path} & path to local cache (use to put file into cache). The \texttt{X509\_USER\_PROXY} and \texttt{X509\_CERT\_DIR} environment variables must be set correctly\\
   \texttt{-r, --recursive} & \textit{recursion\_level} & operate recursively (if possible) up to specified level (0 - no recursion)\\
   \texttt{-R, --retries} & \textit{number} & how many times to retry transfer of every file before failing\\
   \texttt{-t, --timeout}&\textit{seconds}&timeout in seconds (default 20)\\
   \texttt{-d, --debug}&\textit{debuglevel}&debug level, FATAL, ERROR, WARNING, INFO, VERBOSE or DEBUG - default WARNING\\
   \texttt{-z, --conffile}&\textit{filename}& configuration file (default {\$}HOME/.arc/client.conf)\\
   \texttt{-v, --version}&&print version information\\
   \texttt{-h, --help}&&print help page\\
    Arguments:&&\\
    \texttt{source} && source URL\\
    \texttt{destination} && destination URL\\
\end{longtable}

This command transfers contents of a file between 2 end-points.
End-points are represented by URLs or meta-URLs. For supported
endpoints please refer to Section~\ref{sec:urls}.

\texttt{arccp} can perform multi-stream transfers if \texttt{threads}
URL option is specified and server supports it.

Source URL can end with \verb#"/"#. In that case, the whole fileset
(directory) will be copied. Also, if the destination ends with
\verb#"/"#, it is extended with part of source URL after last
\verb#"/"#, thus allowing users to skip the destination file or
directory name if it is meant to be identical to the source.

Usage examples of \texttt{arccp} are:

\begin{verbatim}
    arccp gsiftp://lscf.nbi.dk:2811/jobs/1323842831451666535/job.out \
              file:///home/myname/job2.out
    arccp gsiftp://aftpexp.bnl.gov;threads=10/rep/my.file \
              rls://grid.uio.no/zebra4.f
    arccp http://www.nordugrid.org/data/somefile gsiftp://hathi.hep.lu.se/data/
\end{verbatim}

\subsection{arcrm}\label{sec:arcrm}

The \texttt{arcrm}\index{arcrm}\index{commands:arcrm}
command allows users to erase files at any location specified by a
valid URL.
\hspace*{0.5cm}
\begin{shaded}
   \uicommand{arcrm [options] $<$source$>$}
\end{shaded}
\versions{ARC 0.9}
\begin{longtable}{llp{8cm}}
    Options:&&\\
   \texttt{-f, --force} & &remove logical file name registration even if not all physical instances were removed\\
   \texttt{-t, --timeout}&\textit{seconds}&timeout in seconds (default 20)\\
   \texttt{-d, --debug}&\textit{debuglevel}&debug level, FATAL, ERROR, WARNING, INFO, VERBOSE or DEBUG - default WARNING\\
   \texttt{-z, --conffile}&\textit{filename}& configuration file (default {\$}HOME/.arc/client.conf)\\
   \texttt{-v, --version}&&print version information\\
   \texttt{-h, --help}&&print help page\\
    Arguments:&&\\
   \texttt{source} && source URL\\
\end{longtable}

\begin{framed}
   A convenient use for \texttt{arcrm} is to erase the files in a data
   indexing catalog (LFC, RLS or such), as it will not only remove the
   physical instance, but also will clean up the database record.
\end{framed}

Here is an \texttt{arcrm} example:

\begin{verbatim}
    arcrm lfc://grid.uio.no/grid/atlas/AOD_0947.pool.root
\end{verbatim}

% \subsection{arcacl}\label{sec:arcacl}
% 
% \index{arcacl}\index{commands:arcacl}This command retrieves or modifies
% access control information associated with a stored object if service
% supports GridSite GACL language~\cite{gacl} for access control.
% \hspace*{0.5cm}
% \begin{shaded}
%    \uicommand{arcacl [options] get$|$put $<$URL$>$}
% \end{shaded}
% \versions{ARC 0.9}
% \begin{longtable}{llp{8cm}}
%    Options:&&\\
%     \texttt{-d, -debug} & \textit{debuglevel} &debug level is one of  FATAL, ERROR, WARNING, INFO, VERBOSE or DEBUG\\
%     \texttt{-v} && print version information\\
%     \texttt{-h} && short help\\
%    Arguments:&&\\
%     \texttt{get} &\textit{URL}& get Grid ACL for the object\\
%     \texttt{put} &\textit{URL}& set Grid ACL for the object\\
%     \texttt{URL} && object URL; curently only gsiftp and sse URLs are supported\\
% \end{longtable}
% 
% The ACL document (an XML file) is printed to standard output when
% \texttt{get} is requested, and is acquired from standard input when
% \texttt{set} is specified\footnote{In ARC $\leq$ 0.5.28, \texttt{set}
%   was used instead of \texttt{put}}. Usage examples are:
% \begin{verbatim}
%     arcacl get gsiftp://se1.ndgf.csc.fi/ndgf/tutorial/dirname/filename
%     arcacl set gsiftp://se1.ndgf.csc.fi/ndgf/tutorial/dirname/filename < myacl
% \end{verbatim}

% \subsection{arctransfer}\label{sec:arctransfer}
% \index{arctransfer}\index{commands:arctransfer}
% The \texttt{arctransfer} command is not implemented.



\subsection{chelonia}\label{sec:chelonia}

\index{chelonia}\index{commands:chelonia}\texttt{chelonia} is a client tool for
accessing the Chelonia storage system. With it it is possible to
create, remove and list file collections, upload, download and remove files,
and move and stat collections and files, using Logical Names (LN).
\hspace*{0.5cm}
\begin{shaded}
   \uicommand{chelonia [options] $<$method$>$ [arguments]}
\end{shaded}
\versions{ARC 0.9}
\begin{longtable}{llp{8cm}}
   Options:&&\\
   \texttt{-b} & \textit{URL} & URL of Bartender to connect\\
   \texttt{-x} && print SOAP XML messages\\
   \texttt{-v} && verbose mode\\
   \texttt{-z}&\textit{filename}& configuration file (default {\$}HOME/.arc/client.conf)\\
   \texttt{-w} && allow to run without the ARC python client libraries (with limited functionality)\\
   Methods:&&\\
   \texttt{stat} &\textit{LN [LN ...]}& get detailed information about an entry or several\\
   \texttt{makeCollection, make, mkdir} &\textit{LN}& create a collection\\
   \texttt{unmakeCollection, unmake, rmdir} &\textit{LN}& remove an empty collection\\
   \texttt{list, ls} &\textit{LN}& list the content of a collection\\
   \texttt{move, mv} &\textit{source target}& move entries within the namespace (both LNs)\\
   \texttt{putFile, put} &\textit{source target}& upload a file from a \textit{source} to a \textit{target} (both specified as LNs))\\
   \texttt{getFile, get} &\textit{source [target]}& download a file from a \textit{source} to a \textit{target}\\
   \texttt{delFile, del, rm} &\textit{LN [LN ...]}& remove file(s))\\
   \texttt{modify, mod} &\textit{string}& modify metadata\\
   \texttt{policy, pol} &\textit{string}& modify access policy rules. The string has a form $<$LN$>$ $<$changeType$>$ $<$identity$>$ $<$action list$>$.\\
   & &$<$changeType$>$ could be `set', `change' or `clear'\\
   & & `set': sets the action list to the given user overwriting the old one\\
   & & `change': modify the current action list with adding and removing actions\\
   & & `clear': clear the action list of the given user\\
   & &$<$identity$>$ could be a `$<$UserDN$>$' or a `VOMS:$<$VO name$>$'\\
   & &$<$action list$>$ is a list actions prefixed with `+' or `-', e.g. `+read +addEntry -delete'; possible actions are: \texttt{read}, \texttt{addEntry}, \texttt{removeEntry}, \texttt{delete}, \texttt{modifyPolicy}, \texttt{modifyStates}, \texttt{modifyMetadata}\\
   \texttt{unlink} &\textit{string}& remove a link to an entry from a collection without removing the entry itself\\
   \texttt{credentialsDelegation, cre} &\textit{string}& delegate credentials for using gateway\\
   \texttt{removeCredentials, rem} &\textit{string}& remove previously delegated credentials\\
   \texttt{makeMountPoint, makemount} &\textit{string}& create a mount point\\
\end{longtable}

Without arguments, each method prints its own help.

Examples:
\begin{verbatim}
    chelonia list /
    chelonia put orange /
    chelonia stat /orange
    chelonia get /orange /tmp
    chelonia mkdir /fruits
    chelonia mkdir /fruits/apple
    chelonia mv /orange /fruits
    chelonia ls /fruits
    chelonia rmdir /fruits/apple
    chelonia rmdir /fruits
    chelonia rm /fruits/orange
    chelonia policy / change ALL +read +addEntry
    chelonia modify /pennys-orange set states neededReplicas 2
\end{verbatim}

\subsubsection{stat} % (fold)
\label{ssub:stat}
With the \texttt{stat} method it is possible to get all the metadata about one or more entry (file, collection, etc.). The entries are specified with their Logical Name (LN).
\hspace*{0.5cm}
\begin{shaded}
   \uicommand{chelonia stat $<$LN$>$ [$<$LN$>$ ...]}
\end{shaded}

The output contains key-value pairs grouped in sections. The `states' section contains the size and the checksum of a file, the number of needed replicas, and whether a collection is closed or not; the `entry' section contains the DN of the owner, the globally unique ID (GUID) of the entry, and the type of the entry (file, collection, etc.); the `parents' section contains the GUID of the parent collection(s) of this entry, and the name of this entry in that collection separated with a `/'; the `locations' sections contains the location of the replicas of a file, which contains of the ID (the URL) of the storage element, the ID of the replica within the storage element, and the state of the replica; the `timestamps' section contains the creation time of the entry; the `entries' section contains the name and GUID of the entries of a collection.
Example stat of a file:
\begin{verbatim}
$ chelonia stat /thing
'/thing': found
  states
    checksumType: md5
    neededReplicas: 3
    size: 6
    checksum: a0186a90393bd4a639a1ce35d8ef85f6
  entry
    owner: /C=HU/O=NIIF CA/OU=GRID/OU=NIIF/CN=Nagy Zsombor
    GUID: 398CBDEA-E282-4735-8DF6-2464CD00BE2D
    type: file
  parents
    0/thing: parent
  locations
    https://localhost:60000/Shepherd D519F687-EF65-4AEA-9766-E6E2D42166C4: alive
  timestamps
    created: 1257351119.3
\end{verbatim}
Example stat of a collection:
\begin{verbatim}
$ chelonia stat /
'/': found
  states
    closed: no
  entry
    owner: /C=HU/O=NIIF CA/OU=GRID/OU=NIIF/CN=Nagy Zsombor
    GUID: 0
    type: collection
  timestamps
    created: 1257351114.37
  entries
    thing: 398CBDEA-E282-4735-8DF6-2464CD00BE2D    
\end{verbatim}
% subsubsection stat (end)

\subsubsection{makeCollection} % (fold)
\label{ssub:makecollection}
With the \texttt{makeCollection} or \texttt{mkdir} method it is possible to create a new empty collection. The requested Logical Name (LN) should be specified.
\hspace*{0.5cm}
\begin{shaded}
   \uicommand{chelonia makeCollection $<$LN$>$}
\end{shaded}

The parent collection of the requested Logical Name must exist.

Example output of the method:
\begin{verbatim}
$ chelonia mkdir /newcoll
Creating collection '/newcoll': done

$ chelonia mkdir /nonexistent/newcoll
Creating collection '/nonexistent/newcoll': parent does not exist
\end{verbatim}
% subsubsection makecollection (end)

\subsubsection{unmakeCollection} % (fold)
\label{ssub:unmakecollection}
With the \texttt{unmakeCollection} or \texttt{rmdir} method it is possible to delete an empty collection which is specified by its Logical Name (LN).
\hspace*{0.5cm}
\begin{shaded}
   \uicommand{chelonia unmakeCollection $<$LN$>$}
\end{shaded}

Example output of the method:
\begin{verbatim}
$ chelonia rmdir /newcoll
Removing collection '/newcoll': removed

$ chelonia rmdir /dir
Removing collection '/dir': collection is not empty
\end{verbatim}
% subsubsection unmakecollection (end)

\subsubsection{list} % (fold)
\label{ssub:list}
With the \texttt{list} or \texttt{ls} method it is possible to list the contents of one or more collections which are specified by their Logical Name (LN).
\hspace*{0.5cm}
\begin{shaded}
   \uicommand{chelonia list $<$LN$>$ [$<$LN$>$ ...]}
\end{shaded}

Example output of the method:
\begin{verbatim}
$ chelonia list / /newcoll
'/newcoll': collection
    empty.
'/': collection
  thing   <file>
  dir     <collection>
  newcoll <collection>
\end{verbatim}
% subsubsection list (end)

\subsubsection{move} % (fold)
\label{ssub:move}
With the \texttt{move} or \texttt{mv} method it is possible to move a file or collection within the namespace of chelonia (including renaming the entry). The source path and the target path should be specified as Logical Names
\hspace*{0.5cm}
\begin{shaded}
   \uicommand{chelonia list $<$sourceLN$>$ $<$targetLN$>$}
\end{shaded}

Example output of the method:
\begin{verbatim}
$ chelonia mv /thing /newcoll/
Moving '/thing' to '/newcoll/': moved 

$ chelonia mv /newcoll/thing /newcoll/othername
Moving '/newcoll/thing' to '/newcoll/othername': moved   
\end{verbatim}
% subsubsection move (end)

\subsubsection{putFile} % (fold)
\label{ssub:putfile}
With the \texttt{putFile} or \texttt{put} method it is possible to upload a new file into the system creating a new Logical Name (LN). It is possible the specify the number of needed replicas.
\hspace*{0.5cm}
\begin{shaded}
   \uicommand{chelonia putFile $<$source filename$>$ $<$target LN$>$ [$<$number of replicas needed$>$]}
\end{shaded}

Example output of the method:
\begin{verbatim}
$ chelonia put thing /newcoll/
'thing' (6 bytes) uploaded as '/newcoll/thing'.    
\end{verbatim}
% subsubsection putfile (end)

\subsubsection{getFile} % (fold)
\label{ssub:getfile}
With the \texttt{getFile} or \texttt{get} method it is possible to download a file specified with its Logical Name (LN). If the target local path is not given, then the file will be put into the local directory.
\hspace*{0.5cm}
\begin{shaded}
   \uicommand{chelonia getFile $<$source LN$>$ [$<$target filename$>$]}
\end{shaded}

Example output of the method:
\begin{verbatim}
$ chelonia get /newcoll/thing newlocalname
'/newcoll/thing' (6 bytes) downloaded as 'newlocalname'.
\end{verbatim}
% subsubsection getfile (end)

\subsubsection{delFile} % (fold)
\label{ssub:delfile}
With the \texttt{delFile} or \texttt{rm} method it is possible to delete one or more files from the system.
\hspace*{0.5cm}
\begin{shaded}
   \uicommand{chelonia delFile $<$LN$>$ [$<$LN$>$ ...]}
\end{shaded}

Example output of the method:
\begin{verbatim}
$ chelonia rm /newcoll/othername
/newcoll/othername: deleted    
\end{verbatim}
% subsubsection delfile (end)

\subsubsection{modify} % (fold)
\label{ssub:modify}

With the \texttt{modify} or \texttt{mod} method it is possible to modify some metadata of an entry.
\hspace*{0.5cm}
\begin{shaded}
   \uicommand{chelonia modify $<$LN$>$ $<$changeType$>$ $<$section$>$ $<$property$>$ $<$value$>$}
\end{shaded}

The possible values of `changeType' are `set' (sets the property to value within the given section), `unset' (removes the property from the given section - the `value' does not matter) and `add' (sets the property to value within the given section only if it does not exist yet). 

To change the number of needed replicas for a file:
\begin{verbatim}
chelonia modify <LN> set states neededReplicas <number of needed replicas>    
\end{verbatim}

To close a collection:
\begin{verbatim}
chelonia modify <LN> set states closed yes    
\end{verbatim}

To change metadata key-value pairs:
\begin{verbatim}
chelonia modify <LN> set|unset|add metadata <key> <value>    
\end{verbatim}
% subsubsection modify (end)

%\section{Test suite} \label{sec:testsuite}
%\input{arctest}

%\section{Third-party commands} \label{sec:ui-other}
%\input{ui-other}


\chapter{URLs}\label{sec:urls}
\index{URLs}
File locations in ARC can be specified both as local file
names, and as Internet standard \textit{Uniform Resource Locators
  (URL)}\index{URL}. There are also some additional URL
\textit{options} that can be used.

The following transfer protocols and metadata servers are supported:

\begin{tabular}{lp{10cm}}
   \texttt{ftp} & ordinary \textit{File Transfer Protocol (FTP)}\\
   \texttt{gsiftp} & GridFTP, the \globus\ -enhanced FTP protocol with
security, encryption, etc. developed by The Globus Alliance \cite{globus}\\
   \texttt{http} & ordinary \textit{Hyper-Text Transfer Protocol (HTTP)} with PUT and GET methods using multiple streams\\
   \texttt{https} & HTTP with SSL v3\\
   \texttt{httpg} & HTTP with \globus\  GSI\\
   \texttt{ldap} & ordinary \textit{Lightweight Data Access Protocol (LDAP)}~\cite{ldap}\\
   \texttt{rls} & \globus\  \textit{Replica Location Service (RLS)}~\cite{rls}\\
   \texttt{lfc} & LFC catalog and indexing service of EGEE gLite~\cite{glite}\\
   \texttt{srm} & Storage Resource Manager (SRM) service~\cite{srm}\\
   \texttt{file} & local to the host file name with a full path\\
\end{tabular}

An URL can be used in a standard form, i.e.
\begin{shaded}
   \verb#protocol://[host[:port]]/file#
\end{shaded}

Or, to enhance the performance, it can have additional options:
\begin{shaded}
   \verb#protocol://[host[:port]][;option[;option[...]]]/file#
\end{shaded}

For a metadata service URL, construction is the following:
\begin{shaded}
   \verb#protocol://[url[|url[...]]@]host[:port][;option[;option[...]]/#\\
   \verb#     lfn[:metadataoption[:metadataoption[...]]#
\end{shaded}

For the SRM service, the syntax is
\begin{shaded}
   \verb#srm://host[:port][;options]/[service_path?SFN=]file#
\end{shaded}

Versions 1.1 and 2.2 of the SRM protocol are supported. The
default \emph{service\_path} is srm/managerv2 when the server supports
v2.2, srm/managerv1 otherwise.

The URL components are:

\begin{tabular}{lp{10cm}}
   \verb#host[:port]# & Hostname or IP address [and port] of a server\\
   \verb#lfn# & Logical File Name\\
   \verb#url# & URL of the file as registered in indexing service\\
   \verb#service_path# & End-point path of the web service\\
   \verb#file# & File name with full path\\
   \verb#option# & URL option\\
   \verb#metadataoption# & Metadata option for indexing service\\
\end{tabular}

The following options are supported for location URLs:\index{URL:options}

\begin{longtable}{lp{10cm}}
   \verb#threads=<number># & specifies number of parallel
   streams to be used by GridFTP or HTTP(s,g); default value is 1,
   maximal value is 10\\
   \verb#cache=yes|no|renew|copy# & indicates whether the GM should
   cache the file; default for input files is \verb#yes#. \verb#renew#
   forces a download of the file, even if the cached copy is still valid.
   \verb#copy# forces the cached file to be copied (rather than linked) to
   the session dir, this is useful if for example the file is to be modified.\\
   \verb#readonly=yes|no# & for transfers to \verb#file://# destinations,
   specifies whether the file should be
   read-only (unmodifiable) or not; default is \verb#yes# \\
   \verb#secure=yes|no# & indicates whether the GridFTP data
   channel should be encrypted; default is \verb#no#\\
   \verb#blocksize=<number># & specifies size of
   chunks/blocks/buffers used in GridFTP or HTTP(s,g) transactions;
   default is protocol dependent\\
   \verb#checksum=cksum|md5|adler32|no# & specifies the algorithm for checksum to be
   computed (for transfer verification or provided to the indexing server). This is overridden
   by any metadata options specified (see below). If this option is
   not provided, the default for the protocol is used. \verb#checksum=no#
   disables checksum calculation.\\
   \verb#exec=yes|no# & means the file should be treated as executable\\
   \verb#preserve=yes|no# & specify if file must be uploaded to this
   destination even if job processing failed (default is \verb#no#)\\
   \verb#guid=yes|no# & make software use GUIDs instead of LFNs while
   communicating to indexing services; meaningful for \verb#rls://#
   only\\
   \verb#overwrite=yes|no# & make software try to overwrite existing
   file(s), i.e. before writing to destination, tools will try to remove
   any information/content associated with specified URL\\
   \verb#protocol=gsi|gssapi# & to distinguish between two kinds of 
   \verb#httpg#. \verb#gssapi# stands for implemention using only GSSAPI 
   functions to wrap data and \verb#gsi# uses additional headers as 
   implmented in Globus IO\\
   \verb#spacetoken=<pattern># & specify the space token to be used for
   uploads to SRM storage elements supporting SRM version 2.2 or higher\\
   \verb#autodir=yes|no# & specify if before writing to specified location
   software should try to create all directories mentioned in specified
   URL. Currently this applies to FTP and GridFTP only. Default for those
   protocols is \verb#yes#\\
   \verb#tcpnodelay=yes|no# & controls the use of the TCP\_NODELAY
   socket option (which disables the Nagle algorithm). Applies to
   http(s) only. Default is \verb#no#\\ 
\end{longtable}

Local files are referred to by specifying either a location relative
to the job submission working directory, or by an absolute path (the
one that starts with "/"), preceded with a \verb#file://# prefix.

Metadata service URLs also support metadata options which can be used
for register additional metadata attributes or query the service using
metadata attributes. These options are specified at the end of the LFN
and consist of name and value pairs separated by colons. The following
attributes are supported:

\begin{tabular}{lp{10cm}}
   \verb#guid# & GUID of the file in the metadata service \\
   \verb#checksumtype# & Type of checksum. Supported values are cksum
   (default), md5 and adler32 \\
   \verb#checksumvalue# & The checksum of the file \\
\end{tabular}

Currently these metadata options are only supported for lfc:// URLs.

\begin{framed}
   Examples of URLs are:\\
   \\
   \verb#http://grid.domain.org/dir/script.sh#\\
   \verb#gsiftp://grid.domain.org:2811;threads=10;secure=yes/dir/input_12378.dat#\\
   \verb#ldap://grid.domain.org:389/lc=collection1,rc=Nordugrid,dc=nordugrid,dc=org#\\
   \verb#rls://gsiftp://se.domain.org/datapath/file25.dat@grid.domain.org:61238/myfile02.dat#$^1$\\
   \verb#file:///home/auser/griddir/steer.cra#\\
   \verb#lfc://srm://srm.domain.org/griddir@lfc.domain.org/user/file1:guid=\# \\
   \verb#    bc68cdd0-bf94-41ce-ab5a-06a1512764dc:checksumtype=adler32:checksumvalue=12345678#$^2$\\
   \verb#lfc://lfc.domain.org;cache=no/:guid=bc68cdd0-bf94-41ce-ab5a-06a1512764d#$^3$\\
\end{framed}

$^1$This is a destination URL. The file will be copied to the GridFTP
server at \texttt{se.domain.org} with the path \texttt{datapath/file25.dat} and
registered in the RLS indexing service at \texttt{grid.domain.org} with the LFN
\texttt{myfile02.dat}.

$^2$This is a destination URL. The file will be copied to
\texttt{srm.domain.org} at the path \texttt{griddir/file1} and registered to the LFC
service at \texttt{lfc.domain.org} with the LFN \texttt{/user/file1}. The given GUID
and checksum attributes will be registered.

$^3$This is a source URL. The file is registered in the LFC service at
\texttt{lfc.domain.org} with the given GUID and can be copied or queried by
this URL.


\chapter{ARC Client Configuration} \label{sec:client.conf} 
Default behaviour of an ARC client can be configured by specifying
alternative values for some parameters in the client configuration
file. The file is called \texttt{client.conf} and is located in
directory \texttt{.arc} in user's home area:
\begin{shaded}
 {\$}HOME/.arc/client.conf
\end{shaded}
If this file is not present or does not contain the relevant
configuration information, the global configuration files (if exist)
or default values are used instead. Some client tools may be able to create
the default \texttt{{\$}HOME/.arc/client.conf}, if it does not exist.

The ARC configuration file consists of several configuration blocks.
Each configuration block is identified by a keyword and contains
configuration options for a specific part of the ARC middleware.

The configuration file is written in a plain text format known as INI.
Configuration blocks start with identifying keywords inside square brackets.
Typically, first comes a common block: \verb#[common]#. Thereafter follows one
or more attribute-value pairs written one on each line in the following
format:

\begin{framed}
\begin{verbatim}
[common]
attribute1=value1
attribute2=value2
attribute3=value3 value4
# comment line 1
# comment line 2
...
\end{verbatim}
\end{framed}

%For multi-valued attributes, several elements or attribute-value pairs
%have to be specified -- one per each value.

Most attributes have counterpart command line options. Command line options
always overwrite configuration attributes.

Two blocks are currently recognized, \texttt{[common]} and
\texttt{[alias]}. Following sections describe supported attributes per block.

\section{Block \texttt{[common]}}

\phantomsection
\index{configuration:defaultservices}\addcontentsline{toc}{subsection}{defaultservices}
\hspace*{0.5cm}
\begin{shaded}
  \uicommand{defaultservices}
\end{shaded}
\textbf{This attribute is multi-valued.}

This attribute is used to specify default services to be used. Defining such in
the user configuration file will override the default services set in the system
configuration.

The value of this attribute should follow the format:
\begin{verbatim}
  service_type:flavour:service_url
\end{verbatim}

where \texttt{service\_type} is type of service (e.g. \texttt{computing} or
\texttt{index}), \texttt{flavour} specifies type of middleware plugin to use
when contacting the service (e.g. ARC0, ARC1, CREAM, UNICORE, etc.) and
\texttt{service\_url} is the URL used to contact the service. Several services
can be listed, separated with a blank space (no line breaks allowed).

Example:
\begin{verbatim*}
defaultservices=index:ARC0:ldap://index1.ng.org:2135/Mds-Vo-name=testvo,o=grid
 index:ARC1:https://index2.ng.org:50000/isis
 computing:ARC1:https://ce.arc.org:60000/arex
 computing:CREAM:ldap://ce.glite.org:2170/o=grid
 computing:UNICORE:https://ce.unicore.org:8080/test/services/BESFactory?res=default_bes_factory
\end{verbatim*}

\phantomsection
\index{configuration:rejectservices}\addcontentsline{toc}{subsection}{rejectservices}
\hspace*{0.5cm}
\begin{shaded}
  \uicommand{rejectservices}
\end{shaded}
\textbf{This attribute is multi-valued.}

This attribute can be used to indicate that a certain service should be
rejected (``blacklisted''). Several services can be listed, separated with a
blank space (no line breaks allowed).

Example:
\verb#    rejectservices=computing:ARC1:https://bad.service.org/arex#


 \phantomsection
 \index{configuration:verbosity}\addcontentsline{toc}{subsection}{verbosity}
 \hspace*{0.5cm}
 \begin{shaded}
   \uicommand{verbosity}
 \end{shaded}
 Default verbosity (debug) level to use for the ARC clients. Corresponds to the
\verb#-d# command line option of the clients. Default value is \texttt{WARNING},
possible values are \texttt{FATAL}, \texttt{ERROR}, \texttt{WARNING},
\texttt{INFO}, \texttt{VERBOSE} or \texttt{DEBUG}.

Example:
\verb#    verbosity=INFO#

\phantomsection
\index{configuration:timeout}\addcontentsline{toc}{subsection}{timeout}
\hspace*{0.5cm}
\begin{shaded}
  \uicommand{timeout}
\end{shaded}
Sets the period of time the client should wait for a service (information,
computing, storage etc) to respond when communicating with it. The period
should be given in seconds. Default value is 20 seconds. This attribute
corresponds to the \verb#-t# command line option.

Example:
\verb#    timeout=10#

\phantomsection
\index{configuration:brokername}\addcontentsline{toc}{subsection}{brokername}
\hspace*{0.5cm}
\begin{shaded}
  \uicommand{brokername}
\end{shaded}

Configures which brokering algorithm to use during job submission. This attribute
corresponds to the \verb#-b# command line option. The
default one is the \texttt{Random} broker that chooses targets randomly.
Another possibility is, for example, the \texttt{FastestQueue} broker that
chooses the target with the shortest estimated queue waiting time. For an
overview of brokers, please refer to Section~\ref{sec:arcsub}.

 Example:
\verb#    brokername=Data#

\phantomsection
\index{configuration:brokerarguments}\addcontentsline{toc}{subsection}{brokerarguments}
\hspace*{0.5cm}
\begin{shaded}
  \uicommand{brokerarguments}
\end{shaded}

This attribute is used in case a broker comes with arguments. This corresponds to the
parameter that follows column in the \verb#-b# command line option.

Example:
\verb#    brokerarguments=cow#

\phantomsection
\index{configuration:joblist}\addcontentsline{toc}{subsection}{joblist}
\hspace*{0.5cm}
\begin{shaded}
  \uicommand{joblist}
\end{shaded}

Path to the job list file. This file will be used by commands such as \texttt{arcsub}, \texttt{arcstat},
\texttt{arcsync} etc. to read and write information about jobs. This attribute
corresponds to the \verb#-j# command line option. The default
location of the file is in the {\$}HOME/.arc/client.conf directory with the
name \texttt{jobs.xml}.

Example:
\begin{verbatim}
    joblist=/home/user/run/jobs.xml
    joblist=C:\\run\jobs.xml
\end{verbatim}

\phantomsection
\index{configuration:bartender}\addcontentsline{toc}{subsection}{bartender}
\hspace*{0.5cm}
\begin{shaded}
  \uicommand{bartender}
\end{shaded}

Specifies default \textit{Bartender} services. Multiple Bartender URLs should
be separated with a blank space.
These URLs are used by the \texttt{chelonia} command line tool, the Chelonia
FUSE plugin and by the data tool commands \texttt{arccp}, \texttt{arcls}, \texttt{arcrm}, etc..

Example:
\verb#    bartender=http://my.bar.com/tender#

\phantomsection
\index{configuration:proxypath}\addcontentsline{toc}{subsection}{proxypath}
\hspace*{0.5cm}
\begin{shaded}
  \uicommand{proxypath}
\end{shaded}

Specifies a non-standard location of proxy certificate. It is used by
\texttt{arcproxy} or similar tools during proxy generation, and all other tools
during establishing of a secure connection. This attribute
corresponds to the \verb#-P# command line option of \texttt{arcproxy}.

Example:
\verb#    proxypath=/tmp/my-proxy#

\phantomsection
\index{configuration:keypath}\addcontentsline{toc}{subsection}{keypath}
\hspace*{0.5cm}
\begin{shaded}
  \uicommand{keypath}
\end{shaded}

Specifies a non-standard location of user's private key. It is used by
\texttt{arcproxy} or similar tools during proxy generation. This attribute
corresponds to the \verb#-K# command line option of \texttt{arcproxy}.

Example:
\verb#    keypath=/home/username/key.pem#

\phantomsection
\index{configuration:certificatepath}\addcontentsline{toc}{subsection}{certificatepath}
\hspace*{0.5cm}
\begin{shaded}
  \uicommand{certificatepath}
\end{shaded}

Specifies a non-standard location of user's public certificate. It is used by
\texttt{arcproxy} or similar tools during proxy generation. This attribute
corresponds to the \verb#-C# command line option of \texttt{arcproxy}.

Example:
\verb#    certificatepath=/home/username/cert.pem#

\phantomsection
\index{configuration:cacertificatesdirectory}\addcontentsline{toc}{subsection}{cacertificatesdirectory}
\hspace*{0.5cm}
\begin{shaded}
  \uicommand{cacertificatesdirectory}
\end{shaded}

Specifies non-standard location of the directory containing CA-certificates.
This attribute
corresponds to the \verb#-T# command line option of \texttt{arcproxy}.

Example:
\verb#    cacertificatesdirectory=/home/user/cacertificates#

\phantomsection
\index{configuration:cacertificatepath}\addcontentsline{toc}{subsection}{cacertificatepath}
\hspace*{0.5cm}
\begin{shaded}
  \uicommand{cacertificatepath}
\end{shaded}

Specifies an explicit path to the certificate of the CA that issued user's credentials.

Example:
\verb#    cacertificatepath=/home/user/myCA.0#

\phantomsection
\index{configuration:vomsserverpath}\addcontentsline{toc}{subsection}{vomsserverpath}
\hspace*{0.5cm}
\begin{shaded}
  \uicommand{vomsserverpath}
\end{shaded}

Specifies non-standard path to the file which contians list of VOMS services and
associated configuration parameters. This attribute
corresponds to the \verb#-V# command line option of \texttt{arcproxy}.

Example:
\verb#    vomsserverpath=/etc/voms/vomses#

\phantomsection
\index{configuration:username}\addcontentsline{toc}{subsection}{username}
\hspace*{0.5cm}
\begin{shaded}
  \uicommand{username}
\end{shaded}

Sets default username to be used for requesting credentials from Short Lived
Credentials Service. This attribute
corresponds to the \verb#-U# command line option of \texttt{arcslcs}.

Example:
\verb#    username=johndoe#

\phantomsection
\index{configuration:password}\addcontentsline{toc}{subsection}{password}
\hspace*{0.5cm}
\begin{shaded}
  \uicommand{password}
\end{shaded}

Sets default password to be used for requesting credentials from Short Lived
Credentials Service. This attribute
corresponds to the \verb#-P# command line option of \texttt{arcslcs}.

Example:
\verb#    password=secret#

\phantomsection
\index{configuration:keypassword}\addcontentsline{toc}{subsection}{keypassword}
\hspace*{0.5cm}
\begin{shaded}
  \uicommand{keypassword}
\end{shaded}

Sets default password to be used to encode the private key of credentials obtained
from a Short Lived Credentials Service. This attribute
corresponds to the \verb#-K# command line option of \texttt{arcslcs}.

Example:
\verb#    keypassword=secret2#

\phantomsection
\index{configuration:keysize}\addcontentsline{toc}{subsection}{keysize}
\hspace*{0.5cm}
\begin{shaded}
  \uicommand{keysize}
\end{shaded}

Sets size (strength) of the private key of credentials obtained from a Short
Lived Credentials Service. Default value is 1024. This attribute
corresponds to the \verb#-Z# command line option of \texttt{arcslcs}.

Example:
\verb#    keysize=2048#

\phantomsection
\index{configuration:certificatelifetime}\addcontentsline{toc}{subsection}{certificatelifetime}
\hspace*{0.5cm}
\begin{shaded}
  \uicommand{certificatelifetime}
\end{shaded}

Sets lifetime (in hours, starting from current time) of user certificate which
will be obtained from a Short Lived Credentials Service. This attribute
corresponds to the \verb#-L# command line option of \texttt{arcslcs}.

Example:
\verb#    certificatelifetime=12#

\phantomsection
\index{configuration:slcs}\addcontentsline{toc}{subsection}{slcs}
\hspace*{0.5cm}
\begin{shaded}
  \uicommand{slcs}
\end{shaded}

Sets the URL to the Short Lived Certificate Service. This attribute
corresponds to the \verb#-S# command line option of \texttt{arcslcs}.

Example:
\verb#    slcs=https://127.0.0.1:60000/slcs#

\phantomsection
\index{configuration:storedirectory}\addcontentsline{toc}{subsection}{storedirectory}
\hspace*{0.5cm}
\begin{shaded}
  \uicommand{storedirectory}
\end{shaded}

Sets directory which will be used to store credentials obtained from a Short
Lived Credential Servise. This attribute
corresponds to the \verb#-D# command line option of \texttt{arcslcs}.

Example:
\verb#    storedirectory=/home/mycredentials#

\phantomsection
\index{configuration:idpname}\addcontentsline{toc}{subsection}{idpname}
\hspace*{0.5cm}
\begin{shaded}
  \uicommand{idpname}
\end{shaded}

Sets Identity Provider name (Shibboleth) to which user belongs. It is used for
contacting Short Lived Certificate Services. This attribute
corresponds to the \verb#-I# command line option of \texttt{arcslcs}.

Example:
\verb#    idpname=https://idp.testshib.org/idp/shibboleth#


\section{Block \texttt{[alias]}}

Users often prefer to submit jobs to a specific site; since contact URLs (and
especially end-point references) are very long, it is very convenient to replace
them with aliases. Block \texttt{[alias]} simply contains a list of
alias-value pairs.

 Alias substitutions is performed in connection with the \verb#-c# command line
 switch of the ARC clients.

 Aliases can refer to a list of services (separated by a blank space).

 Alias definitions can be recursive. Any alias
 defined in a list that is read before a given list can be used in
 alias definitions in that list. An alias defined in a list can also
 be used in alias definitions later in the same list.

 Examples:
\begin{verbatim}
[alias]

arc0=computing:ARC0:ldap://ce.ng.org:2135/nordugrid-cluster-name=ce.ng.org,Mds-Vo-name=local,o=grid
arc1=computing:ARC1:https://arex.ng.org:60000/arex
cream=computing:CREAM:ldap://cream.glite.org:2170/o=grid
unicore=computing:UNICORE:https://bes.unicore.org:8080/test/services/BESFactory?res=default_bes
crossbrokering=arc0 arc1 cream unicore
\end{verbatim}

\index{configuration:deprecated files}\section{Deprecated configuration files}

ARC configuration file in releases 0.6 and 0.8 has the same name and the same
format. Only one attribute is preserved (\texttt{timeout}); other attributes
unknown to newer ARC versions are ignored.

In ARC $\leq$ 0.5.48, configuration was done via files {\$}\texttt{HOME/.ngrc},
{\$}\texttt{HOME/.nggiislist} and {\$}\texttt{HOME/.ngalias}.

The main configuration file \textbf{{\$}\texttt{HOME/.ngrc}} could contain
user's default settings for the debug level, the information system
query timeout and the download directory used by \texttt{ngget}. A
sample file could be the following:
\begin{verbatim}
# Sample .ngrc file
# Comments starts with #
NGDEBUG=1
NGTIMEOUT=60
NGDOWNLOAD=/tmp
\end{verbatim}

If the environment variables NGDEBUG, NGTIMEOUT or NGDOWNLOAD were
defined, these took precedence over the values defined in this
configuration. Any command line options override the defaults.

The file \textbf{{\$}\texttt{HOME/.nggiislist}} was used to keep the
list of default GIIS server URLs, one line per GIIS (see \texttt{giis}
attribute description above).

The file \textbf{{\$}\texttt{HOME/.ngalias}} was used to keep the
list of site aliases, one line per alias (see \texttt{alias}
attribute description above).


\bibliography{grid}
\printindex

\end{document}
