Default behaviour of an ARC client can be configured by specifying
alternative values for some parameters in the client configuration
file. The file is called \texttt{client.conf} and is located in
directory \texttt{.arc} in user's home area:
\begin{shaded}
 {\$}HOME/.arc/client.conf
\end{shaded}
If this file is not present or does not contain the relevant
configuration information, the global configuration files (if exist)
or default values are used instead. Some client tools may be able to create
the default \texttt{{\$}HOME/.arc/client.conf}, if it does not exist.

The ARC configuration file consists of several configuration blocks.
Each configuration block is identified by a keyword and contains
configuration options for a specific part of the ARC middleware. 

The configuration file is written in a plain text format known as INI.
Configuration blocks start with identifying keywords inside square brackets.
Typically, first comes a common block: \verb#[common]#. Thereafter follows one
or more attribute-value pairs written one on each line in the following
format:

\begin{framed}
\begin{verbatim}
[common]
attribute1=value1
attribute2=value2
attribute3=value3 value4
# comment line 1
# comment line 2
attribute2=value5
...
\end{verbatim}
\end{framed}

%For multi-valued attributes, several elements or attribute-value pairs
%have to be specified -- one per each value.

The following blocks and respective attributes are recognized.

\section{Block \texttt{common}}

\phantomsection
\index{configuration:defaultservices}\addcontentsline{toc}{subsection}{defaultservices}
\hspace*{0.5cm}
\begin{shaded}
  \uicommand{defaultservices}
\end{shaded}
\textbf{This attribute is multi-valued.}

This attribute is used to specify default services to be used. Defining such in
the user configuration file will override the default services set in the system
configuration.

The value of this attribute should follow the format:
\begin{verbatim}
  service_type:flavour:service_url
\end{verbatim}

where \texttt{service\_type} is type of service (e.g. \texttt{computing} or
\texttt{index}), \texttt{flavour} specifies type of middleware plugin to use
when contacting the service (e.g. ARC0, ARC1, CREAM, UNICORE, etc.) and
\texttt{service\_url} is the URL used to contact the service.

Example:
\begin{itemize}
\item{plain text:}\\ 
\verb#defaultservices=index:ARC0:ldap://index1.nordugrid.org:2135/Mds-Vo-name=NorduGrid,o=grid index:ARC1:https://knowarc2.grid.niif.hu:50000/isis computing:ARC1:https://knowar c1.grid.niif.hu:60000/arex computing:CREAM:ldap://cream.grid.upjs.sk:2170/o=grid computing:UNICORE:https://testbed5.grid.upjs.sk:8080/KnowARC-testbed/services/BESFactory?res=default_bes_factory#
\end{itemize}

% \phantomsection
% \index{configuration:giis}\addcontentsline{toc}{subsection}{giis}
% \hspace*{0.5cm}
% \begin{shaded}
%   \uicommand{giis}
% \end{shaded}
% \textbf{This attribute is multi-valued.}
% 
% Configures which GIISes (site indices) to use to discover computing
% and storage resources. The default is to use the four top level
% GIISes:
% 
% \begin{verbatim}
%   ldap://index1.nordugrid.org:2135/O=Grid/Mds-Vo-name=NorduGrid
%   ldap://index2.nordugrid.org:2135/O=Grid/Mds-Vo-name=NorduGrid
%   ldap://index3.nordugrid.org:2135/O=Grid/Mds-Vo-name=NorduGrid
%   ldap://index4.nordugrid.org:2135/O=Grid/Mds-Vo-name=NorduGrid
% \end{verbatim}
% 
% Examples:
% \begin{itemize}
% \item{plain text:}\\ 
% \verb#    giis="ldap://atlasgiis.nbi.dk:2135/O=Grid/Mds-Vo-name=Atlas"#\\
% \verb#    giis="ldap://index1.nordugrid.org:2135/O=Grid/Mds-Vo-name=NorduGrid"#\\
% \verb#    giis="ldap://grid.tsl.uu.se:2135/O=Grid/Mds-Vo-name=local"#
% \item{XML:}\\ 
% \verb#    <giis>ldap://odin.switch.ch:2135/O=Grid/Mds-Vo-name=Switzerland</giis>#\\
% \verb#    <giis>ldap://index1.nordugrid.org:2135/O=Grid/Mds-Vo-name=NorduGrid</giis>#\\
% \verb#    <giis>ldap://grid.tsl.uu.se:2135/O=Grid/Mds-Vo-name=local</giis>#
% \end{itemize}
% 
% \begin{framed}
% Note that LDAP URLs containing "Mds-Vo-name=local" refer not to
% GIISes, but to individual sites. This feature can be used to add
% "hidden" sites that do not register to any GIIS, or to list explicitly
% preferred submission sites.
% \end{framed}
% 
% \phantomsection
% \index{configuration:debug}\addcontentsline{toc}{subsection}{debug}
% \hspace*{0.5cm}
% \begin{shaded}
%   \uicommand{debug}
% \end{shaded}
% Default debug level to use for the ARC clients. Corresponds to the \verb#-d#
% command line switch of the clients. Default value is 0, possible range
% is from -3 to 3.
% 
% Examples:
% \begin{itemize}
% \item{plain text:}\\ \verb#    debug="2"#
% \item{XML:}\\ \verb#    <debug>2</debug>#
% \end{itemize}
% 
% 
% \phantomsection
% \index{configuration:timeout}\addcontentsline{toc}{subsection}{timeout}
% \hspace*{0.5cm}
% \begin{shaded}
%   \uicommand{timeout}
% \end{shaded}
% Timeout to use for interaction with LDAP servers, gridftp servers,
% etc. If a server, during e.g. job submission, does not answer in the
% specified number of seconds, the connection is timed out and
% closed. Default value is 20 seconds.
% 
% Examples:
% \begin{itemize}
% \item{plain text:}\\ \verb#    timeout="20"#
% \item{XML:}\\ \verb#    <timeout>10</timeout>#
% \end{itemize}
% 
% 
% \phantomsection
%
\index{configuration:downloaddir}\addcontentsline{toc}{subsection}{downloaddir}
% \hspace*{0.5cm}
% \begin{shaded}
%   \uicommand{downloaddir}
% \end{shaded}
% Default download directory to download files to when retrieving output
% files from jobs using \texttt{ngget}.  Default is the current working
% directory.
% 
% Examples:
% \begin{itemize}
% \item{plain text:}\\ \verb#    downloaddir="/home/johndoe/arc-downloads"#
% \item{XML:}\\ \verb#    <downloaddir>/tmp/johndoe</downloaddir>#
% \end{itemize}
% 
% 
% \phantomsection
% \index{configuration:alias}\addcontentsline{toc}{subsection}{alias}
% \hspace*{0.5cm}
% \begin{shaded}
%   \uicommand{alias}
% \end{shaded}
% \textbf{This attribute is multi-valued.}
% 
% Alias substitutions to perform in connection with the \verb#-c# command line
% switch of the ARC clients. Alias definitions are recursive. Any alias
% defined in a block that is read before a given block can be used in
% alias definitions in that block. An alias defined in a block can also
% be used in alias definitions later in the same block.
% 
% Examples:
% \begin{itemize}
% \item{plain text:}\\ \verb#    alias="host1=somehost.nbi.dk"#\\
% \verb#    alias="host2=otherhost.uu.se"#\\
% \verb#    alias="myhosts=host1 host2"#
% \item{XML:}\\ \verb#    <alias>host1=somehost.nbi.dk</alias>#\\
% \verb#    <alias>host2=otherhost.uu.se</alias>#\\
% \verb#    <alias>myhosts=host1 host2</alias>#
% \end{itemize}
% 
% With the example above, \verb#ngsub -c myhosts# will resolve to \verb#ngsub -c somehost.nbi.dk -c otherhost.uu.se#.
% 
% \phantomsection
% \index{configuration:broker}\addcontentsline{toc}{subsection}{broker}
% \hspace*{0.5cm}
% \begin{shaded}
%   \uicommand{broker}
% \end{shaded}
% Configures which brokering algorithm to use during job submission. The
% default one is the \texttt{FastestCpus} broker that chooses, among the
% possible targets, the target with the fastest CPUs. Another
% possibility is, for example, the
% \texttt{RandomSort} broker that chooses the target randomly among the targets
% surviving the job description matchmaking.
% 
% Examples:
% \begin{itemize}
% \item{plain text:}\\ \verb#    broker="RandomSort"#
% \item{XML:}\\ \verb#    <broker>RandomSort</broker>#
% \end{itemize}


\index{configuration:deprecated files}\section{Deprecated configuration files}

ARC configuration file in releases 0.6 and 0.8 has the same name and the same
format; attributes unknown to newer ARC versions are ignored.

In ARC $\leq$ 0.5.48, configuration was done via files {\$}\texttt{HOME/.ngrc},
{\$}\texttt{HOME/.nggiislist} and {\$}\texttt{HOME/.ngalias}.

The main configuration file \textbf{{\$}\texttt{HOME/.ngrc}} could contain
user's default settings for the debug level, the information system
query timeout and the download directory used by \texttt{ngget}. A
sample file could be the following:
\begin{verbatim}
# Sample .ngrc file
# Comments starts with #
NGDEBUG=1
NGTIMEOUT=60
NGDOWNLOAD=/tmp
\end{verbatim}

If the environment variables NGDEBUG, NGTIMEOUT or NGDOWNLOAD were
defined, these took precedence over the values defined in this
configuration. Any command line options override the defaults.

The file \textbf{{\$}\texttt{HOME/.nggiislist}} was used to keep the
list of default GIIS server URLs, one line per GIIS (see \texttt{giis}
attribute description above). 

The file \textbf{{\$}\texttt{HOME/.ngalias}} was used to keep the
list of site aliases, one line per alias (see \texttt{alias}
attribute description above). 
