Default behaviour of an ARC client can be configured by specifying
alternative values for some parameters in the client configuration
file. The file is called \texttt{client.conf} and is located in
directory \texttt{.arc} in user's home area:
\begin{shaded}
 {\$}HOME/.arc/client.conf
\end{shaded}
If this file is not present or does not contain the relevant
configuration information, the global configuration files (if exist)
or default values are used instead. Some client tools may be able to create
the default \texttt{{\$}HOME/.arc/client.conf}, if it does not exist.

The ARC configuration file consists of several configuration blocks.
Each configuration block is identified by a keyword and contains
configuration options for a specific part of the ARC middleware.

The configuration file is written in a plain text format known as INI.
Configuration blocks start with identifying keywords inside square brackets.
Typically, first comes a common block: \verb#[common]#. Thereafter follows one
or more attribute-value pairs written one on each line in the following
format:

\begin{framed}
\begin{verbatim}
[common]
attribute1=value1
attribute2=value2
attribute3=value3 value4
# comment line 1
# comment line 2
...
\end{verbatim}
\end{framed}

%For multi-valued attributes, several elements or attribute-value pairs
%have to be specified -- one per each value.

Most attributes have counterpart command line options. Command line options
always overwrite configuration attributes.

Two blocks are currently recognized, \texttt{[common]} and
\texttt{[alias]}. Following sections describe supported attributes per block.

\section{Block \texttt{[common]}}

\phantomsection
\index{configuration:defaultservices}\addcontentsline{toc}{subsection}{defaultservices}
\hspace*{0.5cm}
\begin{shaded}
  \uicommand{defaultservices}
\end{shaded}
\textbf{This attribute is multi-valued.}

This attribute is used to specify default services to be used. Defining such in
the user configuration file will override the default services set in the system
configuration.

The value of this attribute should follow the format:
\begin{verbatim}
  service_type:flavour:service_url
\end{verbatim}

where \texttt{service\_type} is type of service (e.g. \texttt{computing} or
\texttt{index}), \texttt{flavour} specifies type of middleware plugin to use
when contacting the service (e.g. ARC0, ARC1, CREAM, UNICORE, etc.) and
\texttt{service\_url} is the URL used to contact the service. Several services
can be listed, separated with a blank space (no line breaks allowed).

Example:
\begin{verbatim*}
defaultservices=index:ARC0:ldap://index1.ng.org:2135/Mds-Vo-name=testvo,o=grid
 index:ARC1:https://index2.ng.org:50000/isis
 computing:ARC1:https://ce.arc.org:60000/arex
 computing:CREAM:ldap://ce.glite.org:2170/o=grid
 computing:UNICORE:https://ce.unicore.org:8080/test/services/BESFactory?res=default_bes_factory
\end{verbatim*}

\phantomsection
\index{configuration:rejectservices}\addcontentsline{toc}{subsection}{rejectservices}
\hspace*{0.5cm}
\begin{shaded}
  \uicommand{rejectservices}
\end{shaded}
\textbf{This attribute is multi-valued.}

This attribute can be used to indicate that a certain service should be
rejected (``blacklisted''). Several services can be listed, separated with a
blank space (no line breaks allowed).

Example:
\verb#    rejectservices=computing:ARC1:https://bad.service.org/arex#


 \phantomsection
 \index{configuration:verbosity}\addcontentsline{toc}{subsection}{verbosity}
 \hspace*{0.5cm}
 \begin{shaded}
   \uicommand{verbosity}
 \end{shaded}
 Default verbosity (debug) level to use for the ARC clients. Corresponds to the
\verb#-d# command line option of the clients. Default value is \texttt{WARNING},
possible values are \texttt{FATAL}, \texttt{ERROR}, \texttt{WARNING},
\texttt{INFO}, \texttt{VERBOSE} or \texttt{DEBUG}.

Example:
\verb#    verbosity=INFO#

\phantomsection
\index{configuration:timeout}\addcontentsline{toc}{subsection}{timeout}
\hspace*{0.5cm}
\begin{shaded}
  \uicommand{timeout}
\end{shaded}
Sets the period of time the client should wait for a service (information,
computing, storage etc) to respond when communicating with it. The period
should be given in seconds. Default value is 20 seconds. This attribute
corresponds to the \verb#-t# command line option.

Example:
\verb#    timeout=10#

\phantomsection
\index{configuration:brokername}\addcontentsline{toc}{subsection}{brokername}
\hspace*{0.5cm}
\begin{shaded}
  \uicommand{brokername}
\end{shaded}

Configures which brokering algorithm to use during job submission. This attribute
corresponds to the \verb#-b# command line option. The
default one is the \texttt{Random} broker that chooses targets randomly.
Another possibility is, for example, the \texttt{FastestQueue} broker that
chooses the target with the shortest estimated queue waiting time. For an
overview of brokers, please refer to Section~\ref{sec:arcsub}.

 Example:
\verb#    brokername=Data#

\phantomsection
\index{configuration:brokerarguments}\addcontentsline{toc}{subsection}{brokerarguments}
\hspace*{0.5cm}
\begin{shaded}
  \uicommand{brokerarguments}
\end{shaded}

This attribute is used in case a broker comes with arguments. This corresponds to the
parameter that follows column in the \verb#-b# command line option.

Example:
\verb#    brokerarguments=cow#

\phantomsection
\index{configuration:joblist}\addcontentsline{toc}{subsection}{joblist}
\hspace*{0.5cm}
\begin{shaded}
  \uicommand{joblist}
\end{shaded}

Path to the job list file. This file will be used by commands such as \texttt{arcsub}, \texttt{arcstat},
\texttt{arcsync} etc. to read and write information about jobs. This attribute
corresponds to the \verb#-j# command line option. The default
location of the file is in the {\$}HOME/.arc/client.conf directory with the
name \texttt{jobs.xml}.

Example:
\begin{verbatim}
    joblist=/home/user/run/jobs.xml
    joblist=C:\\run\jobs.xml
\end{verbatim}

\phantomsection
\index{configuration:bartender}\addcontentsline{toc}{subsection}{bartender}
\hspace*{0.5cm}
\begin{shaded}
  \uicommand{bartender}
\end{shaded}

Specifies default \textit{Bartender} services. Multiple Bartender URLs should
be separated with a blank space.
These URLs are used by the \texttt{chelonia} command line tool, the Chelonia
FUSE plugin and by the data tool commands \texttt{arccp}, \texttt{arcls}, \texttt{arcrm}, etc..

Example:
\verb#    bartender=http://my.bar.com/tender#

\phantomsection
\index{configuration:proxypath}\addcontentsline{toc}{subsection}{proxypath}
\hspace*{0.5cm}
\begin{shaded}
  \uicommand{proxypath}
\end{shaded}

Specifies a non-standard location of proxy certificate. It is used by
\texttt{arcproxy} or similar tools during proxy generation, and all other tools
during establishing of a secure connection. This attribute
corresponds to the \verb#-P# command line option of \texttt{arcproxy}.

Example:
\verb#    proxypath=/tmp/my-proxy#

\phantomsection
\index{configuration:keypath}\addcontentsline{toc}{subsection}{keypath}
\hspace*{0.5cm}
\begin{shaded}
  \uicommand{keypath}
\end{shaded}

Specifies a non-standard location of user's private key. It is used by
\texttt{arcproxy} or similar tools during proxy generation. This attribute
corresponds to the \verb#-K# command line option of \texttt{arcproxy}.

Example:
\verb#    keypath=/home/username/key.pem#

\phantomsection
\index{configuration:certificatepath}\addcontentsline{toc}{subsection}{certificatepath}
\hspace*{0.5cm}
\begin{shaded}
  \uicommand{certificatepath}
\end{shaded}

Specifies a non-standard location of user's public certificate. It is used by
\texttt{arcproxy} or similar tools during proxy generation. This attribute
corresponds to the \verb#-C# command line option of \texttt{arcproxy}.

Example:
\verb#    certificatepath=/home/username/cert.pem#

\phantomsection
\index{configuration:cacertificatesdirectory}\addcontentsline{toc}{subsection}{cacertificatesdirectory}
\hspace*{0.5cm}
\begin{shaded}
  \uicommand{cacertificatesdirectory}
\end{shaded}

Specifies non-standard location of the directory containing CA-certificates.
This attribute
corresponds to the \verb#-T# command line option of \texttt{arcproxy}.

Example:
\verb#    cacertificatesdirectory=/home/user/cacertificates#

\phantomsection
\index{configuration:cacertificatepath}\addcontentsline{toc}{subsection}{cacertificatepath}
\hspace*{0.5cm}
\begin{shaded}
  \uicommand{cacertificatepath}
\end{shaded}

Specifies an explicit path to the certificate of the CA that issued user's credentials.

Example:
\verb#    cacertificatepath=/home/user/myCA.0#

\phantomsection
\index{configuration:vomsserverpath}\addcontentsline{toc}{subsection}{vomsserverpath}
\hspace*{0.5cm}
\begin{shaded}
  \uicommand{vomsserverpath}
\end{shaded}

Specifies non-standard path to the file which contians list of VOMS services and
associated configuration parameters. This attribute
corresponds to the \verb#-V# command line option of \texttt{arcproxy}.

Example:
\verb#    vomsserverpath=/etc/voms/vomses#

\phantomsection
\index{configuration:username}\addcontentsline{toc}{subsection}{username}
\hspace*{0.5cm}
\begin{shaded}
  \uicommand{username}
\end{shaded}

Sets default username to be used for requesting credentials from Short Lived
Credentials Service. This attribute
corresponds to the \verb#-U# command line option of \texttt{arcslcs}.

Example:
\verb#    username=johndoe#

\phantomsection
\index{configuration:password}\addcontentsline{toc}{subsection}{password}
\hspace*{0.5cm}
\begin{shaded}
  \uicommand{password}
\end{shaded}

Sets default password to be used for requesting credentials from Short Lived
Credentials Service. This attribute
corresponds to the \verb#-P# command line option of \texttt{arcslcs}.

Example:
\verb#    password=secret#

\phantomsection
\index{configuration:keypassword}\addcontentsline{toc}{subsection}{keypassword}
\hspace*{0.5cm}
\begin{shaded}
  \uicommand{keypassword}
\end{shaded}

Sets default password to be used to encode the private key of credentials obtained
from a Short Lived Credentials Service. This attribute
corresponds to the \verb#-K# command line option of \texttt{arcslcs}.

Example:
\verb#    keypassword=secret2#

\phantomsection
\index{configuration:keysize}\addcontentsline{toc}{subsection}{keysize}
\hspace*{0.5cm}
\begin{shaded}
  \uicommand{keysize}
\end{shaded}

Sets size (strength) of the private key of credentials obtained from a Short
Lived Credentials Service. Default value is 1024. This attribute
corresponds to the \verb#-Z# command line option of \texttt{arcslcs}.

Example:
\verb#    keysize=2048#

\phantomsection
\index{configuration:certificatelifetime}\addcontentsline{toc}{subsection}{certificatelifetime}
\hspace*{0.5cm}
\begin{shaded}
  \uicommand{certificatelifetime}
\end{shaded}

Sets lifetime (in hours, starting from current time) of user certificate which
will be obtained from a Short Lived Credentials Service. This attribute
corresponds to the \verb#-L# command line option of \texttt{arcslcs}.

Example:
\verb#    certificatelifetime=12#

\phantomsection
\index{configuration:slcs}\addcontentsline{toc}{subsection}{slcs}
\hspace*{0.5cm}
\begin{shaded}
  \uicommand{slcs}
\end{shaded}

Sets the URL to the Short Lived Certificate Service. This attribute
corresponds to the \verb#-S# command line option of \texttt{arcslcs}.

Example:
\verb#    slcs=https://127.0.0.1:60000/slcs#

\phantomsection
\index{configuration:storedirectory}\addcontentsline{toc}{subsection}{storedirectory}
\hspace*{0.5cm}
\begin{shaded}
  \uicommand{storedirectory}
\end{shaded}

Sets directory which will be used to store credentials obtained from a Short
Lived Credential Servise. This attribute
corresponds to the \verb#-D# command line option of \texttt{arcslcs}.

Example:
\verb#    storedirectory=/home/mycredentials#

\phantomsection
\index{configuration:idpname}\addcontentsline{toc}{subsection}{idpname}
\hspace*{0.5cm}
\begin{shaded}
  \uicommand{idpname}
\end{shaded}

Sets Identity Provider name (Shibboleth) to which user belongs. It is used for
contacting Short Lived Certificate Services. This attribute
corresponds to the \verb#-I# command line option of \texttt{arcslcs}.

Example:
\verb#    idpname=https://idp.testshib.org/idp/shibboleth#


\section{Block \texttt{[alias]}}

Users often prefer to submit jobs to a specific site; since contact URLs (and
especially end-point references) are very long, it is very convenient to replace
them with aliases. Block \texttt{[alias]} simply contains a list of
alias-value pairs.

 Alias substitutions is performed in connection with the \verb#-c# command line
 switch of the ARC clients.

 Aliases can refer to a list of services (separated by a blank space).

 Alias definitions can be recursive. Any alias
 defined in a list that is read before a given list can be used in
 alias definitions in that list. An alias defined in a list can also
 be used in alias definitions later in the same list.

 Examples:
\begin{verbatim}
[alias]

arc0=computing:ARC0:ldap://ce.ng.org:2135/nordugrid-cluster-name=ce.ng.org,Mds-Vo-name=local,o=grid
arc1=computing:ARC1:https://arex.ng.org:60000/arex
cream=computing:CREAM:ldap://cream.glite.org:2170/o=grid
unicore=computing:UNICORE:https://bes.unicore.org:8080/test/services/BESFactory?res=default_bes
crossbrokering=arc0 arc1 cream unicore
\end{verbatim}

\index{configuration:deprecated files}\section{Deprecated configuration files}

ARC configuration file in releases 0.6 and 0.8 has the same name and the same
format. Only one attribute is preserved (\texttt{timeout}); other attributes
unknown to newer ARC versions are ignored.

In ARC $\leq$ 0.5.48, configuration was done via files {\$}\texttt{HOME/.ngrc},
{\$}\texttt{HOME/.nggiislist} and {\$}\texttt{HOME/.ngalias}.

The main configuration file \textbf{{\$}\texttt{HOME/.ngrc}} could contain
user's default settings for the debug level, the information system
query timeout and the download directory used by \texttt{ngget}. A
sample file could be the following:
\begin{verbatim}
# Sample .ngrc file
# Comments starts with #
NGDEBUG=1
NGTIMEOUT=60
NGDOWNLOAD=/tmp
\end{verbatim}

If the environment variables NGDEBUG, NGTIMEOUT or NGDOWNLOAD were
defined, these took precedence over the values defined in this
configuration. Any command line options override the defaults.

The file \textbf{{\$}\texttt{HOME/.nggiislist}} was used to keep the
list of default GIIS server URLs, one line per GIIS (see \texttt{giis}
attribute description above).

The file \textbf{{\$}\texttt{HOME/.ngalias}} was used to keep the
list of site aliases, one line per alias (see \texttt{alias}
attribute description above).
