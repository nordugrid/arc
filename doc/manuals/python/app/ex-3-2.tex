\section{Examples of \ref{aclient}}
\subsection{Creating a client and calling the echo service (XMLTree)}
(Example \ref{clientex1})
\label{cclientex1}
\begin{verbatim}
import arc
import arcom
# Import the Client class
Client = arcom.import_class_from_string('arcom.client.Client')
# Import the XMLTree class
XMLTree = arcom.import_class_from_string('arcom.xmltree.XMLTree')
# Create namespace - it will be used for the message sent
ns = arc.NS({'echo':'urn:echo'})
# Create client
c = Client('http://your.server.example.com:50000/Echo',ns,print_xml=True)
# Create message
msg = XMLTree(from_tree = ('echo:echo',[('echo:say', 'Hello, World!')]))
# Let the client do what it is meant for
# Note that we created the client with print_xml=True
# so both request and response will be displayed in an easy-to-read form
c.call(msg)
\end{verbatim}

\subsection{Creating a client and calling the echo service (SOAP)}
(Example \ref{clientex2})
\label{cclientex2}
\begin{verbatim}
import arc
import arcom
# Import the Client class
Client = arcom.import_class_from_string('arcom.client.Client')
# Create namespace - it will be used for the message sent
ns = arc.NS({'echo':'urn:echo'})
# Create client
c = Client('http://arctest.ki.iif.hu:50000/Echo',ns,print_xml=True)
# Create SOAP Payload
pl = arc.PayloadSOAP(ns)
# Create message structure and set content
pl.NewChild('echo:echo',ns).NewChild('echo:say',ns).Set('Hello, World!')
# Let client do the call; see response
c.call_raw(pl)
\end{verbatim}

