\chapter{Introduction}

The {\em Janitor} is a service for the automated installation of runtime environments for grid computing elements.
Its command line interface allows for a direct interaction with site administrators. However, the main stimulus
for its development was the idea integrate such a service with the regular handling of compute jobs. For ARC this
is performed by the \AREX module.

From the programmer's view, the Janitor is mostly a Perl script and the routines within \AREX to invoke it. The
site administrator will also associate with it also the Catalog files, that describe the availability of runtime
environments, and the repository of installable runtime envionments themselves, which are regular tar archives obeying
to particular structure and reside in a separate folder. In order to minimise the latency for the invocation of
the Perl script and the associated parsing of files, a Janitor web service was developed, which still is a Perl
script.

\section{Motivation}

A major motivation for grid projects is to stimulate new communities to adopt computational grids for their
causes. From the current grid user's viewpoint, the admission of users of a very different scientific disciplines
or compute skills will impose difficulties in the communication between site maintainers. One will not even understand
the respective other side's research aims. Hence, the proper installation of non-standard software (read Runtime
Environments) is not guaranteed.

A core problem remains to distribute a locally working solution, the Know-How, quickly across all contributing
sites, i. e., without manual interference. Every scientific discipline has its respective own set of
technologies for the distribution of work load. For instance, research in bioinformatics requires access to so many
different tools and databases, that few sites, if any, install them all. Instead, the use of web services became
a commodity, with all the problems with respect to bottlenecks and restrictions of repeated access.
The EU project KnowARC\footnote{\href{http://www.knowarc.eu}{http://www.knowarc.eu}},  amongst other challenges,
with the here presented work extends the NorduGrid?s
Advanced Research Connector (ARC) grid middleware~\cite{ELLERT_2007} towards an the
automated installation of software packages.

An automation of the software installation, referred to as dynamic Runtime Environments, seems the only
approach to use the computational grid to its full potential. Components of workflows shall be spawned
as jobs in a computational grid using dynamic Runtime Environments rather than as shared web services.
The grid introduces an extra level of parallelism that web services cannot provide. The required short
response times and the heterogeneous education of site-administrators on a grid demand an automatism for
the installation of software and databases without manual interference~\cite{BAYER_2007}.

\section{Overview}

This document will start with a chapter of how to set up janitor locally. The following chapter will give an instruction on how to 
use Janitor with \AREX and/or without \AREX. Afterwards, in the third chapter, the maintenance of the program will be presented, 
which is basically covering the preparation of runtime environments. Deeper knowledge about the design of the Janitor will 
be given by the subsequent forth chapter. In the fifth chapter, an outlook on ongoing or future developments will presented.


% BASIC CONCEPT

% * Installation

% * Usage

% * Maintainance

% * Programming concept
 
% * Future work

% * APPENDIX
