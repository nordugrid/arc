\documentclass{article}
%
%\usepackage{amsmath}
%\usepackage{amssymb}
%\usepackage{graphics}
%
\usepackage{graphicx}                              %for PNG images
\usepackage{color}                                 %for defining custom colors
\usepackage{framed}                                %for shaded and framed paragraphs
\usepackage{makeidx}                               %for index page
\usepackage{geometry}                              %for defining page size
\usepackage[linkbordercolor={0 0.8 0.8}]{hyperref} %for \url tag
\usepackage{doc}                                   %for pfill used by gist index style
%
\geometry{verbose,a4paper,tmargin=2.5cm,bmargin=2.5cm,lmargin=2.5cm,rmargin=2cm}
\makeindex
\hypersetup{
  pdfauthor = {Henrik Thostrup Jensen},
  pdftitle = {ARC Usage Record Logger},
  pdfsubject = {Setup Guide},
  pdfkeywords = {Grid,ARC,usage},
  pdfcreator = {PDFLaTeX with hyperref package},
  pdfproducer = {PDFLaTeX}
}
%
\usepackage[numbers]{natbib}
\bibliographystyle{plainnat}
%
\def\efill{\hfill\nopagebreak}%
\hyphenation{Nordu-Grid}
\setlength{\parindent}{0cm}
\setlength{\FrameRule}{1pt}
\setlength{\FrameSep}{8pt}
\addtolength{\parskip}{5pt}
\renewcommand{\thefootnote}{\fnsymbol{footnote}}
\renewcommand{\arraystretch}{1.3}
\newcommand{\dothis}{\colorbox{shadecolor}}
\definecolor{shadecolor}{rgb}{1,1,0.6}
\definecolor{salmon}{rgb}{1,0.9,1}
\definecolor{cyan}{rgb}{0,1,1}

\begin{document}
  \def\today{\number\day/\number\month/\number\year}
  
  \begin{titlepage}
    
    \begin{tabular}{rl}
      \resizebox*{3cm}{!}{\includegraphics{ng-logo.png}}
      &\parbox[b]{2cm}{\textbf \it {\hspace*{-1.5cm}NORDUGRID\vspace*{0.5cm}}}
    \end{tabular}
    
    \hrulefill
    
    {\raggedleft NORDUGRID-MANUAL-16\par}
    
    {\raggedleft \today\par}
    
    \vspace*{2cm}
    
%%%%---- The title ----
    {\centering \textsc{\Large ARC Usage Record Logger}\Large \par}
    
    \vspace*{0.5cm}
    
%%%%---- A subtitle, if necessary ----
    {\centering \textit{\large Setup Guide}\large \par}
    
    \vspace*{2cm}
			
%%%%---- A list of authors ----
    {\centering \large Henrik Thostrup Jensen \texttt{<htj@ndgf.org>} \large \par}
    
    \vspace*{1.5cm}
    
%%%%---- An abstract ----
%\begin{abstract}
%\end{abstract}

\end{titlepage}
\thispagestyle{empty} $ $
\newpage
$\ $

\section{Setup}

From version 0.8.1 ARC includes a usage record generator and registrator to
SGAS. This document describe how to configure these.

\subsection{Requirements}

\begin{itemize}
\item Python 2.4 or later
\item Twisted Core and Web (http://twistedmatrix.com/)
\item PyOpenSSL (https://launchpad.net/pyopenssl)
\item ElementTree (http://effbot.org/zone/element-index.htm - only needed with Python 2.4)
\end{itemize}

\noindent
Debian/Ubuntu package names:

\texttt{python-twisted-core, python-twisted-web, python-openssl} \\
\texttt{python-elementtree} (only needed with Python 2.4)

CentOS and similar:

\texttt{python-twisted-core python-twisted-web pyOpenSSL} \\
\texttt{python-elementtree} (only needed with Python 2.4)

Other Linux distributions: You are on your own :-)


\subsection{arc.conf}

To invoke the UR generator, an authplugin line must be set in the
\texttt{[grid-manager]} section:

\begin{verbatim}
[grid-manager]
authplugin="FINISHED timeout=10,onfailure=pass /opt/nordugrid/libexec//arc-ur-logger %C %I %S %U"
\end{verbatim}

Change the path if you've installed ARC elsewhere than /opt/nordugrid. You'll
need to restart the grid-manager for this to take effect (but wait until you
finished reading this document).

The plugin will log to the file: /var/log/arc-ur-logger.log. This file will not
appear until a job has finished.

Additional logger configuration happens in a seperate logger section. Example:

\begin{verbatim}
[logger]
log_dir=/var/spool/nordugrid/usagerecords/
log_all="https://sgas.ndgf.org:6143/sgas"
log_vo="bio.ndgf.org https://biosgas.ndgf.org:6143/sgas"
ur_lifetime=30
\end{verbatim}

The \texttt{log\_dir} option will set the top directory for the generated usage
records. The option will default to \texttt{/var/spool/nordugrid/usagerecords/}
and can be left out. We suggest you leave out this option, unless you have a
reason not to.

The \texttt{log\_all} and \texttt{log\_vo} options configure where to usage
records are registered. All usage records will be registered to the URLs
specified with the \texttt{log\_all} option. It is possible specifiy multiple
URLs be having them space seperated, e.g.:

\begin{verbatim}
log_all="https://sgas.ndgf.org:6143/sgas https://sgas.grid.dk:6143/sgas"
\end{verbatim}

The \texttt{log\_vo} option makes it possible to only register usage records run
with certain VO users to given URL. The above example will register all usage
records where the VO information includes the VO bio.ndgf.org to the url
\texttt{https://biosgas.ndgf.org:6143/sgas}. Is is possible to have multiple
entries, by seperating entries with comma, e.g.,

\begin{verbatim}
log_vo="vo1 url1, vo2 url2"
\end{verbatim}

It is NOT possible to have multiple \texttt{log\_all} or \texttt{log\_vo} lines.


The \texttt{ur\_lifetime} option specifies how many days usage records are kept
after being archived. The default is 30, and the option can be left out.


It is possible to set the logfiles for both the logger and the registrant.
Furthermore it is possible to specify logging level for the logger (not
possible or needed for the registrant, as it is not that verbose). Examples:

\begin{verbatim}
urlogger_logfile="/tmp/arc-ur-logger.log"
urlogger_loglevel="info"
registrant_logfile="/tmp/arc-ur-registrant.log"
\end{verbatim}

By default the ur logger will write to \texttt{/var/log/arc-ur-logger.log} and
the registrant to \\ \texttt{/var/log/arc-ur-registration.log}. Valid options
for logger log level are: \texttt{debug}, \texttt{info}, \texttt{warning}. The
default is \texttt{info} which will write one line log per job, assuming
everything goes as planned.



\subsection{cron}

To register the usage records and registrant should be invoked regularly be
CRON. We suggest every hour. The crontab entry would typically look like this:

\begin{verbatim}
0 * * * * /opt/nordugrid/libexec/arc-ur-registrant
\end{verbatim}

To ensure that the registrant is working, you can run the script from the
command line first. Note that the script will still write its log to to
/var/log/arc-ur-registration.log. By running the script with -s its ouput will
be directed to stdout.



\end{document}


