  \hspace*{0.5cm}\phantomsection
  \index{XRSL:attribute:executable}\addcontentsline{toc}{subsection}{executable}
  \begin{shaded}
    \xrsl{executable}
  \end{shaded}
  \versions{ARC 0.6, ARC 0.8}
  \begin{tabular}{lp{13cm}}
    Unique:&yes\\
    Operators:&\verb#=#\\
    User input:&\verb#(executable=<string>)#\\
    GM input:& for ARC $\leq$ 0.5.25:  \verb#(executable="/bin/echo")#\\
    Example:&\verb#(executable="local_to_job.exe")#\\
  \end{tabular}

  The executable to be submitted as a main task to a Local Resource
  Management System (LRMS).

  \begin{tabular}{llp{10cm}}
    \hspace*{1cm}&\texttt{string}&file name (including path), local to the computing
    element (CE)\\
  \end{tabular}

  Executable is a file that has to be executed as the main process
  of the task. It could be either a pre-compiled binary, or a
  script. Users may transfer their own executables, or use the ones
  known to be already installed on the remote system (CE).

  If an executable has to be transferred to the destination site (CE) from some source, it
  has to be specified in the \texttt{inputFiles} list. If it is not specified in \texttt{inputFiles},
  the source is expected to be local to the user (client) and
  will be added as such to the \texttt{inputFiles} list by the ARC Client.

  If the file name starts with a leading slash ("\verb#/#"), it is considered
  to be \textbf{the full path to the executable at the destination site (CE)}; otherwise
  the location of the file is \textbf{relative} to the session
  directory (where job input and files are stored). 

  If the xRSL string is entered from the command line and is enclosed
  in double quotes, standard shell expansion of variables takes place.
  That is, if the file name
  contains an environment variable ("\verb#$...#"), %$
  the value of this variable is resolved locally, but if the name
  itself is also
  enclosed in double quotes, it will be resolved at the remote
  computing element:\\
  \verb#(executable=$ROOT_DIR/myprog.exe)# -- \verb#$ROOT_DIR# is resolved
  locally (\textit{will cause errors if the path does not exist at the
  execution machine})\\
  \verb#(executable="$ROOT_DIR/myprog.exe")# -- \verb#$ROOT_DIR# will be
  resolved remotely\\

  \hspace*{0.5cm}\phantomsection
  \index{XRSL:attribute:arguments}\addcontentsline{toc}{subsection}{arguments}
  \begin{shaded}
    \xrsl{arguments}
  \end{shaded}
  \versions{ARC 0.3, ARC 0.4, ARC 0.5, ARC 0.6, ARC 0.8}
  \begin{tabular}{lp{13cm}}
    Unique:&yes\\
    Operators:&\verb#=#\\
    User input:&\verb#(arguments=<string> [string] ... )#\\
    GM input:&\verb#(arguments=<executable> <string> [string] ... )#\\
    Example:&\verb#(arguments="10000" $(ATLAS)/input.dat)#\\ %$
  \end{tabular}

  List of the arguments for the executable.

  \begin{tabular}{llp{10cm}}
    \hspace*{1cm}&\texttt{string}&an argument\\
    \hspace*{1cm}&\texttt{executable}&the executable to be run by LRMS, taken by the
    ARC Client from the user-specified \texttt{executable} attribute\\
  \end{tabular}

  \hspace*{0.5cm}\phantomsection
  \index{XRSL:attribute:inputFiles}\addcontentsline{toc}{subsection}{inputFiles}
  \begin{shaded}
    \xrsl{inputFiles}
  \end{shaded}
  \versions{ARC 0.3, ARC 0.4, ARC 0.5, ARC 0.6, ARC 0.8}
  \begin{tabular}{lp{12cm}}
    Unique:&yes\\
    Operators:&\verb#=#\\
    User input:&\verb#(inputFiles=(<filename> <source>) ... )#\\
    GM input:&\verb#(inputFiles=(<filename> <URL>)#\\
    &\verb#        (<filename> [size][.checksum]) ... )#\\
    Example:&\verb#(inputFiles=("local_to_job" "gsiftp://se1.lu.se/p1/remote_one")#\\
    &\verb#        ("local_to_job.dat" "/scratch/local_to_me.dat")#\\
    &\verb#        ("same_name_as_in_my_current_dir" ""))#\\
  \end{tabular}

  List of files to be copied to the computing element before the
  execution.

  \begin{tabular}{llp{10cm}}
    \hspace*{1cm}&\texttt{filename} & destination file name, local to the computing element and
    always relative to the session directory\\
    \hspace*{1cm}&\texttt{source} & source of the file: (\texttt{gsiftp},
    \texttt{https}, \texttt{ftp}, \texttt{http} URLs, or a path, local
    to the submission node). If void ("", use the quotes!), the input file is taken
    from the submission directory.\\
    \hspace*{1cm}&\texttt{URL} & URL of the file (see Section~\ref{sec:url})\\
    \hspace*{1cm}&\texttt{size} & file size in bytes\\
    \hspace*{1cm}&\texttt{checksum} & file checksum (as returned by \texttt{cksum})\\
  \end{tabular}

  \begin{framed}
   ARC Client does not check whether the same destination name is specified for different source locations. Make sure all \texttt{filename} values are different!
  \end{framed}

  If the list does not contain the standard input file (as specified
  by \texttt{stdin}) and/or the executable file (as specified by
  \texttt{executable} if the given name), the ARC Client appends these files
  to the list. If the \verb#<source># is a URL, it is passed by the
  ARC Client to the GM without changes; if it is a local path (or void, ""),
  the ARC Client converts it to \verb#<size># and \verb#<checksum># and
  uploads those files to the execution cluster.

  \begin{framed}
  Please note that the \texttt{inputFiles} attribute is not meant to
  operate with directories, for reasons of access control and checksum verifications. You must specify a pair
  \verb#("<local_to_job>" "<source>")# for each file.
  \end{framed}

  \hspace*{0.5cm}\phantomsection
  \index{XRSL:attribute:executables}\addcontentsline{toc}{subsection}{executables}
  \begin{shaded}
    \xrsl{executables}
  \end{shaded}
  \versions{ARC 0.3, ARC 0.4, ARC 0.5, ARC 0.6, ARC 0.8}
  \begin{tabular}{lp{13cm}}
    Unique:&yes\\
    Operators:&\verb#=#\\
    User input:&\verb#(executables=<string> [string] ...)#\\
    GM input:&\verb#-"-#\\
    Example:&\verb#(executables="myscript.sh" "myjob.exe")#\\
  \end{tabular}

  List of files from the \texttt{inputFiles} set, which will be given
  executable permissions.

  \begin{tabular}{llp{10cm}}
    \hspace*{1cm}&\texttt{string}& file name, local to the computing element and
    relative to the session directory\\
  \end{tabular}

  If the executable file (as specified in \texttt{executable} and is
  relative to the session directory)  is not listed, it will be added
  to the list by the ARC Client.

  \hspace*{0.5cm}\phantomsection
  \index{XRSL:attribute:cache}\addcontentsline{toc}{subsection}{cache}
  \begin{shaded}
    \xrsl{cache}
  \end{shaded}
  \versions{ARC $\geq$ 0.3.34, ARC 0.4, ARC 0.5, ARC 0.6, ARC 0.8}
  \begin{tabular}{lp{13cm}}
    Unique:&yes\\
    Operators:&\verb#=#\\
    User input:&\verb#(cache="yes"|"no")#\\
    GM input:&\verb#-"-#\\
    Example:&\verb#(cache="yes")#\\
  \end{tabular}

  Specifies whether input files specified in the \texttt{inputFiles}
  should be placed by default in the cache or not. This does not
  affect files described by the \texttt{executables}, which will be
  placed in the session directory always.

  If not specified, default value is "yes".

  \begin{framed} 
  Cached files can not be modified by jobs by default. If your job has
  to modify input files, please use the \texttt{(readonly="no")} URL
  option for those files. This option does not affect whether or not
  the file is cached.
  \end{framed}

  \hspace*{0.5cm}\phantomsection
  \index{XRSL:attribute:outputFiles}\addcontentsline{toc}{subsection}{outputFiles} 
  \begin{shaded}
    \xrsl{outputFiles}
  \end{shaded}
  \versions{ARC 0.3, ARC 0.4, ARC 0.5, ARC 0.6, ARC 0.8}
  \begin{tabular}{lp{13cm}}
    Unique:&yes\\
    Operators:&\verb#=#\\
    User input:&\verb#(outputFiles=(<string> <URL>) ... )#\\
    GM input:&\verb#-"-#\\
    Example:&\verb#(outputFiles=("local_to_job.dat" "gsiftp://se1.uo.no/stored.dat")#\\
    &\verb#    ("local_to_job.dat" "rc://rc1.uc.dk/set1/stored_and_indexed.dat")#\\  
    &\verb#    ("local_to_job_dir/" ""))#\\  
  \end{tabular}

  List of files to be retrieved by the user or uploaded by the GM and
  indexed (registered) in a data indexing service, e.g. \globus\  RLS or Fireman.

  \begin{tabular}{llp{10cm}}
    \hspace*{1cm}&\texttt{string}& file name, local to the
    \textit{Computing Element (CE)}\index{Computing Element}. If this
    string ends with a backslash "/" and \verb#<URL># is empty, the
    entire directory is being kept at the execution site. If however
    this string ends with a backslash "/" but the \verb#<URL>#
    starts with \texttt{gsiftp} or \texttt{ftp}, the whole directory
    is transferred to the destination.\\
    \hspace*{1cm}&\texttt{URL} & destination URL of the remote file (see
    Section~\ref{sec:url}); if void ("", use the quotes!), the file is
    kept for manual retrieval.  Note that this can not be a local
    \texttt{file://} URL.\\
  \end{tabular}

  Using a Replica Catalog pseudo-URL (see Section~\ref{sec:url}), you should make
  sure that the location is already defined in the Replica Catalog. If more
  than one such location is specified, only those found in the Replica Catalog
  for this collection will be used. Grid Manager will store output files in
  \textbf{one} location only: the first randomly picked location that
  exists. If no locations are specified, all those found in the
  Replica Catalog for this collection will be tried.

  If in a \verb#rc://# pseudo-URL the server component \texttt{host[:port]/DN} is not
  specified, server specified in the \texttt{replicaCollection} attribute will be used.

  If the list does not contain standard output, standard error
  file names and GM log-files directory name (as specified by \texttt{stdout},
  \texttt{stderr} and \texttt{gmlog}), the ARC Client appends these
  items to the \texttt{outputFiles} list. If the \verb#<URL># is not specified (void,
  "", use the quotes!), files will be kept on SE and should be downloaded 
  by the user via the ARC Client. If specified name of file ends with "/", the 
  entire directory is kept.

  A convenient way to keep the entire job directory at the remote site
  for a manual retrieval is to specify \texttt{(outputfiles=("/" ""))}.

  In some cases, the list of output files may only be known after the
  job has completed. ARC allows a user to specify a list of output
  files dynamically in a file or files in the session directory as
  part of their job. The file(s) containing the output file
  information can be defined in the xRSL script as the path to the
  file relative to the session directory preceeded by '@'. The format
  of these files is lines of 2 values separated by a space. The first
  value contains name of the output file relative to the session
  directory and the second value is a URL to which the file will be
  uploaded.

  \begin{tabular}{lp{13cm}}
    Example:&\verb#(outputFiles=("@output.files" "")#\\
    & \emph{output.files} is generated by the user and contains \\
    &\verb#   file1 gsiftp://grid.domain.org/file1#\\
    &\verb#   file2 gsiftp://grid.domain.org/file2#\\
  \end{tabular}

  After the job completes, the file output.files in the session
  directory will be read and any files described within will be
  uploaded to the given URLs.

  \begin{framed}
    Please note that the \texttt{outputFiles} attribute is not meant to
    operate with directories: you must specify a pair
    \verb#("<local_to_job>" "[destination]")# for each file. One
    exception is when you want to preserve an entire directory at the
    remote computer for later \textbf{manual} download via
    \texttt{ngget}. In that case, simply add the trailing backslash
    "/" at the end of the remote directory name. You can not upload a
    directory to a URL location, only to your local disk.
  \end{framed}

  \hspace*{0.5cm}\phantomsection
  \index{XRSL:attribute:cpuTime}\addcontentsline{toc}{subsection}{cpuTime}
  \begin{shaded}
    \xrsl{cpuTime}
  \end{shaded}
  \versions{ARC $\geq$ 0.3.17, ARC 0.4, ARC 0.5, ARC 0.6, ARC 0.8}
  \begin{tabular}{lp{13cm}}
    Unique:&yes\\
    Operators:&\verb#=#\\
    User input:&\verb#(cpuTime=<time>)#\\
    GM input:&\verb#(cpuTime=<tttt>)#\\
    Example:&\verb#(cpuTime="240")#\\
  \end{tabular}

  Maximal CPU time request for the job. For a multi-processor job, this is a sum over all requested processors.

  \begin{tabular}{llp{10cm}}
    \hspace*{1cm}&\texttt{time} & time (in minutes if no unit is specified)\\
    \hspace*{1cm}&\texttt{tttt} & time converted by the ARC Client from \texttt{time} to 
    seconds.\\
  \end{tabular}
  
  The client converts time specified in the user-side XRSL file to seconds. 
  If no time unit is specified, the client assumes the time given in minutes. 
  Otherwise, a text format is accepted, i.e., any of the following will be interpreted
  properly (make sure to enclose such strings in quotes!):

  \begin{tabular}{ll}
    \hspace*{1cm}&\texttt{"1 week"}\\
    \hspace*{1cm}&\texttt{"3 days"}\\
    \hspace*{1cm}&\texttt{"2 days, 12 hours"}\\
    \hspace*{1cm}&\texttt{"1 hour, 30 minutes"}\\
    \hspace*{1cm}&\texttt{"36 hours"}\\
    \hspace*{1cm}&\texttt{"9 days"}\\
    \hspace*{1cm}&\texttt{"240 minutes"}\\
  \end{tabular}
 
  If both \texttt{cpuTime} and \texttt{wallTime} are specified, the ARC Client converts them 
  both. \texttt{cpuTime} can not be specified together with \texttt{gridTime}
  or \texttt{benchmarks}.
 
  \begin{framed}
    This attribute should be used to direct the job to a system with
    sufficient CPU resources, typically, a batch queue with the
    sufficient upper time limit. Jobs exceeding this maximum most
    likely will be \textbf{terminated} by remote systems! If
    time limits are not specified, the limit is not set and jobs
    can run as long as the system settings allow (note that in this
    case you can not avoid queues with too short time limits).
  \end{framed}

  \hspace*{0.5cm}\phantomsection
  \index{XRSL:attribute:wallTime}\addcontentsline{toc}{subsection}{wallTime}
  \begin{shaded}
    \xrsl{wallTime}
  \end{shaded}
  \versions{ARC $\geq$ 0.5.27, ARC 0.6, ARC 0.8}
  \begin{tabular}{lp{13cm}}
    Unique:&yes\\
    Operators:&\verb#=#\\
    User input:&\verb#(wallTime=<time>)#\\
    GM input:&\verb#(wallTime=<tttt>)#\\
    Example:&\verb#(wallTime="240")#\\
  \end{tabular}

  Maximal wall clock time request for the job. 

  \begin{tabular}{llp{10cm}}
    \hspace*{1cm}&\texttt{time} & time (in minutes if no unit is specified)\\
    \hspace*{1cm}&\texttt{tttt} & time converted by the ARC Client to seconds\\
  \end{tabular}
  
  The client converts time specified in the user-side XRSL file seconds. If no time unit is specified,
  the client assumes the time given in minutes. Otherwise, a text
  format is accepted, i.e., any of the following will be interpreted
  properly (make sure to enclose such strings in quotes!):

  \begin{tabular}{ll}
    \hspace*{1cm}&\texttt{"1 week"}\\
    \hspace*{1cm}&\texttt{"3 days"}\\
    \hspace*{1cm}&\texttt{"2 days, 12 hours"}\\
    \hspace*{1cm}&\texttt{"1 hour, 30 minutes"}\\
    \hspace*{1cm}&\texttt{"36 hours"}\\
    \hspace*{1cm}&\texttt{"9 days"}\\
    \hspace*{1cm}&\texttt{"240 minutes"}\\
  \end{tabular}
 
  If both \texttt{cpuTime} and \texttt{wallTime} are specified, the ARC Client converts them both.
  \texttt{wallTime} can not be specified together with \texttt{gridTime}
  or \texttt{benchmarks}. If only \texttt{wallTime} is specified, but not \texttt{cpuTime},
  the corresponding  \texttt{cpuTime} value is evaluated by the ARC Client and added to the job description.
 
  \begin{framed}
    This attribute should be used to direct the job to a system with
    sufficient CPU resources, typically, a batch queue with the
    sufficient upper time limit. Jobs exceeding this maximum most
    likely will be \textbf{terminated} by remote systems! If 
    time limits are not specified, the limit is not set and jobs
    can run as long as the system settings allow (note that in this
    case you can not avoid queues with too short time limits).
  \end{framed}

  \hspace*{0.5cm}\phantomsection
  \index{XRSL:attribute:gridTime}\addcontentsline{toc}{subsection}{gridTime}
  \begin{shaded}
    \xrsl{gridTime}
  \end{shaded}
  \versions{ARC $\geq$ 0.3.31, ARC 0.4, ARC 0.5, ARC 0.6, ARC 0.8}
  \begin{tabular}{lp{13cm}}
    Unique:&yes\\
    Operators:&\verb#=#\\
    User input:&\verb#(gridTime=<time>)#\\
    GM input:& none\\
    Example:&\verb#(gridTime="2 h")#\\
  \end{tabular}

  Maximal CPU time request for the job scaled to the 2.8~GHz
  Intel\textsuperscript{\textregistered}
  Pentium\textsuperscript{\textregistered}~4 processor.

  \begin{tabular}{llp{10cm}}
    \hspace*{1cm}&\texttt{time} & time (in minutes if no unit is specified)\\
  \end{tabular}
  
  The attribute is completely analogous to \texttt{cpuTime}, except
  that it will be recalculated to the actual CPU time request for each
  queue, depending on the published processor clock speed.
  
  \texttt{gridTime} can not be specified together with \texttt{cpuTime}
  or \texttt{wallTime}. If only \texttt{gridTime} is specified, but not \texttt{cpuTime},
  the corresponding  \texttt{cpuTime} value is evaluated by the ARC Client and added to the job description.

  \hspace*{0.5cm}\phantomsection
  \index{XRSL:attribute:benchmarks}\addcontentsline{toc}{subsection}{benchmarks}
  \begin{shaded}
    \xrsl{benchmarks}
  \end{shaded}
  \versions{ARC $\geq$ 0.5.14, ARC 0.6, ARC 0.8}
  \begin{tabular}{lp{13cm}}
    Unique:&yes\\
    Operators:&\verb#=#\\
    User input:&\verb#(benchmarks=(<string> <value> <time>) ... )#\\
    GM input:&\\
    Example:&\verb#(benchmarks=("mybenchmark" "10" "1 hour, 30 minutes"))#\\
  \end{tabular}

  Evaluate a job's \texttt{cpuTime} based on benchmark values.

  \begin{tabular}{llp{10cm}}
    \hspace*{1cm}&\texttt{string} & benchmark name\\
    \hspace*{1cm}&\texttt{value} & benchmark value of reference machine\\
    \hspace*{1cm}&\texttt{time} & the \texttt{cpuTime} the job requires on the reference machine\\
  \end{tabular}
  
  \texttt{benchmarks} can not be specified together with \texttt{cpuTime}
  or \texttt{wallTime}. If only \texttt{benchmarks} is specified, but not \texttt{cpuTime},
  the corresponding  \texttt{cpuTime} value is evaluated by the ARC Client and added to the job description.

  \hspace*{0.5cm}\phantomsection
  \index{XRSL:attribute:memory}\addcontentsline{toc}{subsection}{memory}
  \begin{shaded}
    \xrsl{memory}
  \end{shaded}
  \versions{ARC $\geq$ 0.3.17\footnote{Was called \texttt{maxMemory} for ARC $\leq$ 0.3.16}, ARC 0.4, ARC 0.5, ARC 0.6, ARC 0.8}
  \begin{tabular}{lp{13cm}}
    Unique:&yes\\
    Operators:&\verb#=#\\
    User input:&\verb#(memory=<integer>)#\\
    GM input:&\verb#-"-#\\
    Example:&\verb#(memory>="500")#\\
  \end{tabular}

  Memory required for the job, per rank.

  \begin{tabular}{llp{10cm}}
    \hspace*{1cm}&\texttt{integer} & size (Mbytes)\\
  \end{tabular}

  \begin{framed}
    Similarly to \texttt{cpuTime}, this attribute should be used to
    direct a job to a resource with a sufficient capacity. Jobs
    exceeding this memory limit will most likely be \textbf{terminated}
    by the remote system.
  \end{framed}

  \hspace*{0.5cm}\phantomsection
  \index{XRSL:attribute:disk}\addcontentsline{toc}{subsection}{disk}
  \begin{shaded}
    \xrsl{disk}
  \end{shaded}
  \versions{ARC $\geq$ 0.3.17\footnote{Was called \texttt{maxDisk} for ARC $\leq$ 0.3.16}, ARC 0.4, ARC 0.5, ARC 0.6, ARC 0.8}
  \begin{tabular}{lp{13cm}}
    Unique:&no\\
    Operators:&\verb#=   !=   >   <   >=   <=#\\
    User input:&\verb#(disk=<integer>)#\\
    GM input:&none\\
    Example:&\verb#(disk="500")#\\
  \end{tabular}

  Disk space required for the job.

  \begin{tabular}{llp{10cm}}
    \hspace*{1cm}&\texttt{integer}  & disk space, Mbytes\\
  \end{tabular}

  \begin{framed}
    This attribute is used at the job submission time to find a system
    with sufficient disk space. However, it \textbf{does not
    guarantee} that this space will be available at the end of the
    job, as most known systems do not allow for disk space
    allocation. Eventually, a remote system can terminate a job that
    exceeds the requested disk space.
  \end{framed}

  \hspace*{0.5cm}\phantomsection
  \index{XRSL:attribute:runTimeEnvironment}\addcontentsline{toc}{subsection}{runTimeEnvironment}
  \begin{shaded}
    \xrsl{runTimeEnvironment}
  \end{shaded}
  \versions{ARC 0.3, ARC 0.4, ARC 0.5, ARC 0.6, ARC 0.8}
  \begin{tabular}{lp{13cm}}
    Unique:&no\\
    Operators:&\verb#=   !=   >   <   >=   <=#\\
    User input:&\verb#(runTimeEnvironment=<string>)(runTimeEnvironment=<string>)#\\
    GM input:&For ARC $\leq$ 0.5.39: \verb#(runTimeEnvironment="<string>" "<string>")#\\
    Example:&\verb#(runTimeEnvironment>="APPS/HEP/ATLAS-10.0.1")#\\
  \end{tabular}

  Required runtime environment.

  \begin{tabular}{llp{10cm}}
    \hspace*{1cm}&\texttt{string} & environment name\\
  \end{tabular}

  The site to submit the job to will be chosen by the ARC Client among those
  advertising specified runtime environments. Before starting the job,
  the GM will set up environment variables and paths according to
  those requested. Runtime environment names are defined by Virtual 
  Organizations, and tend to be organized in name spaces.

  To request several environments, repeat the attribute string:\\
  \verb#(runTimeEnvironment="ENV1")(runTimeEnvironment="ENV2")# etc.   

  To make a disjunct-request, use a boolean expression:\\
  \verb#(|(runTimeEnvironment="env1")(runTimeEnvironment="env2"))#.

  You can use "\verb#>=#" or "\verb#<=#" operators: job will be submitted to
  any suitable site that satisfies such requirements, and among the available at the sites
  runtime environments, the highest version satisfying a requirement will be requested
  in the pre-processed xRSL script.

  Runtime environment string interpretation is
  case-insensitive.  If a runtime environment string consists of a name
  and a version number, a partial specification is possible: it is sufficient
  to request only the name and use "\verb#>#" or "\verb#>=#" operators to select the highest version.

  \hspace*{0.5cm}\phantomsection
  \index{XRSL:attribute:middleware}\addcontentsline{toc}{subsection}{middleware}
  \begin{shaded}
    \xrsl{middleware}
  \end{shaded}
  \versions{ARC 0.3, ARC 0.4, ARC 0.5, ARC 0.6, ARC 0.8}
  \begin{tabular}{lp{13cm}}
    Unique:&no\\
    Operators:&\verb#=   !=   >   <   >=   <=#\\
    User input:&\verb#(middleware=<string>)#\\
    GM input:&\verb#-"-#\\
    Example:&\verb#(middleware="nordugrid-arc-0.5.99")#\\
  \end{tabular}

  Required middleware version. Make sure to specify full name and version number.

  \begin{tabular}{llp{10cm}}
    \hspace*{1cm}&\texttt{string} & Grid middleware name.\\
  \end{tabular}

  The site to submit the job to will be chosen by the ARC Client among those
  advertising specified middleware. Usage is identical to that of the
  \texttt{runTimeEnvironment}. Use the "\verb#>=#" operator to
  request a version "equal or higher".% Not true (see bug 1485): Request
  %\texttt{(middleware="nordugrid-arc")} defaults to
  %\verb#(middleware>="nordugrid-arc-0.0.0.0")#.

  \hspace*{0.5cm}\phantomsection
  \index{XRSL:attribute:opsys}\addcontentsline{toc}{subsection}{opsys}
  \begin{shaded}
    \xrsl{opsys}
  \end{shaded}
  \versions{ARC $\geq$ 0.3.21, ARC 0.4, ARC 0.5, ARC 0.6, ARC 0.8}
  \begin{tabular}{lp{13cm}}
    Unique:&no\\
    Operators:&\verb#=   !=   >   <   >=   <=#\\
    User input:&\verb#(opsys=<string>)#\\
    GM input:&\verb#-"-#\\
    Example:&\verb#(opsys="FC3")#\\
  \end{tabular}

  Required operating system.

  \begin{tabular}{llp{10cm}}
    \hspace*{1cm}&\texttt{string} & Operating system name and version.\\
  \end{tabular}

  The site to submit the job to will be chosen by the ARC Client among those
  advertising specified operating system. Usage is identical to that of
  \texttt{runTimeEnvironment} and \texttt{middleware}. Use the "\verb#>=#" operator to
  request a version "equal or higher".

  \hspace*{0.5cm}\phantomsection
  \index{XRSL:attribute:stdin}\addcontentsline{toc}{subsection}{stdin}
  \begin{shaded}
    \xrsl{stdin}
  \end{shaded}
  \versions{ARC 0.3, ARC 0.4, ARC 0.5, ARC 0.6, ARC 0.8}
  \begin{tabular}{lp{13cm}}
    Unique:&yes\\
    Operators:&\verb#=#\\
    User input:&\verb#(stdin=<string>)#\\
    GM input:&\verb#-"-#\\
    Example:&\verb#(stdin="myinput.dat")#\\
  \end{tabular}

  The standard input file.

  \begin{tabular}{llp{10cm}}
    \hspace*{1cm}&\texttt{string} & file name, local to the computing element\\
  \end{tabular}

  The standard input file should be listed in the \texttt{inputFiles}
  attribute; otherwise it will be forced to that list by the ARC Client.

  \hspace*{0.5cm}\phantomsection
  \index{XRSL:attribute:stdout}\addcontentsline{toc}{subsection}{stdout}
  \begin{shaded}
    \xrsl{stdout}
  \end{shaded}
  \versions{ARC 0.3, ARC 0.4, ARC 0.5, ARC 0.6, ARC 0.8}
  \begin{tabular}{lp{13cm}}
    Unique:&yes\\
    Operators:&\verb#=#\\
    User input:&\verb#(stdout=<string>)#\\
    GM input:&\verb#-"-#\\
    Example:&\verb#(stdout="myoutput.txt")#\\
  \end{tabular}

  The standard output file.

  \begin{tabular}{llp{10cm}}
    \hspace*{1cm}&\texttt{string} & file name, local to the computing element and
    relative to the session directory.\\
  \end{tabular}

  The standard output file should be listed in the
  \texttt{outputFiles} attribute; otherwise it will be forced to that
  list by the ARC Client. If the standard output is not defined, ARC Client assigns a
  name.

  \hspace*{0.5cm}\phantomsection
  \index{XRSL:attribute:stderr}\addcontentsline{toc}{subsection}{stderr}
  \begin{shaded}
    \xrsl{stderr}
  \end{shaded}
  \versions{ARC 0.3, ARC 0.4, ARC 0.5, ARC 0.6, ARC 0.8}
  \begin{tabular}{lp{13cm}}
    Unique:&yes\\
    Operators:&\verb#=#\\
    User input:&\verb#(stderr=<string>)#\\
    GM input:&\verb#-"-#\\
    Example:&\verb#(stderr="myjob.err")#\\
  \end{tabular}

  The standard error file.

  \begin{tabular}{llp{10cm}}
    \hspace*{1cm}&\texttt{string} & file name, local to the computing element and
    relative to the session directory.\\
  \end{tabular}

  The standard error file should be listed as an \texttt{outputFiles}
  attribute; otherwise it will be forced to that list by the ARC Client. If
  the standard error is not defined, ARC Client assigns a name. If \texttt{join} is 
  specified with value "yes", ARC Client adds \texttt{stderr} to the 
  pre-processed xRSL script with the same value as \texttt{stdout}.

  \hspace*{0.5cm}\phantomsection
  \index{XRSL:attribute:join}\addcontentsline{toc}{subsection}{join}
  \begin{shaded}
    \xrsl{join}
  \end{shaded}
  \versions{ARC 0.3, ARC 0.4, ARC 0.5, ARC 0.6, ARC 0.8}
  \begin{tabular}{lp{13cm}}
    Unique:&yes\\
    Operators:&\verb#=#\\
    User input:&\verb#(join="yes"|"no")#\\
    GM input:none\\
    Example:&\verb#(join="yes")#\\
  \end{tabular}

  If "yes", joins \texttt{stderr} and  \texttt{stdout} files into
  the \texttt{stdout} one. Default is \verb#no#.

  \hspace*{0.5cm}\phantomsection
  \index{XRSL:attribute:gmlog}\addcontentsline{toc}{subsection}{gmlog}
  \begin{shaded}
    \xrsl{gmlog}
  \end{shaded}
  \versions{ARC 0.3, ARC 0.4, ARC 0.5, ARC 0.6, ARC 0.8}
  \begin{tabular}{lp{13cm}}
    Unique:&yes\\
    Operators:&\verb#=#\\
    User input:&\verb#(gmlog=<string>)#\\
    GM input:&\verb#-"-#\\
    Example:&\verb#(gmlog="myjob.log")#\\
  \end{tabular}

  A name of the directory containing grid-specific diagnostics per job.

  \begin{tabular}{llp{10cm}}
    \hspace*{1cm}&\texttt{string} & a directory, local to the computing element and
    relative to the session directory\\
  \end{tabular}

%  The job log file should be listed as an \texttt{outputFiles} attribute;
%  otherwise it will be forced to that list by the ARC Client.
	This directory is kept in the session directory to be
	available for retrieval (ARC Client forces it to the list if
	\texttt{outputFiles})

  \hspace*{0.5cm}\phantomsection
  \index{XRSL:attribute:jobName}\addcontentsline{toc}{subsection}{jobName}
  \begin{shaded}
    \xrsl{jobName}
  \end{shaded}
  \versions{ARC 0.3, ARC 0.4, ARC 0.5, ARC 0.6, ARC 0.8}
  \begin{tabular}{lp{13cm}}
    Unique:&yes\\
    Operators:&\verb#=#\\
    User input:&\verb#(jobName=<string>)#\\
    GM input:&\verb#-"-#\\
    Example:&\verb#(jobName="My Job nr. 1")#\\
  \end{tabular}

  User-specified job name.

  \begin{tabular}{llp{10cm}}
    \hspace*{1cm}&\texttt{string} & job name\\
  \end{tabular}

  This name is meant for convenience of the user. It can be used to
  select the job while using the ARC Client. It is also available through the
  Information System.

  \hspace*{0.5cm}\phantomsection
  \index{XRSL:attribute:ftpThreads}\addcontentsline{toc}{subsection}{ftpThreads}
  \begin{shaded}
    \xrsl{ftpThreads}
  \end{shaded}
  \versions{ARC 0.3, ARC 0.4, ARC 0.5, ARC 0.6, ARC 0.8}
  \begin{tabular}{lp{13cm}}
    Unique:&yes\\
    Operators:&\verb#=#\\
    User input:&\verb#(ftpThreads=<integer>)#\\
    GM input:&\verb#-"-#\\
    Example:&\verb#(ftpThreads="4")#\\
  \end{tabular}

  Defines how many parallel streams will be used by the GM during
  \texttt{gsiftp} and \texttt{http(s|g)}transfers of files.

  \begin{tabular}{llp{10cm}}
    \hspace*{1cm}&\texttt{integer} & a number from 1 to 10\\
  \end{tabular}

  If not specified, parallelism is not used.

  \hspace*{0.5cm}\phantomsection
  \index{XRSL:attribute:acl}\addcontentsline{toc}{subsection}{acl}
  \begin{shaded}
    \xrsl{acl}
  \end{shaded}
  \versions{ARC $\geq$ 0.5.12, ARC 0.6, ARC 0.8}

  \begin{tabular}{lp{13cm}}
    Unique:&no\\
    Operators:&\verb#=#\\
    User input:&\verb#(acl=<xml>)#\\
    GM input:&\verb#-"-#\\
    Example:&\verb#(acl="<?xml version=""1.0""?>#\\
    &\verb#    <gacl version=""0.0.1""><entry><any-user></any-user>#\\
    &\verb#    <allow><write/><read/><list/><admin/></allow></entry></gacl>")#\\
  \end{tabular}



  Makes use of GACL~\cite{gacl} rules to list users who are allowed to
  access and control job in addition to job's owner. Access and
  control levels are specified per user. \texttt{any-user} tag refers
  to any user authorized at the execution cluster. To get more
  information about GACL please refer to \url{http://www.gridsite.org}.

  \begin{tabular}{llp{10cm}}
    \hspace*{1cm}&\texttt{xml} & a GACL-compliant XML string defining
    access control list\\
  \end{tabular}

  Following job control levels can be specified via \texttt{acl}:

  \begin{tabular}{llll}
    \hspace*{1cm}&\texttt{write}&--&\parbox[t]{11cm}{allows to modify contents of job data (job directory) and control job flow (cancel, clean, etc.)}\\
    \hspace*{1cm}&\texttt{read}&--&\parbox[t]{11cm}{allows to read content of job data
    (contents of job directory)}\\
    \hspace*{1cm}&\texttt{list}&--&\parbox[t]{11cm}{allows to list files available for
    the job (contents of job directory)}\\
    \hspace*{1cm}&\texttt{admin}&--&\parbox[t]{11cm}{allows to do everything --
    full equivalence to job ownership}\\
  \end{tabular}

  \hspace*{0.5cm}\phantomsection
  \index{XRSL:attribute:cluster}\addcontentsline{toc}{subsection}{cluster}
  \begin{shaded}
    \xrsl{cluster}
  \end{shaded}
  \versions{ARC 0.3, ARC 0.4, ARC 0.5, ARC 0.6, ARC 0.8}
  \begin{tabular}{lp{13cm}}
    Unique:&yes\\
    Operators:&\verb#=   !=#\\
    User input:&\verb#(cluster=<string>)#\\
    GM input:&\verb#-"-#\\
    Example:&\verb#(cluster="cluster1.site2.org")#\\
  \end{tabular}

  The name of the execution cluster.

  \begin{tabular}{llp{10cm}}
    \hspace*{1cm}&\texttt{string} & known cluster name, or a substring of it\\
  \end{tabular}

  Use this attribute to explicitly force job submission to a cluster,
  or to avoid such. The job will not be submitted if the cluster does
  not satisfy other requirements of the job. Disjunct-requests of the
  kind \verb#(|(cluster="clus1")(cluster="clus2"))# are supported. To
  exclude a cluster, use \verb#(cluster!="clus3")#.

  \hspace*{0.5cm}\phantomsection
  \index{XRSL:attribute:queue}\addcontentsline{toc}{subsection}{queue}
  \begin{shaded}
    \xrsl{queue}
  \end{shaded}
  \versions{ARC 0.3, ARC 0.4, ARC 0.5, ARC 0.6, ARC 0.8}
  \begin{tabular}{lp{13cm}}
    Unique:&yes\\
    Operators:&\verb#=   !=#\\
    User input:&\verb#(queue=<string>)#\\
    GM input:&\verb#-"-#\\
    Example:&\verb#(queue="pclong")#\\
  \end{tabular}
  
  The name of the remote batch queue. 

  \begin{framed}
    Use only when you are sure that the queue by this name does exist.
  \end{framed}

  \begin{tabular}{llp{10cm}}
    \hspace*{1cm}&\texttt{string} & known queue name\\
  \end{tabular}

  While users are not expected to specify \texttt{queue} in job descriptions, this attribute \textbf{must} be present in the GM-side xRSL. In fact, this is primarily an internal attribute, added to the job description by client tools after resource discovery and matchmaking. Still, users can specify this attribute to explicitly force job submission to a queue: when specified explicitly by the user, this value will not be overwritten by the ARC Client, and an attempt will be made to submit the job to the specified queue.

  If for some reason (e.g. due to a client tool error) \texttt{queue} is absent from the GM-side xRSL, GM on the selected cluster will attempt to submit the job to the default queue if such is specified in the GM configuration.

  \hspace*{0.5cm}\phantomsection
  \index{XRSL:attribute:startTime}\addcontentsline{toc}{subsection}{startTime}
  \begin{shaded}
    \xrsl{startTime}
  \end{shaded}
  \versions{ARC 0.3, ARC 0.4, ARC 0.5, ARC 0.6, ARC 0.8}
  \begin{tabular}{lp{13cm}}
    Unique:&yes\\
    Operators:&\verb#=#\\
    User input:&\verb#(startTime=<time>)#\\
    GM input:&\verb#(startTime=<tttt>)#\\
    Example:&\verb#(startTime="2002-05-25 21:30")#\\
  \end{tabular}

  Time to start job processing by the Grid Manager, such as e.g. start downloading input files.

  \begin{tabular}{llp{10cm}}
    \hspace*{1cm}&\texttt{time} & time string, YYYY-MM-DD hh:mm:ss\\ 
    \hspace*{1cm}&\texttt{tttt} & time string, YYYYMMDDhhmmss[Z] (converted by the ARC Client from
    \texttt{time})\\
  \end{tabular}
  
  \begin{framed}
   Actual job processing on a worker node starts depending on local scheduling mechanisms, but not sooner than \texttt{startTime}. 
   \end{framed}

  \hspace*{0.5cm}\phantomsection
  \index{XRSL:attribute:lifeTime}\addcontentsline{toc}{subsection}{lifeTime}
  \begin{shaded}
    \xrsl{lifeTime}
  \end{shaded}
  \versions{ARC 0.3, ARC 0.4, ARC 0.5, ARC 0.6, ARC 0.8}
  \begin{tabular}{lp{13cm}}
    Unique:&yes\\
    Operators:&\verb#=#\\
    User input:&\verb#(lifeTime=<time>)#\\
    GM input:&\verb#(lifeTime=<tttt>)#\\
    Example:&\verb#(lifeTime="2 weeks")#\\
  \end{tabular}

  Maximal time to keep job files (the session directory) on the
  gatekeeper upon job completion.

  \begin{tabular}{llp{10cm}}
    \hspace*{1cm}&\texttt{time} & time (in minutes if no unit is specified)\\
    \hspace*{1cm}&\texttt{tttt} & time (seconds, converted by the ARC Client from \texttt{time})\\
  \end{tabular}

  Typical life time is 1 day (24 hours). Specified life time can not
  exceed local settings.

  \hspace*{0.5cm}\phantomsection
  \index{XRSL:attribute:notify}\addcontentsline{toc}{subsection}{notify}
  \begin{shaded}
    \xrsl{notify}
  \end{shaded}
  \versions{ARC 0.3, ARC 0.4, ARC 0.5, ARC 0.6, ARC 0.8}
  \begin{tabular}{lp{13cm}}
    Unique:&yes\\
    Operators:&\verb#=#\\
    User input:&\verb#(notify=<string> [string] ... )#\\
    GM input:&\verb#-"-#\\
    Example:&\verb#(notify="be your.name@your.domain.com")#\\
  \end{tabular}

  Request e-mail notifications on job status change.

  \begin{tabular}{llp{10cm}}
    \hspace*{1cm}&\texttt{string} & string of the format:
    \verb#[b][q][f][e][c][d] user1@domain1 [user2@domain2] ...#\\
    \hspace*{1cm}&&here flags indicating the job status are:\\
    \hspace*{1cm}&&\texttt{b} -- begin (PREPARING)\\
    \hspace*{1cm}&&\texttt{q} -- queued (INLRMS)\\
    \hspace*{1cm}&&\texttt{f} -- finalizing (FINISHING)\\
    \hspace*{1cm}&&\texttt{e} -- end (FINISHED)\\
    \hspace*{1cm}&&\texttt{c} -- cancellation (CANCELLED)\\
    \hspace*{1cm}&&\texttt{d} -- deleted (DELETED)\\
  \end{tabular}

  When no notification flags are specified, default value of ``\verb#eb#''
  will be used, i.e., notifications will be sent at the job's beginning
  and at its end.

  No more than 3 e-mail addresses per status change accepted.

  \hspace*{0.5cm}\phantomsection
  \index{XRSL:attribute:replicaCollection}\addcontentsline{toc}{subsection}{replicaCollection}
  \begin{shaded}
    \xrsl{replicaCollection}
  \end{shaded}
  \versions{ARC 0.3, ARC 0.4, ARC 0.5, ARC 0.6, ARC 0.8}
  \begin{tabular}{lp{13cm}}
    Unique:&no\\
    Operators:&\verb#=#\\
    User input:&\verb#(replicaCollection=<URL>)#\\
    GM input:&\verb#-"-#\\
    Example:&\verb#(replicaCollection="ldap://grid.uio.no:389/lc=TestCollection,#\\
    &\verb#          rc=NorduGrid,nordugrid,dc=org")#\\
  \end{tabular}

  Location of a logical collection in the Replica Catalog.

  \begin{tabular}{llp{10cm}}
    \hspace*{1cm}&\texttt{URL} & LDAP directory specified as an URL (\verb#ldap://host[:port]/dn#)\\
  \end{tabular}

  \hspace*{0.5cm}\phantomsection
  \index{XRSL:attribute:rerun}\addcontentsline{toc}{subsection}{rerun}
  \begin{shaded}
    \xrsl{rerun}
  \end{shaded}
  \versions{ARC 0.3, ARC 0.4, ARC 0.5, ARC 0.6, ARC 0.8}
  \begin{tabular}{lp{13cm}}
    Unique:&yes\\
    Operators:&\verb#=#\\
    User input:&\verb#(rerun=<integer>)#\\
    GM input:&\verb#-"-#\\
    Example:&\verb#(rerun="2")#\\
  \end{tabular}

  Number of reruns (if a system failure occurs).

  \begin{tabular}{llp{10cm}}
    \hspace*{1cm}&\texttt{integer}  & an integer number\\
  \end{tabular}
    
  If not specified, the default is 0. Default maximal allowed value is
  5. The job may be rerun after failure in any state for which reruning 
  has sense. To initiate rerun user has to use the \texttt{ngresume} command.

  \hspace*{0.5cm}\phantomsection
  \index{XRSL:attribute:architecture}\addcontentsline{toc}{subsection}{architecture}
  \begin{shaded}
    \xrsl{architecture}
  \end{shaded}
  \versions{ARC 0.3, ARC 0.4, ARC 0.5, ARC 0.6, ARC 0.8}
  \begin{tabular}{lp{13cm}}
    Unique:&no\\
    Operators:&\verb#=   !=#\\
    User input:&\verb#(architecture=<string>)#\\
    GM input:&\\
    Example:&\verb#(architecture="i686")#\\
  \end{tabular}

  Request a specific architecture.

  \begin{tabular}{llp{10cm}}
    \hspace*{1cm}&\texttt{string} & architecture  (e.g., as produced by uname -a)\\
  \end{tabular}

  \hspace*{0.5cm}\phantomsection
  \index{XRSL:attribute:nodeAccess}\addcontentsline{toc}{subsection}{nodeAccess}
  \begin{shaded}
    \xrsl{nodeAccess}
  \end{shaded}
  \versions{ARC $\geq$ 0.3.24, ARC 0.4, ARC 0.5, ARC 0.6, ARC 0.8}
  \begin{tabular}{lp{13cm}}
    Unique:&yes\\
    Operators:&\verb#=#\\
    User input:&\verb#(nodeAccess="inbound"|"outbound")#\\
    GM input:&\\
    Example:&\verb#(nodeAccess="inbound")#\\
  \end{tabular}

  Request cluster nodes with inbound or outbound IP connectivity. If
  both are needed, a conjunct request should be specified.

  \hspace*{0.5cm}\phantomsection
  \index{XRSL:attribute:dryRun}\addcontentsline{toc}{subsection}{dryRun}
  \begin{shaded}
    \xrsl{dryRun}
  \end{shaded}
  \versions{ARC 0.3, ARC 0.4, ARC 0.5, ARC 0.6, ARC 0.8}
  \begin{tabular}{lp{13cm}}
    Unique:&yes\\
    Operators:&\verb#=#\\
    User input:&\verb#(dryRun="yes"|"no")#\\
    GM input:&\verb#-"-#\\
    Example:&\verb#(dryRun="yes")#\\
  \end{tabular}

  If "yes", do dry-run: job description is sent to the optimal destination, input files are transferred, but no actual job submission to LRMS
  is made. Typically used for xRSL and communication validation.

  \hspace*{0.5cm}\phantomsection
  \index{XRSL:attribute:rsl\_substitution}\addcontentsline{toc}{subsection}{rsl\_substitution}
  \begin{shaded}
    \xrsl{rsl\_substitution}
  \end{shaded}
  \versions{ARC 0.3, ARC 0.4, ARC 0.5, ARC 0.6, ARC 0.8}
  \begin{tabular}{lp{13cm}}
    Unique:&no\\
    Operators:&\verb#=#\\
    User input:&\verb#(rsl_substitution=(<string1> <string2>))#\\
    GM input:&\verb#-"-#\\
    Example:&\verb#(rsl_substitution=("ATLAS" "/opt/atlas"))#\\
  \end{tabular}

  Substitutes \verb#<string2># with \verb#<string1># for
  \textbf{internal} RSL use.

  \begin{tabular}{llp{10cm}}
    \hspace*{1cm}&\texttt{string1} & new internal RSL variable\\
    \hspace*{1cm}&\texttt{string2} & any string, e.g., existing combination of
    variables or a path\\
  \end{tabular}
  
  Use this attribute to define variables that simplify xRSL editing,
  e.g. when same path is used in several values, typically in
  \texttt{inputFiles}. Only one pair per substitution is allowed. To
  request several substitution, concatenate such requests. Bear in
  mind that substitution must be defined \textbf{prior} to actual use
  of a new variable \texttt{string1}.

  After the substitution is defined, it should be used in a way
  similar to shell variables in scripts: enclosed in round brackets,
  preceded with a dollar sign, \textbf{without quotes}:\\
  \verb#(inputfiles=("myfile" $(ATLAS)/data/somefile))#

  Unlike the \texttt{environment} attribute, \texttt{rsl\_substitution}
  definition is only used by the client and is valid inside xRSL
  script. It can not be used to define environment or shell variable
  at the execution site.

  \hspace*{0.5cm}\phantomsection
  \index{XRSL:attribute:environment}\addcontentsline{toc}{subsection}{environment}
  \begin{shaded}
    \xrsl{environment}
  \end{shaded}
  \versions{ARC 0.3, ARC 0.4, ARC 0.5, ARC 0.6, ARC 0.8}
  \begin{tabular}{lp{13cm}}
    Unique:&no\\
    Operators:&\verb#=#\\
    User input:&\verb#(environment=(<VAR> <string>) [(<VAR> <string>)] ... )#\\
    GM input:&\verb#-"-#\\
    Example:&\verb#(environment=("ATLSRC" "/opt/atlas/src")#\\
    &\verb#         ("ALISRC" "/opt/alice/src"))#\\
  \end{tabular}

  Defines execution shell environment variables.

  \begin{tabular}{llp{10cm}}
    \hspace*{1cm}&\texttt{VAR} & new variable name\\
    \hspace*{1cm}&\texttt{string} & any string, e.g., existing combination of
    variables or a path\\
  \end{tabular}

  Use this to define variables at an execution site. Unlike the
  \texttt{rsl\_substitution} attribute, it can not be used to define
  variables on the client side.

%%   \hspace*{0.5cm}\phantomsection
%%   \index{XRSL:attribute:jobtype}\addcontentsline{toc}{subsection}{jobtype}
%%   \begin{shaded}
%%     \xrsl{jobtype}
%%   \end{shaded}
%%   \versions{ARC $\geq$ 0.3.12, ARC 0.4}
%%   \begin{tabular}{lp{13cm}}
%%    Unique:&yes\\
%%    Operators:&\verb#=#\\
%%     User input:&\verb#(jobtype= "single"|"multiple"|"mpi"|"condor" )#\\
%%     Example:&\verb#(jobtype="mpi")#\\
%%   \end{tabular}

%%   Specifies a type of job (default is \texttt{single}).

  \hspace*{0.5cm}\phantomsection
  \addcontentsline{toc}{subsection}{count}\index{XRSL:attribute:count}
  \begin{shaded}
    \xrsl{count}
  \end{shaded}
  \versions{ARC $\geq$ 0.3.12, ARC 0.4, ARC 0.5, ARC 0.6, ARC 0.8}
  \begin{tabular}{lp{13cm}}
    Unique:&yes\\
    Operators:&\verb#=#\\
    User input:&\verb#(count=<integer>)#\\
    GM input:&\verb#-"-#\\
    Example:&\verb#(count="4")#\\
  \end{tabular}

  Specifies amount of sub-jobs to be submitted for parallel tasks.

  \hspace*{0.5cm}\phantomsection
  \addcontentsline{toc}{subsection}{jobreport}\index{XRSL:attribute:jobreport}
  \begin{shaded}
    \xrsl{jobreport}
  \end{shaded}
  \versions{ARC $\geq$ 0.3.34, ARC 0.4, ARC 0.5, ARC 0.6, ARC 0.8}
  \begin{tabular}{lp{13cm}}
    Unique:&yes\\
    Operators:&\verb#=#\\
    User input:&\verb#(jobreport=<URL>)#\\
    GM input:&\verb#-"-#\\
    Example:&\verb#(jobreport="https://grid.uio.no:8001/logger")#\\
  \end{tabular}

  Specifies an URL for a logging service to send reports about job
  to. The default is set up in the cluster configuration.

  \begin{tabular}{llp{10cm}}
    \hspace*{1cm}&\texttt{URL} & URL\\
  \end{tabular}

  It is up to a user to make sure the requested logging service
  accepts reports from the set of clusters she intends to use.

  \hspace*{0.5cm}\phantomsection
  \addcontentsline{toc}{subsection}{credentialserver}\index{XRSL:attribute:credentialserver}
  \begin{shaded}
    \xrsl{credentialserver}
  \end{shaded}
  \versions{ARC $\geq$ 0.5.45, ARC 0.6, ARC 0.8}
  \begin{tabular}{lp{13cm}}
    Unique:&yes\\
    Operators:&\verb#=#\\
    User input:&\verb#(credentialserver=<URL>)#\\
    GM input:&\verb#-"-#\\
    Example:&\verb#(credentialserver="myproxy://myproxy.nordugrid.org;username=user")#\\
  \end{tabular}

  Specifies an URL which Grid Manager may contact to renew/extend delegated
  proxy of job. Only MyProxy servers are supported.

  \begin{tabular}{llp{10cm}}
    \hspace*{1cm}&\texttt{URL} & URL of MyProxy server\\
  \end{tabular}

  It is up to a user to make sure the specified MyProxy server will accept
  requests from Grid Manager to renew expired credentials.
  \texttt{URL} may contain options \texttt{username} and \texttt{credname}
  to specify user name and credentials name which Grid Manager should pass
  to MyProxy server. If \texttt{username} is not specified DN of user 
  credentials is used instead.

%  
% To uncomment when implemented in clients; more info needed
%
%   \hspace*{0.5cm}\phantomsection
%   \addcontentsline{toc}{subsection}{sessiondirectorytype}\index{XRSL:attribute:sessiondirectorytype}
%   \begin{shaded}
%     \xrsl{sessiondirectorytype}
%   \end{shaded}
%   \versions{not supported yet}
%   \begin{tabular}{lp{13cm}}
%    Unique:&yes\\
%    Operators:&\verb#=#\\
%     User input:&\verb#(sessiondirectorytype="readonly"|"limited"|"internal"})#\\
%     GM input:&\verb#-"-#\\
%     Example:&\verb#(sessiondirectorytype="limited")#\\
%   \end{tabular}
% 
%   Instructs Grid Manager to set specific access permissions for the session directory. Normally, session directory is read-only for external processes. In case of interactive jobs, however, it may be necessary to allow writing from an external client.
% 
%   \begin{tabular}{llp{10cm}}
%     \hspace*{1cm}&\texttt{readonly} & Default value, session directory is read-only\\
%     \hspace*{1cm}&\texttt{limited} & \\
%     \hspace*{1cm}&\texttt{internal} & \\
%   \end{tabular}
