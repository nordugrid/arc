\documentclass{article}
\usepackage{graphicx}                              %for PNG images (pdflatex)
%\usepackage{graphics}                              %for EPS images (latex)
\usepackage[linkbordercolor={1.0 1.0 0.0}]{hyperref} %for \url tag
%~ \usepackage{color}                                 %for defining custom colors
\usepackage{framed}                                %for shaded and framed paragraphs
\usepackage{textcomp}                              %for various symbols, e.g. Registered Mark
\usepackage{geometry}                              %for defining page size
\usepackage{array}
\usepackage{booktabs}
\usepackage{ltablex}
\usepackage[english]{babel}
\usepackage{multirow}
\usepackage[table]{xcolor} % Note: The rowcolors command does not work properly with the tabularx environment.


\colorlet{tableheadcolor}{gray!25}
\colorlet{tablerowcolor}{gray!12.5}

\newcolumntype{C}{>{\ttfamily\footnotesize}c}
\newcolumntype{T}{>{\ttfamily\footnotesize\centering\arraybackslash}X}

\newcommand{\TODO}{\normalsize{\textnormal{TODO}}}
\newcommand{\NM}{\normalsize{\textnormal{Not mapped}}}
\newcommand{\NST}[1]{\normalsize{\textnormal{#1}}}
\newcommand{\defineheadfoot}{\endfirsthead %
\midrule %
\multicolumn{3}{c}{Continued from previous page}\\%
\midrule %
\endhead %
\midrule %
\multicolumn{3}{c}{Continues at next page}\\%
\midrule %
\endfoot %
\bottomrule %
\endlastfoot}


\renewcommand{\tabularxcolumn}[1]{m{#1}}

\geometry{verbose,a4paper,tmargin=2.5cm,bmargin=2.5cm,lmargin=2.5cm,rmargin=2cm}

\hypersetup{
  pdfauthor = {Author Name},
  pdftitle = {Paper title},
  pdfsubject = {Paper subject},
  pdfkeywords = {Paper,keyword,comma-separated},
  pdfcreator = {PDFLaTeX with hyperref package},
  pdfproducer = {PDFLaTeX}
}
%
\usepackage[numbers]{natbib}
\bibliographystyle{plainnat}
%
\def\efill{\hfill\nopagebreak}%
\hyphenation{Nordu-Grid}
\setlength{\parindent}{0cm}
\setlength{\FrameRule}{1pt}
\setlength{\FrameSep}{8pt}
\addtolength{\parskip}{5pt}
\renewcommand{\thefootnote}{\fnsymbol{footnote}}
\renewcommand{\arraystretch}{1.3}
\newcommand{\dothis}{\colorbox{shadecolor}}
\newcommand{\globus}{Globus Toolkit\textsuperscript{\textregistered}~2~}
\newcommand{\GT}{Globus Toolkit\textsuperscript{\textregistered}}
\newcommand{\ngdl}{\url{http://ftp.nordugrid.org/download}~}
\definecolor{shadecolor}{rgb}{1,1,0.6}
\definecolor{salmon}{rgb}{1,0.9,1}
\definecolor{bordeaux}{rgb}{0.75,0.,0.}
\definecolor{cyan}{rgb}{0,1,1}

\newcommand{\todo}{TODO}

\newcommand{\same}{{\textquotesingle}{\textquotesingle}}
\newcommand{\subsubsubsection}[1]{\paragraph{#1}}
\newcommand{\subsubsubsubsection}[1]{\subparagraph{#1}}

\newenvironment{inouttabular}%
{\begin{center}\begin{tabular}{l>{\ttfamily\footnotesize}c>{\ttfamily\footnotesize}c}%
\toprule
& \textnormal{\normalsize{In}} & \textnormal{\normalsize{Out}}\\ \cmidrule{2-3}}
{\bottomrule\end{tabular}\end{center}}

\makeatletter
\newcommand\arraybslash{\let\\\@arraycr}
\makeatother

%
%----- DON'T CHANGE HEADER MATTER
\begin{document}
\def\today{\number\day/\number\month/\number\year}

\begin{titlepage}

\begin{tabular}{rl}
\resizebox*{3cm}{!}{\includegraphics{ng-logo.png}}
&\parbox[b]{2cm}{\textbf \it {\hspace*{-1.5cm}NORDUGRID\vspace*{0.5cm}}}
\end{tabular}

\hrulefill

%-------- Change this to NORDUGRID-XXXXXXX-NN

\raggedleft {NORDUGRID-XXXXXXX-NN\par}

{\raggedleft \today\par}

\vspace*{2cm}

%%%%---- The title ----
{\centering \textsc{\Large The ARC Job Description}\Large \par}
\vspace*{0.5cm}

%%%%---- A subtitle, if necessary ----
{\centering \textit{\large Internal Representation Mapping}\large \par}

\vspace*{1.5cm}
%%%%---- A list of authors ----
    {\centering \large M. S. Andersen\footnote{skou@nbi.dk} \large \par}

%%%%---- An abstract - if style is article ----
%\begin{abstract}
%The abstract
%\end{abstract}
\end{titlepage}

\tableofcontents                          %Comment if use article style
\newpage

\section{Introduction}
This document contains information about how XRSL, JDL and the various
supported JSDL flavours are mapped to and from the job description
internal representation in ARC{}-lib.

\section{Data Structure}
The internal representation of the job description is contained in the
JobDescription class, which consist of the following objects:
JobIdentification, Application, Resources, DataStaging and JobMetaData.
JobIdentification contains data used to identify the job description
through various types, tags and names. The Application object is used
to describe explicitly the executable which should be executed at the
Computing Endpoint (CE), environment variables, logging, standard in
and out, credential service, etc. The Resources class contains
information about requirements and the type of execution node preferred
by the user. The DataStaging class contains information about input
files and output files/directories created by the job. Finally the job
description may contain meta data, describing the context in which the
description was generated, and it is stored in the JobMetaData class.

\subsection{JobIdentification}
\subsubsection{JobName}
\begin{table}[h]
\begin{inouttabular}
XRSL & jobname & jobname\\
JSDL & JobIdentification/JobName & JobIdentification/JobName\\
ARC-JSDL & JobIdentification/JobName & JobIdentification/JobName \\
\end{inouttabular}
\end{table}

\subsubsection{Description}
\begin{table}[h]
\begin{inouttabular}
ARC-JSDL & JobIdentification/Description & JobIdentification/Description\\
\end{inouttabular}
\end{table}

\subsubsection{JobType}
\begin{inouttabular}
ARC-JSDL & JobIdentification/JobType & JobIdentification/JobType\\
\end{inouttabular}
JDL has a jobtype and a type attribute, however a mapping has not been
made.

\subsubsection{UserTag}
\begin{inouttabular}
JDL & usertags & usertags\\
ARC-JSDL & JobIdentification/UserTag & JobIdentification/UserTag\\
\end{inouttabular}

\subsection{JobVOName}
\begin{inouttabular}
XRSL & jobname & jobname\\
ARC-JSDL & JobIdentification/JobVOName & JobIdentification/JobVOName\\
\end{inouttabular}

\subsection{Application}
\subsubsection{Executable}
\begin{inouttabular}
XRSL & executable, arguments & executable, arguments\\
JDL & Executable, Arguments & Executable, Arguments\\
\multirow{2}{*}{Posix-JSDL} & Application/POSIXApplication/Executable, & Application/POSIXApplication/Executable,\\
&  Application/POSIXApplication/Arguments & Application/POSIXApplication/Arguments\\
\multirow{2}{*}{HPCP-JSDL} & Application/HPCProfileApplication/Executable, & Application/HPCProfileApplication/Executable,\\
&  Application/HPCProfileApplication/Arguments & Application/HPCProfileApplication/Arguments\\
\end{inouttabular}
The Executable data member is a composite structure of a string and a
list of strings, where the string holds the executable and the list
holding the arguments.

\subsubsection{Input}
\begin{inouttabular}
XRSL & stdin & stdin, inputFiles\\
JDL & StdInput & StdInput, InputSandbox\\
Posix-JSDL & Application/POSIXApplication/Input & Application/POSIXApplication/Input\\
HPCP-JSDL & Application/HPCProfileApplication/Input & Application/HPCProfileApplication/Input\\
\end{inouttabular}
If the Input data member does not exist in the file list, it will be
added to it.

\subsubsection{Output}
\begin{inouttabular}
XRSL & stdout & stdout, outputFiles\\
JDL & StdOutput & StdOutput, OutputSandbox, OutputSandboxDestURI\\
Posix-JSDL & Application/POSIXApplication/Output & Application/POSIXApplication/Output\\
HPCP-JSDL & Application/HPCProfileApplication/Output & Application/HPCProfileApplication/Output\\
\end{inouttabular}
If the Output data member does not exist in the File
list, it will be added to it.

\subsubsection{Error}
\begin{inouttabular}
XRSL & stderr & stderr, outputFiles\\
JDL & StdError & StdError, OutputSandbox, OutputSandboxDestURI\\
Posix-JSDL & Application/POSIXApplication/Error & Application/POSIXApplication/Error\\
HPCP-JSDL & Application/HPCProfileApplication/Error & Application/HPCProfileApplication/Error\\
\end{inouttabular}
If the Error data member does not exist in the File list, it will be
added to it.

\subsubsection{Join}
\begin{inouttabular}
XRSL & join & \NM\\
ARC-JSDL & Application/Join & Application/Join\\
\end{inouttabular}


\subsubsection{Environment}
\subsubsubsection{Name}
\begin{inouttabular}
XRSL & environment* & environment*\\
JDL & Environment** & Environment**\\
Posix-JSDL & Application/POSIXApplication/Environment.name & Application/POSIXApplication/Environment.name\\
HPCP-JSDL & Application/HPCProfileApplication/Environment.name & Application/HPCProfileApplication/Environment.name\\
\end{inouttabular}
The XRSL attribute environment is a list of
pairs. The first element of each pair is mapped to the Name data member.
The JDL attribute Environment is a list of equalities.

\subsubsubsection{Value}
\begin{inouttabular}
XRSL & environment* & environment*\\
JDL & Environment** & Environment**\\
JSDL & No matching attribute & Not mapped\\
Posix-JSDL & Application/POSIXApplication/Environment & Application/POSIXApplication/Environment\\
HPCP-JSDL & Application/HPCProfileApplication/Environment & Applicaion/HPCProfileApplication/Environment\\
\end{inouttabular}

\subsubsection{Prologue}
\begin{inouttabular}
JDL & Prologue, PrologueArgument & Prologue, PrologueArgument\\
\multirow{2}{*}{ARC-JSDL} & Application/Prologue/Path, & Application/Prologue/Path,\\
& Application/Prologue/Argument & Application/Prologue/Argument\\
\end{inouttabular}

\subsubsection{Epilogue}
\begin{inouttabular}
JDL & Epilogue, EpilogueArgument & Epilogue, EpilogueArgument\\
\multirow{2}{*}{ARC-JSDL} & Application/Epilogue/Path, & Application/Epilogue/Path,\\
& Application/Epilogue/Argument & Application/Epilogue/Argument\\
\end{inouttabular}

\subsubsection{LogDir}
\begin{inouttabular}
XRSL & gmlog & gmlog, outputFiles\\
ARC-JSDL & Application/LogDir & Application/LogDir\\
\end{inouttabular}

\subsection{RemoteLogging}
\begin{inouttabular}
XRSL & jobreport & jobreport\\
ARC-JSDL & Application/RemoteLogging & Application/RemoteLogging\\
\end{inouttabular}

\subsubsection{Rerun}
\begin{inouttabular}
XRSL & rerun & rerun\\
JDL & RetryCount ShallowRetryCount* & RetryCount ShallowRetryCount**\\
ARC-JSDL & Application/Rerun & Application/Rerun\\
\end{inouttabular}
* Only the maximum value between RetryCount
and ShallowRetryCount is stored.

** Both RetryCount and ShallowRetryCount will be set if
Rerun is set, and they will both have the value of Rerun.

\subsubsection{ExpiryTime}
\begin{inouttabular}
JDL & ExpiryTime & ExpiryTime\\
ARC-JSDL & Application/ExpiryTime & Application/ExpiryTime\\
\end{inouttabular}

\subsubsection{ProcessingStartTime}
\begin{inouttabular}
ARC-JSDL & Application/ProcessingStartTime & Application/ProcessingStartTime\\
\end{inouttabular}

\subsubsection{Notification}
\begin{inouttabular}
XRSL & notify & notify\\
ARC-JSDL & Application/Notification & Application/Notification\\
\end{inouttabular}

\subsubsection{CredentialService}
\begin{inouttabular}
XRSL & credentialserver & credentialserver\\
JDL & MyProxyServer & MyProxyServer\\
ARC-JSDL & Application/CredentialService & Application/CredentialService\\
\end{inouttabular}

\subsubsection{AccessControl}
\begin{inouttabular}
XRSL & acl & acl\\
ARC-JSDL & Application/AccessControl & Application/AccessControl\\
\end{inouttabular}

\subsection{Resources}
\subsubsection{OperatingSystem}
\begin{inouttabular}
XRSL & opsys & opsys\\
ARC-JSDL & Resources/OperatingSystem & Resources/OperatingSystem\\
\end{inouttabular}

\subsubsection{Platform}
\begin{inouttabular}
XRSL & architecture & architecture\\
JSDL & Resources/CPUAchitecture/CPUArchitectureName & Resources/CPUAchitecture/CPUArchitectureName\\
ARC-JSDL & Resources/Platform & Resources/Platform\\
\end{inouttabular}

\subsubsection{NetworkInfo}
\begin{inouttabular}
JSDL & Resources/IndividualNetworkBandwidth & Resources/IndividualNetworkBandwidth \\
ARC-JSDL & Resources/NetworkInfo & \NM\\
\end{inouttabular}

\subsubsection{NodeAccess}
\begin{inouttabular}
XRSL & nodeAccess & nodeAccess\\
ARC-JSDL & Resources/NodeAccess & Resources/NodeAccess\\
\end{inouttabular}

\subsubsection{IndividualPhysicalMemory}
\begin{inouttabular}
XRSL & memory & memory\\
JSDL & Resources/IndividualPhysicalMemory & Resources/IndividualPhysicalMemory\\
Posix-JSDL & Application/POSIXApplication/MemoryLimit & Application/POSIXApplication/MemoryLimit\\
ARC-JSDL & Resources/IndividualPhysicalMemory & Resources/IndividualPhysicalMemory\\
\end{inouttabular}

\subsubsection{IndividualVirtualMemory}
\begin{inouttabular}
JSDL & Resources/IndividualVirtualMemory & Resources/IndividualVirtualMemory\\
Posix-JSDL & Application/POSIXApplication/VirtualMemoryLimit & Application/POSIXApplication/VirtualMemoryLimit\\
ARC-JSDL & Resources/IndividualVirtualMemory & Resources/IndividualVirtualMemory\\
\end{inouttabular}

\subsubsection{DiskSpaceRequirement}
\subsubsubsection{DiskSpace}
\begin{inouttabular}
XRSL & disk & disk\\
JSDL & Resources/FileSystem/DiskSpace & Resources/FileSystem/DiskSpace\\
ARC-JSDL & Resources/DiskSpaceRequirement/DiskSpace & Resources/DiskSpaceRequirement/DiskSpace\\
\end{inouttabular}

\subsubsubsection{CacheDiskSpace}
\begin{inouttabular}
ARC-JSDL & Resources/CacheDiskSpace & Resources/CacheDiskSpace\\
\end{inouttabular}

\subsubsubsection{SessionDiskSpace}
\begin{inouttabular}
ARC-JSDL & Resources/SessionDiskSpace & Resources/SessionDiskSpace\\
\end{inouttabular}

\subsubsection{SessionLifetime}
\begin{inouttabular}
XRSL & lifeTime & lifeTime\\
ARC-JSDL & Resources/SessionLifeTime & Resources/SessionLifeTime\\
\end{inouttabular}

\subsubsection{IndividualCPUTime}
\begin{inouttabular}
JSDL & Resources/IndividualCPUTime* & Resources/IndividualCPUTime\\
ARC-JSDL & Resources/IndividualCPUTime & Resources/IndividualCPUTime\\
\end{inouttabular}
* The JSDL element is mapped to the Range part of
IndividualCPUTime.

\subsubsection{TotalCPUTime}
\begin{inouttabular}
XRSL & cpuTime, gridtime, benchmark & cpuTime\\
JSDL & Resources/TotalCPUTime* & Resources/TotalCPUTime\\
Posix-JSDL & Application/POSIXApplication/CPUTimeLimit* & Application/POSIXApplication/CPUTimeLimit\\
ARC-JSDL & Resources/TotalCPUTime & Resources/TotalCPUTime\\
\end{inouttabular}
The \texttt{XRSL} attribute \texttt{cputime} is mapped to the
\texttt{Range} part of \texttt{TotalCPUTime}, where as for the
\texttt{gridtime} attibute the pair ("ARC-gridtime", 1800) will
be added to the benchmark part.
The JSDL and POSIX-JSDL elements is mapped to the Range part of
TotalCPUTime.

\subsubsection{IndividualWallTime}
\begin{inouttabular}
ARC-JSDL & Resources/IndividualWallTime & Resources/IndividualWallTime\\
\end{inouttabular}

\subsubsection{TotalWallTime}
\begin{inouttabular}
XRSL & wallTime & wallTime\\
Posix-JSDL & Application/POSIXApplication/WallTimeLimit & Application/POSIXApplication/WallTimeLimit\\
ARC-JSDL & Resources/TotalWallTime & Resources/TotalWallTime\\
\end{inouttabular}

\subsubsection{CEType}
\begin{inouttabular}
XRSL & middleware & middleware\\
ARC-JSDL & Application/CEType & Application/CEType\\
\end{inouttabular}

\subsubsection{SlotRequirement}
\subsubsubsection{NumberOfSlots}
\begin{inouttabular}
Posix-JSDL & Application/POSIXApplication/ProcessCountLimit & Application/POSIXApplication/ProcessCountLimit\\
ARC-JSDL & Resources/SlotRequirement/NumberOfSlots & Resources/SlotRequirement/NumberOfSlots\\
\end{inouttabular}

\subsubsubsection{ProcessPerHost}
\begin{inouttabular}
XRSL & count & count\\
JSDL & Resources/TotalCPUCount & Resources/TotalCPUCount\\
ARC-JSDL & Resources/SlotRequirement/ProcessPerHost & Resources/SlotRequirement/ProcessPerHost\\
\end{inouttabular}

\subsubsubsection{ThreadsPerProcesses}
\begin{inouttabular}
Posix-JSDL & Application/POSIXApplication/ThreadCountLimit & Application/POSIXApplication/ThreadCountLimit\\
ARC-JSDL & Resources/SlotRequirement/ThreadsPerProcesses & Resources/SlotRequirement/ThreadsPerProcesses\\
\end{inouttabular}

\subsubsubsection{SPMDVariation}
\begin{inouttabular}
ARC-JSDL & Resources/SlotRequirement/SPMDVariation & Resources/SlotRequirement/SPMDVariation\\
\end{inouttabular}

\subsubsection{CandidateTarget}
\subsubsubsection{EndPointURL}
\begin{inouttabular}
XRSL & cluster & \NM\\
JSDL & Resources/CandidateHosts/HostName & \NM\\
ARC-JSDL & Resources/CandidateTarget/EndPointURL & Resources/CandidateTarget/EndPointURL\\
\end{inouttabular}

\subsubsubsection{QueueName}
\begin{inouttabular}
XRSL & queue & queue\\
JDL & QueueName & QueueName\\
ARC-JSDL & Resources/CandidateTarget/QueueName & Resources/CandidateTarget/QueueName\\
\end{inouttabular}

\subsubsection{RunTimeEnvironment}
\begin{inouttabular}
XRSL & runTimeEnvironment & runTimeEnvironment\\
ARC-JSDL & Application/RunTimeEnvironment & Application/RunTimeEnvironment\\
\end{inouttabular}

\subsection{DataStaging}
\subsubsection{File}
\subsubsubsection{Name}
\begin{inouttabular}
XRSL & inputFiles, outputFiles & inputFiles, outputFiles\\
JDL & InputSandbox, OutputSandbox & InputSandbox, OutputSandbox\\
JSDL & DataStaging/File/FileName & DataStaging/File/FileName\\
\end{inouttabular}

\subsubsubsection{Source}
\subsubsubsubsection{URI}
\begin{inouttabular}
XRSL & inputFiles & inputFiles\\
JDL & InputSandbox, InputSandboxURI & InputSandbox, InputSandboxURI\\
JSDL & DataStaging/File/Source/URI & DataStaging/File/Source/URI\\
\end{inouttabular}

\subsubsubsubsection{Threads}
\begin{inouttabular}
XRSL & ftpThreads & ftpThreads\\
%~ ARC-JSDL & &\\
\end{inouttabular}

%~ \subsubsubsubsection{DataIndexingService}
%~ \begin{inouttabular}
%~ XRSL & &\\
%~ JDL & &\\
%~ JSDL & &\\
%~ Posix-JSDL & &\\
%~ HPCP-JSDL & &\\
%~ ARC-JSDL & &\\
%~ \end{inouttabular}

\subsubsubsection{Target}
\subsubsubsubsection{URI}
\begin{inouttabular}
XRSL & outputFiles & outputFiles\\
JDL & OutputSandbox, OutputSandboxDestURI & OutputSandbox, OutputSandboxDestURI\\
JSDL & DataStaging/File/Target/URI & DataStaging/File/Target/URI\\
\end{inouttabular}

\subsubsubsubsection{Threads}
\begin{inouttabular}
XRSL & ftpThreads &ftpThreads\\
%~ ARC-JSDL & &\\
\end{inouttabular}

%~ \subsubsubsubsection{Mandatory}
%~ \begin{inouttabular}
%~ XRSL & &\\
%~ JDL & &\\
%~ JSDL & &\\
%~ Posix-JSDL & &\\
%~ HPCP-JSDL & &\\
%~ ARC-JSDL & &\\
%~ \end{inouttabular}

%~ \subsubsubsubsection{DataIndexingService}
%~ \begin{inouttabular}
%~ XRSL & &\\
%~ JDL & &\\
%~ JSDL & &\\
%~ Posix-JSDL & &\\
%~ HPCP-JSDL & &\\
%~ ARC-JSDL & &\\
%~ \end{inouttabular}

%~ \subsubsubsubsection{NeededReplica}
%~ \begin{inouttabular}
%~ XRSL & &\\
%~ JDL & &\\
%~ JSDL & &\\
%~ Posix-JSDL & &\\
%~ HPCP-JSDL & &\\
%~ ARC-JSDL & &\\
%~ \end{inouttabular}

\subsubsubsection{KeepData}
\begin{inouttabular}
JDL & OutputSandboxDestURI & OutputSandboxDestURI\\
JSDL & DataStaging/File/DeleteOnTermination & DataStaging/File/DeleteOnTermination\\
\end{inouttabular}

\subsubsubsection{IsExecutable}
\begin{inouttabular}
XRSL & executables & executables\\
ARC-JSDL & DataStaging/File/IsExecutable & DataStaging/File/IsExecutable\\
\end{inouttabular}

\subsubsubsection{DownloadToCache}
\begin{inouttabular}
XRSL & cache & \NST{cache in URL-option}\\
ARC-JSDL & DataStaging/File/DownloadToCache & DataStaging/File/DownloadToCache\\
\end{inouttabular}

%~ \subsubsection{Directory}

\subsection{JobMetaData}
%~ \subsubsection{Author}
%~ \begin{inouttabular}
%~ XRSL & &\\
%~ JDL & &\\
%~ JSDL & &\\
%~ Posix-JSDL & &\\
%~ HPCP-JSDL & &\\
%~ ARC-JSDL & &\\
%~ \end{inouttabular}
%~
%~ \subsubsection{DocumentExpiration}
%~ \begin{inouttabular}
%~ XRSL & &\\
%~ JDL & &\\
%~ JSDL & &\\
%~ Posix-JSDL & &\\
%~ HPCP-JSDL & &\\
%~ ARC-JSDL & &\\
%~ \end{inouttabular}

\subsubsection{Rank}
\begin{inouttabular}
JDL & Rank & Rank\\
%~ ARC-JSDL & &\\
\end{inouttabular}

\subsubsection{FuzzyRank}
\begin{inouttabular}
JDL & FuzzyRank & FuzzyRank\\
%~ ARC-JSDL & & \\
\end{inouttabular}

\section{XRSL mapping}
\begin{center}
\rowcolors{3}{tablerowcolor}{}
\begin{tabularx}{400pt}{CTC}
\toprule
\NST{XRSL input} & \NST{Internal representaion} & \NST{XRSL output}\\
\midrule
\defineheadfoot
executable & Application.Executable.Name & executable\\
arguments & Application.Executable.Arguments & arguments\\
inputFiles& DataStaging.File.Name,\linebreak DataStaging.File.Source & inputFiles\\
executables & DataStaging.File.IsExecutable & executables\\
cache & DataStaging.File.DownloadToCache & \NST{cache in URL-option} \\
outputFiles & DataStaging.File.Name,\linebreak DataStaging.File.Target & outputFiles \\
cpuTime & Resources.TotalCPUTime & cpuTime\\
wallTime & Resources.TotalWallTime & wallTime\\
gridTime & Resources.TotalCPUTime & \NM \\
benchmarks & Resources.TotalCPUTime & \NM \\
memory & Resources.IndividualPhysicalMemory & memory\\
disk & Resources.DiskSpaceRequirement.DiskSpace & disk\\
runTimeEnvironment & Resources.RunTimeEnvironment & runTimeEnvironment\\
middleware & Resources.CEType & middleware\\
opsys & Resources.OperatingSystem & opsys\\
stdin & Application.Input & stdin\\
stdout & Application.Output & stdout, outputFiles\\
stderr & Application.Error & stderr, outputFiles\\
join & Application.Join & \NM\\
gmlog & Application.LogDir & gmlog\\
jobName & Identification.JobName & jobName\\
ftpThreads & DataStaging.File.Source.Threads,\linebreak DataStaging.File.Target.Threads & \NM\\
acl & Application.AccessControl & acl\\
cluster & Resources.CandidateTarget.EndPointURL & \NM\\
queue & Resources.CandidateTarget.QueueName & queue\\
startTime & Application.ProcessingStartTime & \NM\\
lifeTime & Resources.SessionLifeTime & lifeTime\\
notify & Application.Notification & notify\\
replicaCollection & \NM & \NM\\
rerun & Application.Rerun & rerun\\
architecture & Resources.Platform & architecture\\
nodeAccess & Resources.NodeAccess & \NM\\
dryRun & \NM & \NM\\
rsl\_substitution & \NST{Processed internally by the parser} & \NM\\
environment & Application.Environment & environment\\
count & Resources.SlotRequirement.ProcessPerHost & count\\
jobreport & Application.RemoteLogging & jobreport\\
credentialserver & Application.CredentialService & credentialserver\\
\end{tabularx}
\end{center}

\section{JDL mapping}
\begin{center}
\rowcolors{3}{}{tablerowcolor}
\begin{tabularx}{\textwidth}{CTT}
\toprule
\rowcolor{white}
\NST{JDL input} & \NST{Internal representaion} & \NST{JDL output}\\
\midrule
\defineheadfoot
JobType & \NM & \NM\\
Executable & Application.Executable.Name & Executable, InputSandbox\\
Arguments & Application.Executable.Arguments & Arguments\\
StdInput & Application.Input & StdInput,\linebreak InputSandbox\\
StdOutput & Application.Output & StdOutput,\linebreak OutputSandbox,\linebreak OutputSandboxDestURI\\
StdError & Application.Error & StdError,\linebreak OutputSandbox,\linebreak OutputSandboxDestURI\\
InputSandbox & DataStaging.File.Name,\linebreak DataStaging.File.Source & InputSandbox\\
InputSandboxBaseURI & DataStaging.File.Source.URI & InputSandbox\\
OutputSandbox & DataStaging.File.Name,\linebreak DataStaging.File.Target.URI & OutputSandbox,\linebreak OutputSandboxDestURI\\
OutputSandboxDestURI & DataStaging.File.Target,\linebreak DataStaging.File.KeepData & OutputSandbox,\linebreak OutputSandboxDestURI\\
Prologue & Application.Prologue.Name & Prologue\\
PrologueArguments & Application.Prologue.Arguments & PrologueArguments\\
Epilogue & Application.Epilogue.Name & Epilogue\\
EpilogueArguments & Application.Epilogue.Arguments & EpilogueArguments\\
AllowZippedISB & Stored internally, not processed & AllowZippedISB\\
ZippedISB & Stored internally, not processed & ZippedISB\\
ExpiryTime & Application.ExpiryTime & ExpiryTime\\
Environment & Application.Environment & Environment\\
PerusalFileEnable & \NST{Stored internally, not processed} & PerusalFileEnable\\
PerusalTimeInterval & \NST{Stored internally, not processed} & PerusalTimeInterval\\
PerusalFilesDestURI & \NST{Stored internally, not processed} & PerusalFilesDestURI\\
InputData \NST{(Deprecated)} & \NM & \NM\\
OutputData \NST{(Deprecated)} & \NM & \NM\\
StorageIndex \NST{(Deprecated)} & \NM & \NM\\
DataCatalog \NST{(Deprecated)} & \NM & \NM\\
DataRequirements & \NST{Stored internally, not processed} & DataRequirements\\
DataAccessProtocol & \NST{Stored internally, not processed} & DataAccessProtocol\\
OutputSE & \NST{Stored internally, not processed} & OutputSE\\
VirtualOrganisation & Identification.JobVOName & VirtualOrganisation\\
RetryCount & Application.Rerun & RetryCount ShallowRetryCount\\
ShallowRetryCount & Application.Rerun & RetryCount ShallowRetryCount\\
LBAddress & \NST{Stored internally, not processed} & LBAddress\\
MyProxyServer & Application.CredentialService & MyProxyServer\\
HLRLocation & \NST{Stored internally, not processed} & HLRLocation\\
JobProvenance & \NST{Stored internally, not processed} & JobProvenance\\
NodeNumber & \NST{Stored internally, not processed} & NodeNumber\\
JobSteps \NST{(Deprecated)} & \NM & \NM\\
CurrentStep \NST{(Deprecated)} & \NM & \NM\\
JobState \NST{(Deprecated)} & \NM & \NM\\
ListenerPort & \NST{Stored internally, not processed} & ListenerPort\\
ListenerHost & \NST{Stored internally, not processed} & ListenerHost\\
ListenerPipeName & \NST{Stored internally, not processed} & ListenerPipeName\\
Requirements & \NST{Stored internally, not processed} & Requirements\\
Rank & JobMeta.Rank & Rank\\
FuzzyRank & JobMeta.FuzzyRank & FuzzyRank\\
UserTags & Identification.UserTag & UserTags\\
BatchSystem & \NST{Stored internally, not processed} & BatchSystem\\
QueueName & Resources.CandidateTarget.QueueName & QueueName
\end{tabularx}
\end{center}

\section{JSDL mapping}
\begin{center}
\rowcolors{3}{}{tablerowcolor}
\begin{tabularx}{\textwidth}{TCT}
\toprule
\NST{JSDL input} & \NST{Internal representaion} & \NST{JSDL output}\\
\midrule
\defineheadfoot
\multicolumn{3}{c}{\bfseries JobIdentification}\\
\midrule
JobName & Application.JobName & JobName\\
JobAnnotation & \NM & \NM\\
JobProject & \NM & \NM\\
\midrule
\multicolumn{3}{c}{\bfseries Application}\\
\midrule
ApplicationName & \NM & \NM\\
ApplicationVersion & \NM & \NM\\
\midrule
\multicolumn{3}{c}{\bfseries Resources}\\
\midrule
CandidateHosts/HostName & Resources.CandidateTarget.EndPointURL & \NM\\
FileSystem/FileSystemType & \NM & \NM\\
FileSystem/Description & \NM & \NM\\
FileSystem/MountPoint & \NM & \NM\\
FileSystem/DiskSpace & Resources.DiskSpaceRequirement.DiskSpace & FileSystem/DiskSpace\\
ExclusiveExecution & \NM & \NM\\
OperatingSystem/\linebreak OperatingSystemType/\linebreak OperatingSystemName & Resources.OperatingSystem & OperatingSystem/\linebreak OperatingSystemType/\linebreak OperatingSystemName\\
OperatingSystem/\linebreak OperatingSystemVersion & Resources.OperatingSystem & OperatingSystem/\linebreak OperatingSystemVersion\\
CPUArchitecture/\linebreak CPUArchitectureName & Resources.Platform & CPUAchitecture/\linebreak CPUAchitectureName\\
IndividualCPUSpeed & \NM & \NM\\
IndividualCPUTime & job.Resources.IndividualCPUTime & IndividualCPUTime\\
IndividualCPUCount & \NM & \NM\\
IndividualNetworkBandwidth & Resources.NetworkInfo & IndividualNetworkBandwidth\\
IndividualPhysicalMemory & job.Resources.IndividualPhysicalMemory & IndividualPhysicalMemory\\
IndividualVirtualMemory & job.Resources.IndividualVirtualMemory & IndividualVirtualMemory\\
IndividualDiskSpace & \NM & \NM\\
TotalCPUTime & job.Resources.TotalCPUTime.range & TotalCPUTime\\
TotalCPUCount & job.Resources.SlotRequirement.ProcessPerHost & TotalCPUCount\\
TotalPhysicalMemory & \NM & \NM\\
TotalVirtualMemory & \NM & \NM\\
TotalDiskSpace & \NM & \NM\\
TotalResourceCount & \NM & \NM\\
\midrule
\multicolumn{3}{c}{\bfseries DataStaging}\\
\midrule
FileName & DataStaging.File.FileName & FileName\\
FileSystemName & \NM & \NM\\
CreationFlag & \NM & \NM\\
DeleteOnTermination & DataStaging.File.KeepData & DeleteOnTermination\\
Source/URI & DataStaging.File.Source.URI & Source/URI\\
Target/URI & DataStaging.File.Target.URI & Target/URI\\
\end{tabularx}
\end{center}

\section{POSIX-JSDL mapping}
\begin{center}
\rowcolors{3}{}{tablerowcolor}
\begin{tabularx}{\textwidth}{CCT}
\toprule
\NST{POSIX-JSDL input} & \NST{Internal representation} & \NST{POSIX-JSDL output}\\
\midrule
\defineheadfoot
Executable & Application.Executable.Name & Executable,\linebreak DataStaging.File.Name,\linebreak DataStaging.File.Source\\
Argument & Application.Executable.Arguments & Argument\\
Input & Application.Input & Input,\linebreak DataStaging.File.Name,\linebreak DataStaging.File.Source\\
Output & Application.Output & Output,\linebreak DataStaging.File.Name,\linebreak DataStaging.File.Target\\
Error & Application.Error & Error,\linebreak DataStaging.File.Name,\linebreak DataStaging.File.Target\\
WorkingDirectory & \NM & \NM\\
Environment & Application.Environment & Environment\\
WallTimeLimit & job.Resources.TotalWallTime.range.max & WallTimeLimit\\
FileSizeLimit & \NM & \NM\\
CoreDumpLimit & \NM & \NM\\
DataSegmentLimit & \NM & \NM\\
LockedMemoryLimit & \NM & \NM\\
OpenDescriptorsLimit & \NM & \NM\\
PipeSizeLimit & \NM & \NM\\
StackSizeLimit & \NM & \NM\\
CPUTimeLimit & job.Resources.TotalCPUTime.range.max & CPUTimeLimit\\
ProcessCountLimit & job.Resources.SlotRequirement.NumberOfSlots.max & ProcessCountLimit\\
VirtualMemoryLimit & job.Resources.IndividualVirtualMemory.max & VirtualMemoryLimit\\
ThreadCountLimit & job.Resources.SlotRequirement.ThreadsPerProcesses.max & ThreadCountLimit\\
UserName & \NM & \NM\\
GroupName & \NM & \NM\\
\end{tabularx}
\end{center}

\section{HPCProfile-JSDL mapping}
\begin{center}
\rowcolors{3}{tablerowcolor}{}
\begin{tabularx}{\textwidth}{CCT}
\toprule
\NST{HPCProfile-JSDL input}  & \NST{Internal representaion} & \NST{HPCProfile-JSDL output}\\
\midrule
\defineheadfoot
Executable & Application.Executable.Name & Executable,\linebreak DataStaging.File.Name,\linebreak DataStaging.File.Source\\
Argument & Application.Executable.Arguments & Argument\\
Input & Application.Input & Input,\linebreak DataStaging.File.Name,\linebreak DataStaging.File.Source\\
Output & Application.Output & Output,\linebreak DataStaging.File.Name,\linebreak DataStaging.File.Target\\
Error & Application.Error & Error,\linebreak DataStaging.File.Name,\linebreak DataStaging.File.Target\\
WorkingDirectory & \NM & \NM \\
Environment & Application.Environment & Environment\\
UserName & \NM & \NM\\
\end{tabularx}
\end{center}
\end{document}
